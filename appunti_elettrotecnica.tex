\documentclass[a4paper]{report}
\usepackage[italian]{babel}
\usepackage[latin1]{inputenc}
\usepackage[T1]{fontenc}
\usepackage[pdftex]{graphicx}
\usepackage[pdftex,colorlinks=true]{hyperref}
\hypersetup{linkcolor=blue}
\usepackage{amssymb}
\usepackage{verbatim}
\usepackage{textcomp}
\usepackage{dashrule}

\hypersetup{ colorlinks,
  linkcolor=blue,
  filecolor=green,
  urlcolor=blue,
  citecolor=blue }

\newcommand{\HRule}{\rule{\linewidth}{0.5mm}}

\usepackage[normalem]{ulem}
\usepackage{graphicx}
\usepackage{amssymb}
\usepackage{amsmath}
\usepackage{amsthm}
\usepackage{mathrsfs}
\usepackage{amsfonts}
\usepackage{amsthm}
\renewcommand{\arraystretch}{2}

\newtheorem{definizione}{Teorema}
\newtheorem{prop}{Propriet\`a}

\usepackage{makeidx}
\makeindex

\begin{document}

\begin{titlepage}
 
\begin{center}
 
 
% Upper part of the page
\includegraphics[width=0.15\textwidth]{logo.png}\\[1cm]
 
\textsc{\LARGE Universit\`a degli Studi del Sannio}\\[1.5cm]
 
\textsc{\Large Corso di Elettrotecnica}\\[0.5cm]
 
 
% Title
\HRule \\[0.4cm]
{ \huge \bfseries Appunti}\\ [0.4cm]
 
\HRule \\ [1.5cm]
 
% Author
\begin{minipage}{0.4\textwidth}
\begin{center} \large
\emph{Autore:}\\
Luca \textsc{Pace}
\end{center}
\end{minipage}
 
\vfill
 
% Bottom of the page
{\large \today}
 
\end{center}
 
\end{titlepage}

\setlength{\unitlength}{2cm}
\begin{picture}(1,1)
  \put(0,0){\line(0,1){1}}
  \put(0,0){\line(1,0){1}}
  \put(0,0){\line(1,1){1}}
  \put(0,0){\line(1,2){.5}}
  \put(0,0){\line(1,3){.3333}}
  \put(0,0){\line(1,4){.25}}
  \put(0,0){\line(1,5){.2}}
  \put(0,0){\line(1,6){.1667}}
  \put(0,0){\line(2,1){1}}
  \put(0,0){\line(2,3){.6667}}
  \put(0,0){\line(2,5){.4}}
  \put(0,0){\line(3,1){1}}
  \put(0,0){\line(3,2){1}}
  \put(0,0){\line(3,4){.75}}
  \put(0,0){\line(3,5){.6}}
  \put(0,0){\line(4,1){1}}
  \put(0,0){\line(4,3){1}}
  \put(0,0){\line(4,5){.8}}
  \put(0,0){\line(5,1){1}}
  \put(0,0){\line(5,2){1}}
  \put(0,0){\line(5,3){1}}
  \put(0,0){\line(5,4){1}}
  \put(0,0){\line(5,6){.8333}}
  \put(0,0){\line(6,1){1}}
  \put(0,0){\line(6,5){1}}
\end{picture}

\section*{Prefazione}
\emph{Il presente documento \`e stato scritto con lo scopo di aiutare
  lo studente che deve affrontare l'esame di Elettrotecnica a ripetere
  gli argomenti trattati durante il corso, riassumendo i concetti
  fondamentali del programma. Tale documento NON deve assolutamente
  essere usato in sostituzione dei libri di testo.}
\tableofcontents

\chapter{I Modelli Circuitali}
\section{Potenza ed Energia Elettrica}
L'energia(assorbita o oregata), cos\`i come il lavoro, \`e associata a
un intervallo di tempo. Indichiamo con $\Delta W^{(a)}$ l'energia
assorbita dal bipolo nell'intervallo di tempo $(t,\;t+ \Delta t)$ e
con $\Delta W^{(e)}$ quella erogata $(\Delta W^{(e)}=-\Delta
W^{(a)})$. Possiamo allora definire a valor medio (nell'intervallo
$(t,\;t+\Delta t)$) della potenza elettrica assorbita come:

\begin{equation}\label{eq:potenzaMedia}
<p^{(a)}>=\dfrac{\Delta W^{(a)}}{\Delta t}
\end{equation}

Sotto condiozioni non affatto restrittive $<p^{(a)}>$ tende, per
$\Delta t \rightarrow 0$ ad un valore finito che dipende solo
dall'istante $t$ considerato:

\[
p^{(a)}(t)=\lim_{\Delta t \to 0} \frac{\Delta W^{(a)}}{\Delta t}
\]

Questa \`e, per definizione, la potenza elettrica istantanea assorbita
dal bipolo. Essa si misura in \emph{watt} ($W$) nel Sistema
Internazionale; immediatamente si ottiene $1W=J/s$

L'energia $\Delta W^{(a)}$ assorbita nell'intervallo $(t,\; t+\Delta
t)$ pu\`o essere considerata come differenza tra tutta l'energia
assorbita $W^{(a)}(t+\Delta t)$ nell'intervallo $(t_0,\; t+\Delta t)$
dove $t_0$ \`e un'istante di tempo assegnato, meno quella gi\`a
assorbita sino all'istante $t$, $W^{(a)}(t)$. Allora la
\ref{eq:potenzaMedia} pu\`o essere riscritta:

\[
<p^{(a)}>=\dfrac{W^{(a)}(t+\Delta t)-W^{(a)}(t)}{\Delta t}
\]

Applicando il processo al limite per $\Delta t \rightarrow 0$, si ha:

\begin{equation}\label{equ:potenzaIstantanea}
p^{(a)}(t)=\dfrac{d\;W^{(a)}}{dt}
\end{equation}

La potenza istantanea \`e uguale alla derivata prima dell'energia
elettrica assorbita dallo stesso nel generico intervallo
$(t_0,t)$. Dalla \ref{equ:potenzaIstantanea} segue che l'energia $d\;
W^{(a)}$ assorbita dal bipolo nell'intervallo elementare $(t,\;
t+dt)$ \`e data da $dW^{(a)}=p^{(a)}(t)$. La relazione
\ref{equ:potenzaIstantanea} pu\`o essere riformulata in forma integrale come:

\[
W^{(a)}(t)=W^{(a)}(t_0)+\int_{t_0}^{t}p^{(a)}(t)dt
\]

La potenza elettrica assorbita da un componente con due terminali \`e
con buona approssimazione data da:

\begin{equation}\label{equ:potenzaAssorbita}
p^{(a)}(t)=v(t)i(t)
\end{equation}

dove $v$ ed $i$ sono la tensione e l'intensit\`a di corrente del
componente con versi di riferimento scelti concordamente con la
convenzione dell'utilizzatore. Se invece venisse adottata la
convenzione del generatore l'espressione della potenza assorbita
sarebbe:

\[
p^{(a)}=-v'i
\]

La relazione \ref{equ:potenzaAssorbita} vale esattamente soltanto in
condizioni stazionarie. In condizioni lentamente variabili vale con
buona approssimazione.

%CAPITOLO 5 DEL LIBRO
\chapter{Regime Sinusiodale}
\section{Metodo Simbolico}
Per fissata pulsazione $\omega$ ad ogni funzione sinusoidale \`e
possibile associare un numero complesso $\bar{A}$ secondo la regola
\[
a(t) = A_mcos(\omega t+\alpha)\iff \bar{A} = A_me^{j\alpha}
\]

Essa produce una corrispondenza biunivoca tra l'insieme delle funzioni
sinusoidali di pulsazione assegnata $\omega$ , $\{
a(t)=A_mcos(\omega t+\alpha )\}$ e l'insieme dei fasori $\{ \bar{A} =
A_me^{j\alpha} \}$.
Utilizzando le formule di \emph{Eulero} possiamo riscrivere
\[
\bar{A} = A_mcos(\alpha)+ A_mjsin(\alpha)
\]

Considerazioni:
\begin{itemize}
\item $\alpha = 0$; $\alpha=\pm 2\pi \implies e^{j\alpha} = 1$
\item $\alpha = \pm \pi \implies e^{j\alpha} = -1$
\item $\alpha = \pm \dfrac{\pi}{2} \implies e^{j\alpha} = \pm j$
\item $\alpha = \pm \dfrac{2}{3}\pi \implies e^{j\alpha} = \mp j$
\end{itemize}
Per gli altri valori di $\alpha$, sia la parte reale che la parte
immaginara di $e^{j\alpha}$, sono diverse da $0$.
Quindi la funzione sinusoidale $a(t)$ pu\`o essere espressa in termini
di fasore rappresentativo

\[
a(t)=\Re e\{\bar{A}e^{j\omega t}\}
\]

La corrispondenza biunivoca gode delle seguenti propriet\`a
\begin{itemize}
\item {\bf Unicit\`a}
\item {\bf Linearit\`a}
\item {\bf Derivazione}
\end{itemize}

\subsection{Propriet\`a di Unicit\`a}
Consideriamo le funzioni sinusoidali $a(t) = A_m(\omega t+\alpha)$ e
$b(t) = B_m(\omega t+\beta)$, esse sono uguali se e solo se sono
uguali i corrispondenti fasori rappresentativi $\bar{A} =
A_me^{j\alpha}$ e $\bar{B} = B_me^{j\beta}$ quindi

\[
a(t)=b(t) \Longleftrightarrow \bar{A}=\bar{B}
\]

Ci\`o \`e una diretta conseguenza del fatto che

\[
c(t)=\Re e \{ \bar{C}e^{j\omega t} \}\; \forall\; t \Longleftrightarrow
\bar{C}=0
\]

\subsection{Propriet\`a di Linearit\`a}
Prendiamo in esame la funione sinusoidale

\[
c(t) = k_1a(t)+k_2b(t)
\]

Combinazione lineare delle funzioni sinusoidali $a(t) = A_mcos(\omega
t + \alpha)$ e \\$b(t) = B_mcos(\omega t + \beta)$ dove $k_1$ e $k_2$
sono costanti reali.
Il fasore $\bar{C}$ rappresentativo della funzione sinusoidale $c(t)$
\`e uguale alla stessa combinazione lineare dei fasori $\bar{A} =
A_me^{j\omega}$ e $\bar{B} = B_me^{j\beta}$ rappresentazioni delle
funzioni sinusoidali $a(t)$ e $b(t)$.

\[
\bar{C} = k_1\bar{A}+k_2\bar{B} \implies c(t)=k_1a(t)+k_2b(t)
\Longleftrightarrow \bar{C}=k_1\bar{A}+k_2\bar{B}
\]

\[
\Re e\{(k_1\bar{A}+k_2\bar{B})e^{j\omega t}\} = k_1\cdot \Re e\{\bar{A}e^{j\omega
  t}\}+k_2\cdot \Re e\{\bar{B}e^{j\omega t}\}
\]

\subsection{Propriet\`a di Derivazione}
La derivata prima della funzione sinusoidale

\[
a(t)=A_mcos(\omega t+\alpha)
\]

con pulsazione $\omega$ \`e

\[
\dfrac{d\;a(t)}{dt}=\dfrac{d}{dt}[A_mcos(\omega t+\alpha)]\ = \omega
A_m cos\left(\omega t+\alpha+ \dfrac{\pi}{2}\right)
\]

Sia $\bar{A} = A_me^{j\omega}$ il fasore rappresentativo della
funzione sinusoidale $a(t)$, allora il fasore rappresentativo della
derivata prima di $a(t)$, che indicheremo con $D_a$, \`e dato da $D_a
= \omega A_m e^{j \left(\alpha+ \frac{\pi}{2}\right)} = j\omega
\bar{A}$, quindi

\[
\dfrac {d\;a(t)}{dt}=\dfrac{d}{dt}[A_mcos(\omega
  t+\alpha)]\ \Longleftrightarrow D_a=j\omega \bar{A}
\]

\section{Potenza ed Energia in Regime Sinusoidale}

Siano
\begin{itemize}
\item $v(t)=V_mcos(\omega t+\alpha)$
\item $i(t)=I_mcos(\omega t+\beta)$
\end{itemize}

Tenendo presente le due relazioni sopra citate,la potenza elettrica
istantanea assorbita da un generico bipolo del circuito \`e

\[
p(t)=v(t)i(t)=V_mI_mcos(\omega t+\alpha)cos(\omega t+\beta)
\]

Applichiamo l'identit\`a trigonometrica, che afferma

\[
2\cdot cosxcosy = cos(x+y)+cos(x-y)
\]

Abbiamo che

\[
p(t)=\dfrac{1}{2}V_mI_m[cos(\alpha-\beta)+cos(\underbrace{2\omega}_\text
  {pulsazione} t+\alpha+\beta)]\
\]

\subsection{Potenza Media}
\`E definita a valor medio sul periodo $T$ della potenza istantanea
assorbita

\[
P=\int_0^T p(\tau) d(\tau)
\]

Sostituendo

\[
P=\dfrac{V_mI_m}{2}cos(\alpha -\beta)
\]
La potenza quindi non dipende solo dalle ampiezze massime della
tensione e dell'intensit\`a di corrente, ma anche dalla differenza
delle loro fasi [$cos(\alpha -\beta)$]\ .


\subsection{Potenza Complessa}
La potenza media pu\`o essere espressa direttamente in termini dei
fasori della tensione $\bar V=V_me^{j\alpha}$ e dell'intesit\`a di
corrente $\bar I=I_me^{j\beta}$, infatti:

\[
P=\Re e\left\{\dfrac{1}{2}\bar{V}\bar{I}^*\right\}
\]

\[
\widehat P=\dfrac{1}{2} \bar{V}\bar{I}^* \longrightarrow \mathsf{potenza\;
  complessa\; assorbita}
\]

Poniamo

\[
Q=Im \{\widehat P \} \implies \widehat P=P+jQ
\]

Abbiamo

\[
Q=\dfrac{1}{2}V_mI_m sin (\alpha-\beta)
\]

\subsection{Energia in Regime Sinusoidale}
L'energia assorbita dal bipolo in regime sinusoidale nell'intevallo $[0,
  \widehat {t}]$ \`e data da

\[
\omega (0, \widehat t)=\int_0^{\widehat t} p(\tau)d(\tau)=(nT)P+
\int_{nT}^{nT+\Delta t} pt(\tau)d(\tau)
\]


$n$ \`e un numero intero tale che $\widehat{t}=nT+\Delta t$ con $\Delta t<
T$. $n$ \`e il numero di periodi $T$ contenuti nell'intervallo $[0,
  \widehat t]$.
Se $n>>1$ il contributo dovuto all'energia assorbita nell'intervallo
di tempo $(nT, nt+\Delta t)$ \`e trascurabile rispetto a $(nT)P$,
quindi

\[
\omega(0,\widehat t) \cong (nT)P \cong {\widehat t} P
\]

%CAPITOLO 3 DEL LIBRO

\chapter{Propriet\`a dei Circuiti}

\section{Grafo di un Circuito e sue Propriet\`a}

Un generico circuito pu\`o essere rappresentato tramite il metodo dei
\emph{grafi}, ossia uno schema geometrico del circuito in cui sono
presenti solo i nodi e i collegamenti tra essi (sono assenti i bipoli
che lo caratterizzano). Prendiamo ad esempio il circuito in {\bf
  Figura 2.1} costituito da $4$ \emph{nodi} e $5$ \emph{lati}. Per
convenzione lo indicheremo con $G$.

\ifx\JPicScale\undefined\def\JPicScale{1}\fi
\unitlength \JPicScale mm
\begin{picture}(91,81)(15,15)
\linethickness{0.3mm}
\put(50,40){\line(0,1){36}}
\put(50,40){\circle*{1.5}}
\linethickness{0.3mm}
\put(50,40){\line(1,0){36}}
\linethickness{0.3mm}
\put(86,40){\line(0,1){36}}
\put(86,40){\circle*{1.5}}
\linethickness{0.3mm}
\put(50,76){\line(1,0){36}}
\linethickness{0.3mm}
\multiput(50,76)(0.12,-0.12){300}{\line(1,0){0.12}}
\linethickness{0.3mm}
\multiput(49,59)(0.12,-0.12){8}{\line(1,0){0.12}}
\linethickness{0.3mm}
\multiput(50,58)(0.12,0.12){8}{\line(1,0){0.12}}
\linethickness{0.3mm}
\multiput(85,59)(0.12,-0.12){8}{\line(1,0){0.12}}
\linethickness{0.3mm}
\multiput(86,58)(0.12,0.12){8}{\line(1,0){0.12}}
\linethickness{0.3mm}
\put(68,58){\line(0,1){1}}
\linethickness{0.3mm}
\put(67,58){\line(1,0){1}}
\linethickness{0.3mm}
\multiput(67,41)(0.12,-0.12){8}{\line(1,0){0.12}}
\linethickness{0.3mm}
\multiput(67,39)(0.12,0.12){8}{\line(1,0){0.12}}
\linethickness{0.3mm}
\multiput(67,77)(0.12,-0.12){8}{\line(1,0){0.12}}
\linethickness{0.3mm}
\multiput(67,75)(0.12,0.12){8}{\line(1,0){0.12}}
\linethickness{0.3mm}
\put(89,37){\circle{4}}

\linethickness{0.3mm}
\put(89,79){\circle{4}}

\linethickness{0.3mm}
\put(47,79){\circle{4}}

\linethickness{0.3mm}
\put(47,37){\circle{4}}

\put(50,76){\circle*{1.5}}
\put(86,76){\circle*{1.5}}
\put(47,79){\makebox(0,0)[cc]{$1$}}

\put(88.92,79){\makebox(0,0)[cc]{$4$}}

\put(47,37){\makebox(0,0)[cc]{$2$}}

\put(89,37){\makebox(0,0)[cc]{$3$}}

\put(46,58){\makebox(0,0)[cc]{$1$}}

\put(67,37){\makebox(0,0)[cc]{$2$}}

\put(90,58){\makebox(0,0)[cc]{$3$}}

\put(67,79){\makebox(0,0)[cc]{$4$}}

\put(69,61){\makebox(0,0)[cc]{$5$}}

\put(68,25){\makebox(0,0)[cc]{{\bf Figura 2.1} Grafo $G$}}
\end{picture}

Di seguito sono enunciate le propriet\`a annesse alla teoria dei \emph{grafi}.
\subsection{Grafo, Grafo Orientato, Sottografo}
Prendiamo in considerazione il grafo in {\bf Figura 2.1}.

\begin{itemize}
  \item Un grafo $G(N, L)$ \`e costituito dall'insieme di $n$ nodi, che
    indicheremo con $N=\{1, 2,..., l\}$, dall'insieme di lati $l$
    \emph{lati}, che indicheremo con \\$L=\{1, 2,..., l\}$, e dalla
    \emph{relazione di incidenza} che ogni lato (bipolo) fa
    corrispondere la coppia di nodi nei quali quel lato incide.
  \item Se ogni lato (bipolo) del grafo \`e orientato, il grafo allora
    si dice {\bf orientato}. Per ciascun lato del grafo di un circuito
    \`e orientato con la fraccia che indica il verso di riferimento
    dell'intensit\`a della corrente del corrispondente bipolo.
  \item Si consideri un grafo $G(N, L)$. Il grafo $G_1(N, L)$ si dice
    sottografo di $G(N, L)$, se $N_1$ \`e un sottoinsieme di $N$,
    $L_1$ \`e un sottoinsieme di $L$ e la realzine di incidenza tra i
    nodi di $N_1$ ed i lati di $L_1$ \`e la stessa relazione che si ha
    nel grafo $G(N, L)$
\end{itemize}

\subsection{Grafo Connesso}

Un grafo si dice connesso se ogni nodo \`e collegato ad un qualsiasi
altro nodo attraverso uno o pi\`u lati.
Un grafo connesso contiene sottografi connessi. Circuiti di interesse
con grafi non connessi sono io circuiti che contengono elementi con
pi\`u di due terminali, come, ad esempio, i doppi bipoli.

\subsection{Maglia}
\begin{itemize}
\item Sia un dato grafo connesso $G$. Una maglia di $G$ \`e un sottografo
  connesso in un ciascun nodo incidano due e solo due lati.
\item Se alla maglia viene associato un verso di percorrenza essa \`e
  detta \emph{orientata}.
\end{itemize}

Ogni maglia forma un percorso chiuso, perch\`e essa deve costituire un
sottografo connesso in cui in ogni nodo coincidono due e due soli
lati: percorrendo interamente la maglia ciascun lato e ciascun nodo
vengono incontrati una ed una sola volta.

\subsection{Albero, coalbero}

\begin{itemize}
\item Sia dato un grafo connesso $G$. Un \emph{albero} $A$ di $G$ \`e
  un suo sottografo connesso che comprende tutti i nodi e non
  contiene alcuna maglia.
\item Il \emph{coalbero} $C$ di $G$, corrisponde all'albero $A$, \`e
  l'insieme dei dati complementare a quelli dell'albero: l'unione dei
  lati dell'albero e del coalbero coincide con l'insieme di tutti i
  lati di $G$.
\end{itemize}

Per l'albero di un qualsiasi grafo connesso vale la seguente
propriet\`a

\begin{prop}
  Si consideri un grafo connesso $G$ con $n$ nodi ed $l$ lati. Ciascun
  albero del grafo $G$ \`e costituito da $(n-1)$ lati
  (indipendentemente dal numero dei lati del grafo e della realzione
  di incidenza).
\end{prop}

Una diretta conseguenza della {\bf Propriet\`a 1} \`e relativa al
coalbero. Difatti, essendo sempre $(n-1)$ il numero di lati
dell'albero, e del fatto che il coalbero \`e il componente all'albero,
si ha:

\begin{prop}
  Si consideri un grafo connesso $G$ con $n$ nodi ed $l$ lati. Ciascun
  coalbero del grafo \`e costituito da $[l-(n-1)]$ lati
  (indipendentemente dalla relazione di incidenza del grafo).
\end{prop}

Una maglia {\bf fondamentale} \`e descritta dalla seguente relazione:

\begin{prop}
  Si consideri un sottografo che si ottiene aggiungendo all'albero $A$
  un solo lato di coalbero $C$: esso contiene una e una soloa maglia,
  che si ottiene eliminando tutti quei lati che non appartengono al
  percorso chiuso. Una maglia ottenuta in questo modo prende il nome
  di {\bf maglia fondamentale} del coalbero $C$.
\end{prop}

\`E evidente, allora, che aggiungendo un lato di coalbero per volta
\`e possibile costruire $[l-(n-1)]$ maglie fondamentali
distinte. Questo insieme di maglie prende il nome di \emph{insieme di
  maglie fondamentali} del coalbero $C$ del grafo $G$.

\subsection{Grafo Planare}

Un grafo si dice planare se pu\`o essere tracciato su un piano senza
che nessuna coppia di lati si intersechi in un punto che non sia un
nodo.
Ogni grafo planare gode della seguente propriet\`a:

\begin{prop}
Ogni maglia partiziona il piano in due regioni, quella interna al
cammino chiuso e quella esterna. Tra tutte le possibili maglie di un
grafo planare, rivestono particolare interesse quelle che non
contengono nessun lato al loro interno.  
\end{prop}

\subsection{Anello}

Un anello \`e una maglia di un grafo planare che non contiene lati al
suo interno.

\begin{prop}
  Si consideri un grafo planare connesso $G$ con $n$ nodi e $l$
  lati. Il grafo $G$ ha $[l-(n-1)]$ anelli.
\end{prop}
\subsection{Insieme di Taglio}
Si consideri un grafo connesso $G(N, L)$. Un sottoinsieme $T$ dei lati
$L$ del grafo, si dice insieme di taglio se, contemporaneamente:
\begin{itemize}
\item la rimozione del grafo di tutti i lati dell'insieme di taglio
  conduce a due sottografi non connessi;
\item il ripristino si uno qualsiasi dei lati dell'insieme di taglio
  connette nuovamente i due sottografi.
\end{itemize}
Se il grafo \`e orientato, l'insieme di taglio si dice orientato.

\subsubsection{Legge di Kirchhoff per gli Insiemi di Taglio}
  La somma algebrica delle intensit\`a di corrente dei bipoli che
  formano un qualsiasi insieme di taglio \`e uguale a zero istante per
  istante.


\section{Equazioni di Kirchhoff Indipendenti}
le equazioni circuitali sono costituite dalle equazioni di kirchhoff e
dalle equazioni caratteristiche degli elementi circuitali. le
equazioni di kirchhoff sono algebriche lineari ed omogenee. invece le
equazioni di caratteristiche possono essere, in generale, algebriche o
differenziali, lineari o non lineari, tempo-invarianti o
tempo-varianti, omogenee o non omogenee, a seconda della natura degli
elementi circuitali.

\subsection{Indipendenza delle Equazioni di Kirchhoff per le Correnti}
Prendiamo in esame le equazioni di Kirchhoff per le correnti
analizzando un circuito avente un grafo come di seguito.

\ifx\JPicScale\undefined\def\JPicScale{1}\fi
\unitlength \JPicScale mm
\begin{picture}(91,81)(15,15)
\linethickness{0.3mm}
\put(50,40){\line(0,1){36}}
\put(50,40){\circle*{1.5}}
\linethickness{0.3mm}
\put(50,40){\line(1,0){36}}
\linethickness{0.3mm}
\put(86,40){\line(0,1){36}}
\put(86,40){\circle*{1.5}}
\linethickness{0.3mm}
\put(50,76){\line(1,0){36}}
\linethickness{0.3mm}
\multiput(50,76)(0.12,-0.12){300}{\line(1,0){0.12}}
\linethickness{0.3mm}
\multiput(49,59)(0.12,-0.12){8}{\line(1,0){0.12}}
\linethickness{0.3mm}
\multiput(50,58)(0.12,0.12){8}{\line(1,0){0.12}}
\linethickness{0.3mm}
\multiput(85,59)(0.12,-0.12){8}{\line(1,0){0.12}}
\linethickness{0.3mm}
\multiput(86,58)(0.12,0.12){8}{\line(1,0){0.12}}
\linethickness{0.3mm}
\put(62,63){\line(0,1){1}}
\linethickness{0.3mm}
\put(62,64){\line(1,0){1}}
\linethickness{0.3mm}
\multiput(67,41)(0.12,-0.12){8}{\line(1,0){0.12}}
\linethickness{0.3mm}
\multiput(67,39)(0.12,0.12){8}{\line(1,0){0.12}}
\linethickness{0.3mm}
\multiput(67,77)(0.12,-0.12){8}{\line(1,0){0.12}}
\linethickness{0.3mm}
\multiput(67,75)(0.12,0.12){8}{\line(1,0){0.12}}
\linethickness{0.3mm}
\put(89,37){\circle{4}}

\linethickness{0.3mm}
\put(89,79){\circle{4}}

\linethickness{0.3mm}
\put(47,79){\circle{4}}

\linethickness{0.3mm}
\put(47,37){\circle{4}}

\put(50,76){\circle*{1.5}}
\put(86,76){\circle*{1.5}}
\put(47,79){\makebox(0,0)[cc]{$1$}}

\put(88.92,79){\makebox(0,0)[cc]{$4$}}

\put(47,37){\makebox(0,0)[cc]{$2$}}

\put(89,37){\makebox(0,0)[cc]{$3$}}

\put(46,58){\makebox(0,0)[cc]{$1$}}

\put(67,37){\makebox(0,0)[cc]{$2$}}

\put(90,58){\makebox(0,0)[cc]{$3$}}

\put(67,79){\makebox(0,0)[cc]{$4$}}

\put(65,65){\makebox(0,0)[cc]{$5$}}

\put(68,25){\makebox(0,0)[cc]{{\bf Figura 2.2} Grafo}}
\end{picture}

Applichiamo le LKC ai nodi e abbiamo:

\[
\begin{array}{l l l l}
\text{{\bf nodo 1. }} i_1+i_2-i_5 = 0\\
\text{{\bf nodo 2. }} -i_1+i_4 = 0 \\
\text{{\bf nodo 3. }} -i_3-i_4+i_5 = 0\\
\text{{\bf nodo 4. }} i_2+i_3=0
\end{array}
\]

Esse sono linearmente dipendenti. Infatti sommando membro a membro le
$4$ equazioni del sistema si ottiene l'identit\`a $0=0$. Per tanto la
$4^a$ \`e combinazione lineare delle altre $3$. Dunque tutte le
\emph{informazioni} contenute nella $4^a$ equazione del sistema sono
gi\`a presenti nlle altre $3$ e, quindi essa \`e ridondante.
Questo risultato \`e del tutto generale. Le $n$ equazioni di Kirchhoff
per le correnti ($n$ \`e il numero i nodi):

\[
A_a{\mathbf i}=0
\]

di un circuito sono linearmenti dipendenti, qualunque sia il grafo del
circuito. Questo \`e una diretta conseguenza del il fatto che la somma di tutte le
righe della matrice di incidenza \`e la riga identicamente
nulla. Questo \`e in accordo con il fatto che il rango della matrice
di incidenza $A_a$ \`e minore di $n$. Quindi, a conclusione delle
nostre tesi, qualsiasi equazione del sistema pu\`o essere eliminata,
senza che l'informazione contenuta nel sistema ne risenta in alcun
modo.

\begin{prop}
  Per un circuito con grafo connesso con $n$ nodi, $n-1$ equazioni di
  Kirchhoff per le correnti, scelte in maniera arbitraria tra le
  possibili $n$, sono linearmente indipendenti.
\end{prop}

\subsubsection{Equazioni Indipendenti per gli Insiemi di Taglio}
Fissato un albero, un insieme ti taglio fondamentale contiene un unico
lato dell'albero assieme ad alcuni lati di coalbero. Scrivendo le
equazioni di Kirchhoff per gli insiemi di taglio fondamentali
associati ad un qualisiasi albero si perviene ad un insieme di
equazioni che sono coerentemente indipendenti fra loro inquanto
ciascuna contiene (in esclusiva) l'intensi\`a cirrente relativa ad un
lato dell'albero. Essendo proprio $n-1$ il numero dei lati
dell'albero, abbiamo cos\`i costruito un insieme di $n-1$ equazioni
per gli insieme di taglio indipendenti. Siccome gli altri insiemi di
taglio possono essere sempre espressi come un'opportuna unione di
insiemi di taglio fondamentali, le corrisponenti equazioni sono
esprimibili come combinazione lineare di quelle corrispondenti ad
insieme di taglio fondamentali. In conclusione, le equazioni
indipendenti per gli insiemi di taglio sono $n-1$. Da questo risultato
segue che imporre $n-1$ equazioni indipendenti agli insiemi di taglio
equivale ad imporre le equazioni di Kirchhoff per i nodi

\subsection{Indipendenza delle Equazioni di Kirchhoff per le Tensioni}

Ora consideriamo le leggi di Kirchhoff per le tensioni. Analizziamo
quindi le equazioni ottenute applicando la legge di Kirchhoff per le
tensioni al circuito in {\bf Figura 2.2}:

\[
\begin{array}{l l l}
  \mathbf{M_1}\longrightarrow -v_1+v_2+v_3+v_4=0\\
  \mathbf{M_2}\longrightarrow v_3+v_4-v_5=0\\
  \mathbf{M_3}\longrightarrow -v_1+v_2+v_5=0  
\end{array}  
\]

Queste equazioni non sono tra loro linearmente indipendenti. Ad
esempio la terza equazione \`e combinazione lineare delle restanti due
equazioni. Infatti piu\`o essere ottenuta sottraendo membro a membro
le prime due.
Ora consideriamo un generico circuito e sia $m$ il numero di
maglie. Le $m$ maglie equazioni di Kirchhoff per le tensioni
corrispondenti

\[
B_a\mathbf{v}=0
\]

non sono linearmente indipendenti. Quindi per stabilire quante sono le
equazioni linearmente indipendenti ricorriamo alla seguente
propriet\`a che esprime l'\emph{indipendenza delle equazioni di
  Kirchhoff per le tensioni}:

\begin{prop}
  Per un circuito con grafo connesso con $n$ nodi e $l$ lati, le
  $l-(n-1)$ equazioni di Kirchhoff per le tensioni relative ad un
  insieme di maglie fondamentali sono linearmente indipendenti. Le
  equazioni di Kirchhoff per le maglie del circuito possono essere
  espresse come combinazioni lineari delle equazioni delle maglie fondamentali.
\end{prop}

La propriet\`a di indipendenza pu\`o essere dimostrata semplicemente,
anche nel caso generale, ricordando che ogni maglia di un insieme di
maglie fondamentali ha almeno un lato in esclusiva. Di conseguenza,
ogni equazione dell'insieme delle equazioni di Kirchhoff per un
insieme di maglie fondamentali ha almeno una tensione incognita in
esclusiva, e quindi le $l-(n-1)$ equazioni per le maglie fondamentali
sono linearmente indipendenti.

\section{Conservazione delle Potenze Elettriche e Teorema di Tellegen}
L'espressione della potenza elettrica assorbita dal $k-esimo$ bipolo
$k=(1, 2,..., l)$ del circuito \`e data da

\[
p_k(t)=i_k(t)v_k(t)
\]

\newtheorem{mydef}{Propriet\`a}
\begin{mydef}
  La somma delle potenze elettriche assorbite da tutti i bipoli di
  un circuito \`e, istante per istante, uguale a $0$.
\end{mydef}

\[
\sum_{k=1}^{l} p_k(t) = \sum_{k=1}^{l} i_k(t)v_k(t)=0
\]

A questo punto introduciamo il vettore {\bf i} = $(i_1, i_2,..., i_l)^T$
delle correnti, il vettore {\bf v} = $(v_1, v_2,..., v_l)^T$ delle
tensioni ed il vettore {\bf u} = $(u_1, u_2,..., u_l)^T$ dei
potenziali di nodo del circuito. Quindi:

\[
\sum_{k=1}^l v_ki_k=v_1i_1+v_2i_2+...+v_li_l = \mathbf{v^Ti}
\]

Le tensioni del circuito possono essere scritte come

\[
\mathbf{v= A_a^Tu} \quad \quad \quad \mathsf{\;A_a= \; matrice\; d'\; incidenza}
\]

Quindi sostituendo

\[
\mathbf{v^Ti=(A_a^Tu)i}
\]

Per la propriet\`a di identit\`a possiamo scrivere

\[
\mathbf{(A_a^Tu)^T=u^TA_a}
\]

Quindi

\[
\sum_{k=1}^l v_ki_k=\mathbf{(u^TA_ai) = u^T(A_ai)}
\]

Dalle leggi di {\bf Kirchhoff} sappiamo che $\mathbf{A_ai=0}$ quindi
possiamo scrivere

\[
\sum_{k=1}^{l}v_ki_k=0
\]

1
\subsection{Teorema di Tellegen}

La sommatoria delle potenze virtuali assorbite da ciascun lato del
grafo \`e uguale a $0$.

\[
\sum_{k=1}^li'_k v''_k = 0\; \mathsf {o}\; \left(\sum_{k=1}^li''_k v'_k=0\right)
\]

Consideriamo due bipoli come in figura

\vspace{1.5cm}

\ifx\JPicScale\undefined\def\JPicScale{1}\fi
\unitlength \JPicScale mm
\begin{picture}(83.06,86.38)(10,0)
  \linethickness{0.3mm}
  \multiput(45.17,58.63)(0.12,0.12){47}{\line(1,0){0.12}}
  \multiput(45.17,58.63)(0.12,-0.12){118}{\line(0,-1){0.12}}
  \multiput(50.83,64.28)(0.12,-0.12){118}{\line(0,-1){0.12}}
  \multiput(59.31,44.49)(0.12,0.12){47}{\line(1,0){0.12}}
  \linethickness{0.3mm}
  \multiput(57.9,71.36)(0.12,0.12){47}{\line(1,0){0.12}}
  \multiput(57.9,71.36)(0.12,-0.12){118}{\line(0,-1){0.12}}
  \multiput(63.56,77.01)(0.12,-0.12){118}{\line(0,-1){0.12}}
  \multiput(72.04,57.21)(0.12,0.12){47}{\line(1,0){0.12}}
  \linethickness{0.3mm}
  \put(43.76,64.28){\line(1,0){1.41}}
  \linethickness{0.3mm}
  \put(45.17,64.28){\line(0,1){1.41}}
  \linethickness{0.3mm}
  \put(56.49,77.01){\line(1,0){1.41}}
  \linethickness{0.3mm}
  \put(57.9,77.01){\line(0,1){1.41}}
  \linethickness{0.3mm}
  \multiput(53.66,81.25)(0.12,-0.12){59}{\line(0,-1){0.12}}
  \put(53.66,81.25){\line(-1,1){0.12}}
  \put(53.66,81.25){\circle*{1.5}}
  \linethickness{0.3mm}
  \multiput(40.93,68.53)(0.12,-0.12){59}{\line(1,0){0.12}}
  \put(40.93,68.53){\line(-1,1){0.12}}
  \put(40.93,68.53){\circle*{1.5}}
  \linethickness{0.3mm}
  \multiput(62.14,47.31)(0.12,-0.12){53}{\line(0,-1){0.12}}
  \put(68.51,40.95){\line(1,-1){0.12}}
  \put(68.51,40.95){\circle*{1.5}}
  \linethickness{0.3mm}
  \multiput(74.87,60.04)(0.12,-0.12){53}{\line(0,-1){0.12}}
  \put(81.23,53.68){\line(1,-1){0.12}}
  \put(81.23,53.68){\circle*{1.5}}
  \put(43.05,70.65){\makebox(0,0)[cc]{$+$}}

  \put(55.78,83.38){\makebox(0,0)[cc]{$+$}}

  \put(71.33,42.36){\makebox(0,0)[cc]{$-$}}

  \put(84.06,55.09){\makebox(0,0)[cc]{$-$}}

  \put(47.29,66.41){\makebox(0,0)[cc]{$i'_k$}}

  \put(60.02,79.13){\makebox(0,0)[cc]{$i''_k$}}

  \put(50.12,48.02){\makebox(0,0)[cc]{$v'_k$}}

  \put(73.46,71.36){\makebox(0,0)[cc]{$v''_k$}}

  \put(65.23,20.65){\makebox(0,0)[cc]{{\bf Figura 2.3} Due bipoli generici}}
\end{picture}


Abbiamo che $\widehat{p}=i'_kv''_k$

\[
\widehat{p}_k^{(e)}=-\widehat{p}_k
\]

\section{Propriet\`a Di Non Amplificazione}
\subsection{Non Amplificazione delle Tensioni}

\ifx\JPicScale\undefined\def\JPicScale{1}\fi
\unitlength \JPicScale mm
\begin{picture}(90,69)(15,15)
  \linethickness{0.3mm}
  \put(60,68){\line(1,0){30}}
  \put(60,40){\line(0,1){28}}
  \put(90,40){\line(0,1){28}}
  \put(60,40){\line(1,0){30}}
  \linethickness{0.3mm}
  \put(50,64){\line(1,0){10}}
  \put(50,64){\line(-1,0){0.12}}
  \put(50,64){\circle*{1.5}}
  \linethickness{0.3mm}
  \put(50,44){\line(1,0){10}}
  \put(50,44){\line(-1,0){0.12}}
  \put(50,44){\circle*{1.5}}
  \linethickness{0.3mm}
  \put(47,60){\line(1,0){6}}
  \put(47,48){\line(0,1){12}}
  \put(53,48){\line(0,1){12}}
  \put(47,48){\line(1,0){6}}
  \linethickness{0.3mm}
  \put(50,60){\line(0,1){4}}
  \linethickness{0.3mm}
  \put(50,44){\line(0,1){4}}
  \linethickness{0.3mm}
  \put(50,68){\circle{4}}

  \linethickness{0.3mm}
  \put(50,40){\circle{4}}

  \put(50,68){\makebox(0,0)[cc]{$1$}}

  \put(50,40){\makebox(0,0)[cc]{$n$}}

  \put(42,54){\makebox(0,0)[cc]{$v_a$}}

  \put(46,64){\makebox(0,0)[cc]{$+$}}

  \put(46,44){\makebox(0,0)[cc]{$-$}}

  \linethickness{0.3mm}
  \multiput(68.25,52)(0.12,0.14){19}{\line(0,1){0.14}}
  \linethickness{0.3mm}
  \multiput(70.5,54.62)(0.12,-0.14){19}{\line(0,-1){0.14}}
  \linethickness{0.3mm}
  \multiput(72.75,52)(0.12,0.14){19}{\line(0,1){0.14}}
  \linethickness{0.3mm}
  \multiput(75,54.62)(0.12,-0.14){19}{\line(0,-1){0.14}}
  \linethickness{0.3mm}
  \multiput(77.25,52)(0.12,0.14){19}{\line(0,1){0.14}}
  \linethickness{0.3mm}
  \multiput(79.5,54.62)(0.12,-0.14){19}{\line(0,-1){0.14}}
  \linethickness{0.3mm}
  \put(81.75,52){\line(1,0){2.25}}

  \linethickness{0.15mm}
  \put(66,52){\line(1,0){2.25}}
  \linethickness{0.3mm}
  \put(82,52){\line(1,0){4}}
  \put(86,52){\line(1,0){0.12}}
  \put(86,52){\circle*{1.5}}
  \linethickness{0.3mm}
  \put(64,52){\line(1,0){4}}
  \put(64,52){\line(-1,0){0.12}}
  \put(64,52){\circle*{1.5}}

  \put(69.23,20.65){\makebox(0,0)[cc]{{\bf Figura 2.4} Circuito generico}}
\end{picture}
\\
\\
Si consideri un circuito costituito da resistori strettamente passivi
(i resistori possono essere anche non lineari) e un solo bipolo
attivo.
\begin{definizione}
  La tensione del genrico bipolo strettamente passivo non pu\`o
  superare, in valore assoluto, quella dell'unico bipolo attivo.
\end{definizione}

Per dimostrare il teorema prendiamo in considerazione 4 bipoli
collegati al nodo $o$, come in figura.

\ifx\JPicScale\undefined\def\JPicScale{1}\fi
\unitlength \JPicScale mm
\begin{picture}(101,85)(10,3)
  \linethickness{0.3mm}
  \put(60,74){\line(1,0){6}}
  \put(60,60){\line(0,1){14}}
  \put(66,60){\line(0,1){14}}
  \put(60,60){\line(1,0){6}}
  \linethickness{0.3mm}
  \put(60,34){\line(1,0){6}}
  \put(60,20){\line(0,1){14}}
  \put(66,20){\line(0,1){14}}
  \put(60,20){\line(1,0){6}}
  \linethickness{0.3mm}
  \put(90,44){\line(0,1){6}}
  \put(76,50){\line(1,0){14}}
  \put(76,44){\line(1,0){14}}
  \put(76,44){\line(0,1){6}}
  \linethickness{0.3mm}
  \put(50,44){\line(0,1){6}}
  \put(36,50){\line(1,0){14}}
  \put(36,44){\line(1,0){14}}
  \put(36,44){\line(0,1){6}}
  \linethickness{0.3mm}
  \put(63,34){\line(0,1){26}}
  \linethickness{0.3mm}
  \put(63,47){\line(1,0){13}}
  \linethickness{0.3mm}
  \put(50,47){\line(1,0){13}}
  \linethickness{0.3mm}
  \multiput(68,47)(0.12,-0.12){8}{\line(1,0){0.12}}
  \linethickness{0.3mm}
  \multiput(68,47)(0.12,0.12){8}{\line(1,0){0.12}}
  \linethickness{0.3mm}
  \multiput(57,48)(0.12,-0.12){8}{\line(1,0){0.12}}
  \linethickness{0.3mm}
  \multiput(57,46)(0.12,0.12){8}{\line(1,0){0.12}}
  \linethickness{0.3mm}
  \multiput(62,53)(0.12,-0.12){8}{\line(1,0){0.12}}
  \linethickness{0.3mm}
  \multiput(63,52)(0.12,0.12){8}{\line(1,0){0.12}}
  \linethickness{0.3mm}
  \multiput(62,41)(0.12,0.12){8}{\line(1,0){0.12}}
  \linethickness{0.3mm}
  \multiput(63,42)(0.12,-0.12){8}{\line(1,0){0.12}}
  \linethickness{0.3mm}
  \put(63,74){\line(0,1){8}}
  \put(63,82){\line(0,1){0.12}}
  \put(63,82){\circle*{1.5}}
  \linethickness{0.3mm}
  \put(90,47){\line(1,0){8}}
  \put(98,47){\line(1,0){0.12}}
  \put(98,47){\circle*{1.5}}
  \linethickness{0.3mm}
  \put(63,12){\line(0,1){8}}
  \put(63,12){\line(0,-1){0.12}}
  \put(63,12){\circle*{1.5}}
  \linethickness{0.3mm}
  \put(29,47){\line(1,0){7}}
  \put(29,47){\line(-1,0){0.12}}
  \put(29,47){\circle*{1.5}}
  \put(63,47){\circle*{1.5}}
  \put(66,44){\makebox(0,0)[cc]{$o$}}

  \put(26,47){\makebox(0,0)[cc]{$p$}}

  \put(63,9){\makebox(0,0)[cc]{$s$}}

  \put(63,85){\makebox(0,0)[cc]{$q$}}

  \put(101,47){\makebox(0,0)[cc]{$r$}}

  \put(60,53){\makebox(0,0)[cc]{$i_q$}}

  \put(55,50){\makebox(0,0)[cc]{$i_p$}}

  \put(60,41){\makebox(0,0)[cc]{$i_s$}}

  \put(70,50){\makebox(0,0)[cc]{$i_r$}}

  \put(68.23,0.65){\makebox(0,0)[cc]{{\bf Figura 2.5} Bipoli casuali
      estratti dal generico circuito in \emph {Figura 2.4}}}
\end{picture}

Applichiamo la LKC al nodo $o$
\[
i_p+i_q+i_r+i_s=0
\]

Data questa relazione possiamo avere due possibili alternative.

\begin{itemize}
\item Le correnti sono tutte nulle
\item Abbiamo delle correnti nulle, altre negative altre positive.
\end{itemize}

Prendiamo in esame la seconda possibilit\`a. Quindi

\[
i_q>0 \quad i_r<0
\]

\[
p_p=v_qi_q>0 \quad p_r=v_ri_r>0
\]

Applichiamo il metodo dei potenziali nodali

\[
v_q=u_q-u_o>0 \quad v_r=u_r-u_o<0
\]

Ne consegue la seguente relazione verificando il {\bf Teorema 1}

\[
u_r<u_o<u_q
\]

\subsection{Non Amplificazione Delle Correnti}
Si consideri un circuito costituito da resitori strettamente passivi e
da un solo bipolo attivo.
\begin{definizione}
  L'intensit\`a di corrente del generico bipolo strettamente passivo
  non pu\`o superare, in valore assoluto, quella dell'unico bipolo attivo.
\end{definizione}

\ifx\JPicScale\undefined\def\JPicScale{1}\fi
\unitlength \JPicScale mm
\begin{picture}(139,77)(15,6)
  \linethickness{0.3mm}
  \put(74,76){\circle*{1.5}}

  \linethickness{0.3mm}
  \put(74,58){\circle*{1.5}}

  \linethickness{0.3mm}
  \put(74,40){\circle*{1.5}}

  \linethickness{0.3mm}
  \put(74,20){\circle*{1.5}}

  \linethickness{0.3mm}
  \qbezier(74,58)(68.09,49.98)(70.62,45.06)
  \qbezier(70.62,45.06)(73.16,40.14)(74,40)
  \linethickness{0.3mm}
  \qbezier(74,58)(82.86,50.8)(79.06,45.53)
  \qbezier(79.06,45.53)(75.27,40.26)(74,40)
  \linethickness{0.3mm}
  \qbezier(74,40)(66.45,31.37)(69.69,25.78)
  \qbezier(69.69,25.78)(72.92,20.2)(74,20)
  \linethickness{0.3mm}
  \qbezier(74,40)(82.04,30.14)(78.59,25.08)
  \qbezier(78.59,25.08)(75.15,20.02)(74,20)
  \linethickness{0.3mm}
  \qbezier(74,76)(65.81,69.62)(68.75,64)
  \qbezier(68.75,64)(72.69,58.38)(74,58)
  \linethickness{0.3mm}
  \qbezier(74,76)(81.63,66.84)(78.36,62.41)
  \qbezier(78.36,62.41)(75.09,57.98)(74,58)
  \linethickness{0.3mm}
  \qbezier(74,76)(88.42,68.48)(82.81,63.34)
  \qbezier(82.81,63.34)(76.2,58.21)(74,58)
  \linethickness{0.3mm}
  \qbezier(74,58)(56.8,44.22)(63.83,32.48)
  \qbezier(63.83,32.48)(71.46,20.75)(74,20)
  \linethickness{0.3mm}
  \qbezier(74,20)(41.38,46.06)(54.22,60.89)
  \qbezier(54.22,60.89)(69.05,75.72)(74,76)
  \linethickness{0.3mm}
  \multiput(83.38,67.38)(0.12,-0.12){8}{\line(1,0){0.12}}
  \linethickness{0.3mm}
  \multiput(84.38,66.38)(0.12,0.12){8}{\line(1,0){0.12}}
  \linethickness{0.3mm}
  \multiput(69,49)(0.12,-0.12){8}{\line(1,0){0.12}}
  \linethickness{0.3mm}
  \multiput(70,48)(0.12,0.12){8}{\line(1,0){0.12}}
  \linethickness{0.3mm}
  \multiput(61,42)(0.12,-0.12){8}{\line(1,0){0.12}}
  \linethickness{0.3mm}
  \multiput(62,41)(0.12,0.12){8}{\line(1,0){0.12}}
  \linethickness{0.3mm}
  \multiput(79,50)(0.12,-0.12){8}{\line(1,0){0.12}}
  \linethickness{0.3mm}
  \multiput(80,49)(0.12,0.12){8}{\line(1,0){0.12}}
  \linethickness{0.3mm}
  \multiput(67,67)(0.12,-0.12){8}{\line(1,0){0.12}}
  \linethickness{0.3mm}
  \multiput(68,66)(0.12,0.12){8}{\line(1,0){0.12}}
  \linethickness{0.3mm}
  \multiput(78,68)(0.12,-0.12){8}{\line(1,0){0.12}}
  \linethickness{0.3mm}
  \multiput(79,67)(0.12,0.12){8}{\line(1,0){0.12}}
  \linethickness{0.3mm}
  \multiput(78.5,30.12)(0.12,-0.12){8}{\line(1,0){0.12}}
  \linethickness{0.3mm}
  \multiput(79.5,29.12)(0.12,0.12){8}{\line(1,0){0.12}}
  \linethickness{0.3mm}
  \multiput(67.88,30.75)(0.12,-0.12){8}{\line(1,0){0.12}}
  \linethickness{0.3mm}
  \multiput(68.88,29.75)(0.12,0.12){8}{\line(1,0){0.12}}
  \linethickness{0.3mm}
  \multiput(49.62,49)(0.12,0.12){8}{\line(1,0){0.12}}
  \linethickness{0.3mm}
  \multiput(50.62,50)(0.12,-0.12){8}{\line(1,0){0.12}}
  \put(65,64.38){\makebox(0,0)[cc]{$i_1$}}

  \put(76,68){\makebox(0,0)[cc]{$i_2$}}

  \put(87,67){\makebox(0,0)[cc]{$i_3$}}

  \put(72,48){\makebox(0,0)[cc]{$i_5$}}

  \put(82,49){\makebox(0,0)[cc]{$i_6$}}

  \put(64,41){\makebox(0,0)[cc]{$i_4$}}

  \put(46,49){\makebox(0,0)[cc]{$i_a$}}

  \put(71,30){\makebox(0,0)[cc]{$i_7$}}

  \put(82,29){\makebox(0,0)[cc]{$i_8$}}

  \linethickness{0.3mm}
  \qbezier(58,77)(71.09,68.47)(79.88,72.12)
  \qbezier(79.88,72.12)(88.66,75.78)(89,77)
  \linethickness{0.3mm}
  \qbezier(46,60)(67.25,50.03)(79.5,53.38)
  \qbezier(79.5,53.38)(91.75,56.72)(92,58)
  \linethickness{0.3mm}
  \qbezier(45,42)(67.16,31.44)(80.88,35.5)
  \qbezier(80.88,35.5)(94.59,39.56)(95,41)
  \linethickness{0.3mm}
  \multiput(90,77)(0.16,-0.12){125}{\line(1,0){0.16}}
  \put(90,77){\vector(-4,3){0.12}}
  \linethickness{0.3mm}
  \multiput(94,58)(0.48,0.12){33}{\line(1,0){0.48}}
  \put(94,58){\vector(-4,-1){0.12}}
  \linethickness{0.3mm}
  \multiput(96,42)(0.12,0.17){117}{\line(0,1){0.17}}
  \put(96,42){\vector(-2,-3){0.12}}
  \put(124,62){\makebox(0,0)[cc]{Insiemi di Taglio}}

  \linethickness{0.3mm}
  \put(110,65){\line(1,0){29}}
  \put(110,59){\line(0,1){6}}
  \put(139,59){\line(0,1){6}}
  \put(110,59){\line(1,0){29}}
  \put(75,0){\makebox(0,0)[cc]{{\bf Figura 2.6} Grafo ordinato per
      potenziali di nodi decrescenti}}

\end{picture}
\\
\\
Prendendo in considerazione gli insiemi di taglio sui tre nodi e
applicando la $LKC$ abbiamo
\vspace{1cm}
\[
\left\{
\begin{array}{l l l}
  i_1+i_2+i_3=i_a \\
  i_4+i_5+i_6=i_a \\
  i_4+i_7+i_8=i_a \\
\end{array} \right.
\]

\[
|i_a|\ \leq 0 \quad i_k\geq 0 \longrightarrow k=1, 2,..., 8
\]

Ne consegue che
\[
i_a\geq i_k
\]

\section{Potenziali di Nodo}
Il metodo dei potenziali di nodo consiste nell'esprimere le tensioni
di ciascun lato attraverso delle opportune grandezze ausiliari, in
maniera tale da imporre che la legge di Kirchhoff per le tensioni sia
verificata automaticamente per ogni maglia del circuito.

Consideriamo ancora una volta un generico circuito, con $n$ nodi e $l$
bipoli, assegnamo i versi di riferimento per le intensit\`a di
corrente e fissiamo una volta per tutte le convenzioni
dell'utilizzatore per tutti i bipoli.

Il metodo dei potenziali di nodo consiste nell'associare, a ciascun
nodo, una variabile aleatoria, detta \emph{potenziale di nodo}: al
generico nodo $i\;(i=1, 2,...,n)$ associamo dunque il potenziale di nodo
$u_i$.

Facendo riferimento alla seguente figura.

\ifx\JPicScale\undefined\def\JPicScale{1}\fi
\unitlength \JPicScale mm
\begin{picture}(99.5,58)(10,15)
  \linethickness{0.3mm}
  \put(35,30){\line(1,0){30}}
  \linethickness{0.3mm}
  \multiput(35,30)(0.12,0.2){125}{\line(0,1){0.2}}
  \linethickness{0.3mm}
  \multiput(50,55)(0.12,-0.2){125}{\line(0,-1){0.2}}
  \linethickness{0.3mm}
  \put(65,30){\line(1,0){30}}
  \linethickness{0.3mm}
  \multiput(80,55)(0.12,-0.2){125}{\line(0,-1){0.2}}
  \linethickness{0.3mm}
  \multiput(65,30)(0.12,0.2){125}{\line(0,1){0.2}}
  \linethickness{0.3mm}
  \put(50,55){\line(1,0){30}}
  \linethickness{0.3mm}
  \multiput(49,31)(0.12,-0.12){8}{\line(1,0){0.12}}
  \linethickness{0.3mm}
  \multiput(49,29)(0.12,0.12){8}{\line(1,0){0.12}}
  \linethickness{0.3mm}
  \put(42,43){\line(1,0){1}}
  \linethickness{0.3mm}
  \put(43,42){\line(0,1){1}}
  \linethickness{0.3mm}
  \put(57,42){\line(0,1){1}}
  \linethickness{0.3mm}
  \put(57,43){\line(1,0){1}}
  \linethickness{0.3mm}
  \multiput(64,56)(0.12,-0.12){8}{\line(1,0){0.12}}
  \linethickness{0.3mm}
  \multiput(64,54)(0.12,0.12){8}{\line(1,0){0.12}}
  \linethickness{0.3mm}
  \put(72,43){\line(1,0){1}}
  \linethickness{0.3mm}
  \put(73,42){\line(0,1){1}}
  \linethickness{0.3mm}
  \multiput(79,31)(0.12,-0.12){8}{\line(1,0){0.12}}
  \linethickness{0.3mm}
  \multiput(79,29)(0.12,0.12){8}{\line(1,0){0.12}}
  \linethickness{0.3mm}
  \put(87,42){\line(0,1){1}}
  \linethickness{0.3mm}
  \put(87,43){\line(1,0){1}}
  \linethickness{0.3mm}
  \put(32.5,27.5){\circle{3}}
  
  \linethickness{0.3mm}
  \put(98,27.5){\circle{3}}
  
  \linethickness{0.3mm}
  \put(47,56.5){\circle{3}}
  
  \linethickness{0.3mm}
  \put(83.5,56.5){\circle{3}}
  
  \put(50,55){\circle*{1.5}}
  \put(80,55){\circle*{1.5}}
  \put(35,30){\circle*{1.5}}
  \put(65,30){\circle*{1.5}}
  \put(95,30){\circle*{1.5}}
  
  \put(32.5,27.5){\makebox(0,0)[cc]{$5$}}
  
  \put(46.9,56.5){\makebox(0,0)[cc]{$1$}}
  
  \put(83.51,56.5){\makebox(0,0)[cc]{$2$}}

  \put(98,27.5){\makebox(0,0)[cc]{$3$}}
  
  \linethickness{0.3mm}
  \put(65,27.5){\circle{3}}
  
  \put(64.8,27.5){\makebox(0,0)[cc]{$4$}}
  
  \put(40,43){\makebox(0,0)[cc]{$1$}}
  
  \put(64.5,57.5){\makebox(0,0)[cc]{$2$}}
  
  \put(90,43){\makebox(0,0)[cc]{$3$}}
  
  \put(79,27){\makebox(0,0)[cc]{$4$}}
  
  \put(75,42){\makebox(0,0)[cc]{$5$}}
  
  \put(60,43){\makebox(0,0)[cc]{$6$}}
  
  \put(49,27){\makebox(0,0)[cc]{$7$}}
  
  \put(52,56.8){\makebox(0,0)[cc]{$u_1$}}
  \put(78,56.8){\makebox(0,0)[cc]{$u_2$}}
  \put(94,27.5){\makebox(0,0)[cc]{$u_3$}}
  \put(64.7,34){\makebox(0,0)[cc]{$u_4$}}
  \put(37,27.5){\makebox(0,0)[cc]{$u_5$}}
  
  \put(64,10){\makebox(0,0)[cc]{{\bf Figura 2.7} Potenziali di nodo}}


  
\end{picture}

\vspace{2cm}

Assumiamo che sia possibile esprimere la tensione di ciascun bipolo
del circuito in funzione dei potenziali dei due nodi nei quali il lato
incide secondo la seguente regola: la tensione $v_s$ del generico
bipolo $s\;(s=1, 2,...,l)$ \`e espressa come differenza tra il
potenziale del nodo contrassegnato con il $"+"$ ed il potenziale del
nodo contrassegnato con il $"-"$:

\begin{equation}\label{eq:potenzialiNodo}
v_s=u_p-u_q \quad \text{per} \quad s=1, 2,...,l
\end{equation}

Le tensioni espresse da questa relazione verificano automaticamente la
$LKT$ qualunque siano i valori dei potenziali di nodo $u_1, u_2,...,
u_n$.
Prendendo sempre in considerazione la {\bf Figura 2.7} e la maglia
costituita dai lati $1, 2, 3, 4, 7$, la $LKT$ \`e:

\[
v_1+v_2-v_3-v_4-v_7=0
\]

Esprimendola attraverso i potenziali di nodo abbiamo:

\begin{equation}\label{eq:potenzialiNodoLKT}
  \begin{array}{l l l l l}
    v_1=u_5-u_1\\
    v_2=u_1-u_2\\
    v_3=u_3-u_2\\
    v_4=u_4-u_3\\
    v_7=u_5-u_4
  \end{array}
\end{equation}

Sostituendo nella $LKT$ abbiamo:

\[
(u_5-u_1)+(u_1-u_2)+(u-3-u_2)+(u_4-u_3)+(u_5-u_4)=0
\]

Questa equazione \`e sempre verificata, indipendentemente da valori
dei potenziali di nodo $u_1, u_2,..., u_5$, dunque \`e
un'identit\`a. Quindi la \ref{eq:potenzialiNodoLKT} verificano la $LKT$.

Quindi possiamo affermare che

\begin{definizione}
  Se le tensioni di un circuito sono sempre espresse attraverso i
  potenziali di nodo secondo la \ref{eq:potenzialiNodo}, allora esse
  verificano automaticamente la legge di Kirchhoff per le tensioni per
  qualsiasi maglia del circuito.
\end{definizione}

\subsubsection{Applicazione del Metodo dei Potenziali di Nodo}

Consideriamo il circuito in {\bf Figura 2.8}.

\ifx\JPicScale\undefined\def\JPicScale{1}\fi
\unitlength \JPicScale mm
\begin{picture}(91.25,73)(0,0)
\linethickness{0.3mm}
\multiput(16.75,52.75)(0.12,0.12){10}{\line(1,0){0.12}}
\linethickness{0.3mm}
\multiput(16.75,52.75)(0.12,-0.12){10}{\line(1,0){0.12}}
\linethickness{0.3mm}
\multiput(16.75,50.25)(0.12,0.12){10}{\line(1,0){0.12}}
\linethickness{0.3mm}
\multiput(16.75,50.25)(0.12,-0.12){10}{\line(1,0){0.12}}
\linethickness{0.3mm}
\multiput(16.75,47.75)(0.12,0.12){10}{\line(1,0){0.12}}
\linethickness{0.3mm}
\multiput(16.75,47.75)(0.12,-0.12){10}{\line(1,0){0.12}}
\linethickness{0.3mm}
\put(18,45.25){\line(0,1){1.25}}
\put(19.25,49.25){\makebox(0,0)[bl]{$R_1$}}

\linethickness{0.15mm}
\put(18,54){\line(0,1){1.25}}
\linethickness{0.3mm}
\multiput(37,59)(0.12,0.12){10}{\line(1,0){0.12}}
\linethickness{0.3mm}
\multiput(38.25,60.25)(0.12,-0.12){10}{\line(1,0){0.12}}
\linethickness{0.3mm}
\multiput(39.5,59)(0.12,0.12){10}{\line(1,0){0.12}}
\linethickness{0.3mm}
\multiput(40.75,60.25)(0.12,-0.12){10}{\line(1,0){0.12}}
\linethickness{0.3mm}
\multiput(42,59)(0.12,0.12){10}{\line(1,0){0.12}}
\linethickness{0.3mm}
\multiput(43.25,60.25)(0.12,-0.12){10}{\line(1,0){0.12}}
\linethickness{0.3mm}
\put(44.5,59){\line(1,0){1.25}}
\put(39.5,61.5){\makebox(0,0)[bl]{$R_2$}}

\linethickness{0.15mm}
\put(35.75,59){\line(1,0){1.25}}
\linethickness{0.3mm}
\put(18,54){\line(0,1){5}}
\linethickness{0.3mm}
\put(18,59){\line(1,0){19}}
\linethickness{0.3mm}
\put(31,53){\line(0,1){6}}
\linethickness{0.3mm}
\put(31,50){\circle{6}}

\linethickness{0.3mm}
\put(31,41){\line(0,1){6}}
\linethickness{0.3mm}
\put(18,41){\line(0,1){5}}
\linethickness{0.3mm}
\put(18,41){\line(1,0){13}}
\linethickness{0.3mm}
\put(31,59){\line(0,1){10}}
\linethickness{0.3mm}
\put(41,69){\circle{6}}

\linethickness{0.3mm}
\put(31,69){\line(1,0){7}}
\linethickness{0.3mm}
\put(44,69){\line(1,0){7}}
\linethickness{0.3mm}
\put(51,59){\line(0,1){10}}
\linethickness{0.3mm}
\put(45,59){\line(1,0){6}}
\linethickness{0.3mm}
\multiput(49.75,52.75)(0.12,0.12){10}{\line(1,0){0.12}}
\linethickness{0.3mm}
\multiput(49.75,52.75)(0.12,-0.12){10}{\line(1,0){0.12}}
\linethickness{0.3mm}
\multiput(49.75,50.25)(0.12,0.12){10}{\line(1,0){0.12}}
\linethickness{0.3mm}
\multiput(49.75,50.25)(0.12,-0.12){10}{\line(1,0){0.12}}
\linethickness{0.3mm}
\multiput(49.75,47.75)(0.12,0.12){10}{\line(1,0){0.12}}
\linethickness{0.3mm}
\multiput(49.75,47.75)(0.12,-0.12){10}{\line(1,0){0.12}}
\linethickness{0.3mm}
\put(51,45.25){\line(0,1){1.25}}
\put(52.25,49.25){\makebox(0,0)[bl]{$R_3$}}

\linethickness{0.15mm}
\put(51,54){\line(0,1){1.25}}
\linethickness{0.3mm}
\put(51,54){\line(0,1){5}}
\linethickness{0.3mm}
\put(51,41){\line(0,1){5}}
\linethickness{0.3mm}
\put(31,41){\line(1,0){20}}
\linethickness{0.3mm}
\multiput(59.25,59)(0.12,0.12){10}{\line(1,0){0.12}}
\linethickness{0.3mm}
\multiput(60.5,60.25)(0.12,-0.12){10}{\line(1,0){0.12}}
\linethickness{0.3mm}
\multiput(61.75,59)(0.12,0.12){10}{\line(1,0){0.12}}
\linethickness{0.3mm}
\multiput(63,60.25)(0.12,-0.12){10}{\line(1,0){0.12}}
\linethickness{0.3mm}
\multiput(64.25,59)(0.12,0.12){10}{\line(1,0){0.12}}
\linethickness{0.3mm}
\multiput(65.5,60.25)(0.12,-0.12){10}{\line(1,0){0.12}}
\linethickness{0.3mm}
\put(66.75,59){\line(1,0){1.25}}
\put(61.75,61.5){\makebox(0,0)[bl]{$R_4$}}

\linethickness{0.15mm}
\put(58,59){\line(1,0){1.25}}
\linethickness{0.3mm}
\put(51,59){\line(1,0){8}}
\linethickness{0.3mm}
\put(68,59){\line(1,0){7}}
\linethickness{0.3mm}
\put(75,50){\circle{6}}

\linethickness{0.3mm}
\put(75,53){\line(0,1){6}}
\linethickness{0.3mm}
\put(75,41){\line(0,1){6}}
\linethickness{0.3mm}
\put(51,41){\line(1,0){24}}
\linethickness{0.3mm}
\multiput(88.75,52.75)(0.12,0.12){10}{\line(1,0){0.12}}
\linethickness{0.3mm}
\multiput(88.75,52.75)(0.12,-0.12){10}{\line(1,0){0.12}}
\linethickness{0.3mm}
\multiput(88.75,50.25)(0.12,0.12){10}{\line(1,0){0.12}}
\linethickness{0.3mm}
\multiput(88.75,50.25)(0.12,-0.12){10}{\line(1,0){0.12}}
\linethickness{0.3mm}
\multiput(88.75,47.75)(0.12,0.12){10}{\line(1,0){0.12}}
\linethickness{0.3mm}
\multiput(88.75,47.75)(0.12,-0.12){10}{\line(1,0){0.12}}
\linethickness{0.3mm}
\put(90,45.25){\line(0,1){1.25}}
\put(91.25,49.25){\makebox(0,0)[bl]{$R_5$}}

\linethickness{0.15mm}
\put(90,54){\line(0,1){1.25}}
\linethickness{0.3mm}
\put(75,59){\line(1,0){15}}
\linethickness{0.3mm}
\put(90,54){\line(0,1){5}}
\linethickness{0.3mm}
\put(90,41){\line(0,1){5}}
\linethickness{0.3mm}
\put(75,41){\line(1,0){15}}
\linethickness{0.3mm}
\multiput(17,57)(0.12,0.12){8}{\line(1,0){0.12}}
\linethickness{0.3mm}
\multiput(18,58)(0.12,-0.12){8}{\line(1,0){0.12}}
\linethickness{0.3mm}
\multiput(49,60)(0.12,-0.12){8}{\line(1,0){0.12}}
\linethickness{0.3mm}
\multiput(49,58)(0.12,0.12){8}{\line(1,0){0.12}}
\linethickness{0.3mm}
\multiput(50,57)(0.12,0.12){8}{\line(1,0){0.12}}
\linethickness{0.3mm}
\multiput(51,58)(0.12,-0.12){8}{\line(1,0){0.12}}
\linethickness{0.3mm}
\multiput(73,60)(0.12,-0.12){8}{\line(1,0){0.12}}
\linethickness{0.3mm}
\multiput(73,58)(0.12,0.12){8}{\line(1,0){0.12}}
\linethickness{0.3mm}
\multiput(89,58)(0.12,-0.12){8}{\line(1,0){0.12}}
\linethickness{0.3mm}
\multiput(90,57)(0.12,0.12){8}{\line(1,0){0.12}}
\linethickness{0.3mm}
\put(75,48){\line(0,1){4}}
\put(75,48){\vector(0,-1){0.12}}
\linethickness{0.3mm}
\put(31,48){\line(0,1){4}}
\put(31,52){\vector(0,1){0.12}}
\linethickness{0.3mm}
\put(39,69){\line(1,0){4}}
\put(39,69){\vector(-1,0){0.12}}
\put(47,73){\makebox(0,0)[cc]{$J_2$}}

\put(26,62){\makebox(0,0)[cc]{$u_1$}}

\put(55,62){\makebox(0,0)[cc]{$u_2$}}

\put(76,62){\makebox(0,0)[cc]{$u_3$}}

\put(81,50){\makebox(0,0)[cc]{$J_3$}}

\put(38,50){\makebox(0,0)[cc]{$J_1$}}

\linethickness{0.3mm}
\put(91,40.99){\line(0,1){0.03}}
\multiput(90.99,40.96)(0.01,0.03){1}{\line(0,1){0.03}}
\multiput(90.98,40.93)(0.01,0.03){1}{\line(0,1){0.03}}
\multiput(90.96,40.91)(0.02,0.03){1}{\line(0,1){0.03}}
\multiput(90.93,40.88)(0.03,0.03){1}{\line(1,0){0.03}}
\multiput(90.9,40.85)(0.03,0.03){1}{\line(1,0){0.03}}
\multiput(90.86,40.82)(0.04,0.03){1}{\line(1,0){0.04}}
\multiput(90.81,40.8)(0.05,0.03){1}{\line(1,0){0.05}}
\multiput(90.76,40.77)(0.05,0.03){1}{\line(1,0){0.05}}
\multiput(90.69,40.74)(0.06,0.03){1}{\line(1,0){0.06}}
\multiput(90.63,40.72)(0.07,0.03){1}{\line(1,0){0.07}}
\multiput(90.55,40.69)(0.07,0.03){1}{\line(1,0){0.07}}
\multiput(90.47,40.66)(0.08,0.03){1}{\line(1,0){0.08}}
\multiput(90.38,40.64)(0.09,0.03){1}{\line(1,0){0.09}}
\multiput(90.29,40.61)(0.09,0.03){1}{\line(1,0){0.09}}
\multiput(90.19,40.58)(0.1,0.03){1}{\line(1,0){0.1}}
\multiput(90.08,40.56)(0.11,0.03){1}{\line(1,0){0.11}}
\multiput(89.97,40.53)(0.11,0.03){1}{\line(1,0){0.11}}
\multiput(89.85,40.5)(0.12,0.03){1}{\line(1,0){0.12}}
\multiput(89.72,40.48)(0.13,0.03){1}{\line(1,0){0.13}}
\multiput(89.58,40.45)(0.13,0.03){1}{\line(1,0){0.13}}
\multiput(89.44,40.43)(0.14,0.03){1}{\line(1,0){0.14}}
\multiput(89.3,40.4)(0.15,0.03){1}{\line(1,0){0.15}}
\multiput(89.14,40.37)(0.15,0.03){1}{\line(1,0){0.15}}
\multiput(88.98,40.35)(0.16,0.03){1}{\line(1,0){0.16}}
\multiput(88.82,40.32)(0.17,0.03){1}{\line(1,0){0.17}}
\multiput(88.64,40.3)(0.17,0.03){1}{\line(1,0){0.17}}
\multiput(88.46,40.27)(0.18,0.03){1}{\line(1,0){0.18}}
\multiput(88.28,40.25)(0.19,0.03){1}{\line(1,0){0.19}}
\multiput(88.09,40.22)(0.19,0.03){1}{\line(1,0){0.19}}
\multiput(87.89,40.2)(0.2,0.02){1}{\line(1,0){0.2}}
\multiput(87.68,40.17)(0.2,0.02){1}{\line(1,0){0.2}}
\multiput(87.47,40.15)(0.21,0.02){1}{\line(1,0){0.21}}
\multiput(87.26,40.12)(0.22,0.02){1}{\line(1,0){0.22}}
\multiput(87.04,40.1)(0.22,0.02){1}{\line(1,0){0.22}}
\multiput(86.81,40.08)(0.23,0.02){1}{\line(1,0){0.23}}
\multiput(86.57,40.05)(0.23,0.02){1}{\line(1,0){0.23}}
\multiput(86.33,40.03)(0.24,0.02){1}{\line(1,0){0.24}}
\multiput(86.08,40)(0.25,0.02){1}{\line(1,0){0.25}}
\multiput(85.83,39.98)(0.25,0.02){1}{\line(1,0){0.25}}
\multiput(85.57,39.96)(0.26,0.02){1}{\line(1,0){0.26}}
\multiput(85.31,39.93)(0.26,0.02){1}{\line(1,0){0.26}}
\multiput(85.04,39.91)(0.27,0.02){1}{\line(1,0){0.27}}
\multiput(84.76,39.89)(0.28,0.02){1}{\line(1,0){0.28}}
\multiput(84.48,39.87)(0.28,0.02){1}{\line(1,0){0.28}}
\multiput(84.2,39.84)(0.29,0.02){1}{\line(1,0){0.29}}
\multiput(83.9,39.82)(0.29,0.02){1}{\line(1,0){0.29}}
\multiput(83.61,39.8)(0.3,0.02){1}{\line(1,0){0.3}}
\multiput(83.3,39.78)(0.3,0.02){1}{\line(1,0){0.3}}
\multiput(82.99,39.76)(0.31,0.02){1}{\line(1,0){0.31}}
\multiput(82.68,39.74)(0.31,0.02){1}{\line(1,0){0.31}}
\multiput(82.36,39.72)(0.32,0.02){1}{\line(1,0){0.32}}
\multiput(82.04,39.69)(0.32,0.02){1}{\line(1,0){0.32}}
\multiput(81.71,39.67)(0.33,0.02){1}{\line(1,0){0.33}}
\multiput(81.37,39.65)(0.33,0.02){1}{\line(1,0){0.33}}
\multiput(81.03,39.63)(0.34,0.02){1}{\line(1,0){0.34}}
\multiput(80.69,39.61)(0.34,0.02){1}{\line(1,0){0.34}}
\multiput(80.34,39.6)(0.35,0.02){1}{\line(1,0){0.35}}
\multiput(79.99,39.58)(0.35,0.02){1}{\line(1,0){0.35}}
\multiput(79.63,39.56)(0.36,0.02){1}{\line(1,0){0.36}}
\multiput(79.26,39.54)(0.36,0.02){1}{\line(1,0){0.36}}
\multiput(78.89,39.52)(0.37,0.02){1}{\line(1,0){0.37}}
\multiput(78.52,39.5)(0.37,0.02){1}{\line(1,0){0.37}}
\multiput(78.14,39.48)(0.38,0.02){1}{\line(1,0){0.38}}
\multiput(77.76,39.47)(0.38,0.02){1}{\line(1,0){0.38}}
\multiput(77.38,39.45)(0.39,0.02){1}{\line(1,0){0.39}}
\multiput(76.99,39.43)(0.39,0.02){1}{\line(1,0){0.39}}
\multiput(76.59,39.42)(0.39,0.02){1}{\line(1,0){0.39}}
\multiput(76.19,39.4)(0.4,0.02){1}{\line(1,0){0.4}}
\multiput(75.79,39.38)(0.4,0.02){1}{\line(1,0){0.4}}
\multiput(75.38,39.37)(0.41,0.02){1}{\line(1,0){0.41}}
\multiput(74.97,39.35)(0.41,0.02){1}{\line(1,0){0.41}}
\multiput(74.56,39.34)(0.41,0.02){1}{\line(1,0){0.41}}
\multiput(74.14,39.32)(0.42,0.01){1}{\line(1,0){0.42}}
\multiput(73.72,39.31)(0.42,0.01){1}{\line(1,0){0.42}}
\multiput(73.29,39.29)(0.43,0.01){1}{\line(1,0){0.43}}
\multiput(72.86,39.28)(0.43,0.01){1}{\line(1,0){0.43}}
\multiput(72.43,39.27)(0.43,0.01){1}{\line(1,0){0.43}}
\multiput(71.99,39.25)(0.44,0.01){1}{\line(1,0){0.44}}
\multiput(71.55,39.24)(0.44,0.01){1}{\line(1,0){0.44}}
\multiput(71.11,39.23)(0.44,0.01){1}{\line(1,0){0.44}}
\multiput(70.66,39.21)(0.45,0.01){1}{\line(1,0){0.45}}
\multiput(70.21,39.2)(0.45,0.01){1}{\line(1,0){0.45}}
\multiput(69.76,39.19)(0.45,0.01){1}{\line(1,0){0.45}}
\multiput(69.31,39.18)(0.45,0.01){1}{\line(1,0){0.45}}
\multiput(68.85,39.17)(0.46,0.01){1}{\line(1,0){0.46}}
\multiput(68.39,39.16)(0.46,0.01){1}{\line(1,0){0.46}}
\multiput(67.93,39.15)(0.46,0.01){1}{\line(1,0){0.46}}
\multiput(67.46,39.14)(0.47,0.01){1}{\line(1,0){0.47}}
\multiput(66.99,39.13)(0.47,0.01){1}{\line(1,0){0.47}}
\multiput(66.52,39.12)(0.47,0.01){1}{\line(1,0){0.47}}
\multiput(66.05,39.11)(0.47,0.01){1}{\line(1,0){0.47}}
\multiput(65.58,39.1)(0.47,0.01){1}{\line(1,0){0.47}}
\multiput(65.1,39.09)(0.48,0.01){1}{\line(1,0){0.48}}
\multiput(64.62,39.08)(0.48,0.01){1}{\line(1,0){0.48}}
\multiput(64.14,39.08)(0.48,0.01){1}{\line(1,0){0.48}}
\multiput(63.66,39.07)(0.48,0.01){1}{\line(1,0){0.48}}
\multiput(63.17,39.06)(0.48,0.01){1}{\line(1,0){0.48}}
\multiput(62.69,39.06)(0.49,0.01){1}{\line(1,0){0.49}}
\multiput(62.2,39.05)(0.49,0.01){1}{\line(1,0){0.49}}
\multiput(61.71,39.04)(0.49,0.01){1}{\line(1,0){0.49}}
\multiput(61.22,39.04)(0.49,0.01){1}{\line(1,0){0.49}}
\multiput(60.73,39.03)(0.49,0.01){1}{\line(1,0){0.49}}
\multiput(60.23,39.03)(0.49,0){1}{\line(1,0){0.49}}
\multiput(59.74,39.02)(0.49,0){1}{\line(1,0){0.49}}
\multiput(59.24,39.02)(0.5,0){1}{\line(1,0){0.5}}
\multiput(58.75,39.02)(0.5,0){1}{\line(1,0){0.5}}
\multiput(58.25,39.01)(0.5,0){1}{\line(1,0){0.5}}
\multiput(57.75,39.01)(0.5,0){1}{\line(1,0){0.5}}
\multiput(57.25,39.01)(0.5,0){1}{\line(1,0){0.5}}
\multiput(56.75,39.01)(0.5,0){1}{\line(1,0){0.5}}
\multiput(56.25,39)(0.5,0){1}{\line(1,0){0.5}}
\multiput(55.75,39)(0.5,0){1}{\line(1,0){0.5}}
\multiput(55.25,39)(0.5,0){1}{\line(1,0){0.5}}
\multiput(54.75,39)(0.5,0){1}{\line(1,0){0.5}}
\put(54.25,39){\line(1,0){0.5}}
\put(53.75,39){\line(1,0){0.5}}
\put(53.25,39){\line(1,0){0.5}}
\multiput(52.75,39)(0.5,-0){1}{\line(1,0){0.5}}
\multiput(52.25,39)(0.5,-0){1}{\line(1,0){0.5}}
\multiput(51.75,39)(0.5,-0){1}{\line(1,0){0.5}}
\multiput(51.25,39.01)(0.5,-0){1}{\line(1,0){0.5}}
\multiput(50.75,39.01)(0.5,-0){1}{\line(1,0){0.5}}
\multiput(50.25,39.01)(0.5,-0){1}{\line(1,0){0.5}}
\multiput(49.75,39.01)(0.5,-0){1}{\line(1,0){0.5}}
\multiput(49.25,39.02)(0.5,-0){1}{\line(1,0){0.5}}
\multiput(48.76,39.02)(0.5,-0){1}{\line(1,0){0.5}}
\multiput(48.26,39.02)(0.5,-0){1}{\line(1,0){0.5}}
\multiput(47.77,39.03)(0.49,-0){1}{\line(1,0){0.49}}
\multiput(47.27,39.03)(0.49,-0){1}{\line(1,0){0.49}}
\multiput(46.78,39.04)(0.49,-0.01){1}{\line(1,0){0.49}}
\multiput(46.29,39.04)(0.49,-0.01){1}{\line(1,0){0.49}}
\multiput(45.8,39.05)(0.49,-0.01){1}{\line(1,0){0.49}}
\multiput(45.31,39.06)(0.49,-0.01){1}{\line(1,0){0.49}}
\multiput(44.83,39.06)(0.49,-0.01){1}{\line(1,0){0.49}}
\multiput(44.34,39.07)(0.48,-0.01){1}{\line(1,0){0.48}}
\multiput(43.86,39.08)(0.48,-0.01){1}{\line(1,0){0.48}}
\multiput(43.38,39.08)(0.48,-0.01){1}{\line(1,0){0.48}}
\multiput(42.9,39.09)(0.48,-0.01){1}{\line(1,0){0.48}}
\multiput(42.42,39.1)(0.48,-0.01){1}{\line(1,0){0.48}}
\multiput(41.95,39.11)(0.47,-0.01){1}{\line(1,0){0.47}}
\multiput(41.48,39.12)(0.47,-0.01){1}{\line(1,0){0.47}}
\multiput(41.01,39.13)(0.47,-0.01){1}{\line(1,0){0.47}}
\multiput(40.54,39.14)(0.47,-0.01){1}{\line(1,0){0.47}}
\multiput(40.07,39.15)(0.47,-0.01){1}{\line(1,0){0.47}}
\multiput(39.61,39.16)(0.46,-0.01){1}{\line(1,0){0.46}}
\multiput(39.15,39.17)(0.46,-0.01){1}{\line(1,0){0.46}}
\multiput(38.69,39.18)(0.46,-0.01){1}{\line(1,0){0.46}}
\multiput(38.24,39.19)(0.45,-0.01){1}{\line(1,0){0.45}}
\multiput(37.79,39.2)(0.45,-0.01){1}{\line(1,0){0.45}}
\multiput(37.34,39.21)(0.45,-0.01){1}{\line(1,0){0.45}}
\multiput(36.89,39.23)(0.45,-0.01){1}{\line(1,0){0.45}}
\multiput(36.45,39.24)(0.44,-0.01){1}{\line(1,0){0.44}}
\multiput(36.01,39.25)(0.44,-0.01){1}{\line(1,0){0.44}}
\multiput(35.57,39.27)(0.44,-0.01){1}{\line(1,0){0.44}}
\multiput(35.14,39.28)(0.43,-0.01){1}{\line(1,0){0.43}}
\multiput(34.71,39.29)(0.43,-0.01){1}{\line(1,0){0.43}}
\multiput(34.28,39.31)(0.43,-0.01){1}{\line(1,0){0.43}}
\multiput(33.86,39.32)(0.42,-0.01){1}{\line(1,0){0.42}}
\multiput(33.44,39.34)(0.42,-0.01){1}{\line(1,0){0.42}}
\multiput(33.03,39.35)(0.41,-0.02){1}{\line(1,0){0.41}}
\multiput(32.62,39.37)(0.41,-0.02){1}{\line(1,0){0.41}}
\multiput(32.21,39.38)(0.41,-0.02){1}{\line(1,0){0.41}}
\multiput(31.81,39.4)(0.4,-0.02){1}{\line(1,0){0.4}}
\multiput(31.41,39.42)(0.4,-0.02){1}{\line(1,0){0.4}}
\multiput(31.01,39.43)(0.39,-0.02){1}{\line(1,0){0.39}}
\multiput(30.62,39.45)(0.39,-0.02){1}{\line(1,0){0.39}}
\multiput(30.24,39.47)(0.39,-0.02){1}{\line(1,0){0.39}}
\multiput(29.86,39.48)(0.38,-0.02){1}{\line(1,0){0.38}}
\multiput(29.48,39.5)(0.38,-0.02){1}{\line(1,0){0.38}}
\multiput(29.11,39.52)(0.37,-0.02){1}{\line(1,0){0.37}}
\multiput(28.74,39.54)(0.37,-0.02){1}{\line(1,0){0.37}}
\multiput(28.37,39.56)(0.36,-0.02){1}{\line(1,0){0.36}}
\multiput(28.01,39.58)(0.36,-0.02){1}{\line(1,0){0.36}}
\multiput(27.66,39.6)(0.35,-0.02){1}{\line(1,0){0.35}}
\multiput(27.31,39.61)(0.35,-0.02){1}{\line(1,0){0.35}}
\multiput(26.97,39.63)(0.34,-0.02){1}{\line(1,0){0.34}}
\multiput(26.63,39.65)(0.34,-0.02){1}{\line(1,0){0.34}}
\multiput(26.29,39.67)(0.33,-0.02){1}{\line(1,0){0.33}}
\multiput(25.96,39.69)(0.33,-0.02){1}{\line(1,0){0.33}}
\multiput(25.64,39.72)(0.32,-0.02){1}{\line(1,0){0.32}}
\multiput(25.32,39.74)(0.32,-0.02){1}{\line(1,0){0.32}}
\multiput(25.01,39.76)(0.31,-0.02){1}{\line(1,0){0.31}}
\multiput(24.7,39.78)(0.31,-0.02){1}{\line(1,0){0.31}}
\multiput(24.39,39.8)(0.3,-0.02){1}{\line(1,0){0.3}}
\multiput(24.1,39.82)(0.3,-0.02){1}{\line(1,0){0.3}}
\multiput(23.8,39.84)(0.29,-0.02){1}{\line(1,0){0.29}}
\multiput(23.52,39.87)(0.29,-0.02){1}{\line(1,0){0.29}}
\multiput(23.24,39.89)(0.28,-0.02){1}{\line(1,0){0.28}}
\multiput(22.96,39.91)(0.28,-0.02){1}{\line(1,0){0.28}}
\multiput(22.69,39.93)(0.27,-0.02){1}{\line(1,0){0.27}}
\multiput(22.43,39.96)(0.26,-0.02){1}{\line(1,0){0.26}}
\multiput(22.17,39.98)(0.26,-0.02){1}{\line(1,0){0.26}}
\multiput(21.92,40)(0.25,-0.02){1}{\line(1,0){0.25}}
\multiput(21.67,40.03)(0.25,-0.02){1}{\line(1,0){0.25}}
\multiput(21.43,40.05)(0.24,-0.02){1}{\line(1,0){0.24}}
\multiput(21.19,40.08)(0.23,-0.02){1}{\line(1,0){0.23}}
\multiput(20.96,40.1)(0.23,-0.02){1}{\line(1,0){0.23}}
\multiput(20.74,40.12)(0.22,-0.02){1}{\line(1,0){0.22}}
\multiput(20.53,40.15)(0.22,-0.02){1}{\line(1,0){0.22}}
\multiput(20.32,40.17)(0.21,-0.02){1}{\line(1,0){0.21}}
\multiput(20.11,40.2)(0.2,-0.02){1}{\line(1,0){0.2}}
\multiput(19.91,40.22)(0.2,-0.02){1}{\line(1,0){0.2}}
\multiput(19.72,40.25)(0.19,-0.03){1}{\line(1,0){0.19}}
\multiput(19.54,40.27)(0.19,-0.03){1}{\line(1,0){0.19}}
\multiput(19.36,40.3)(0.18,-0.03){1}{\line(1,0){0.18}}
\multiput(19.18,40.32)(0.17,-0.03){1}{\line(1,0){0.17}}
\multiput(19.02,40.35)(0.17,-0.03){1}{\line(1,0){0.17}}
\multiput(18.86,40.37)(0.16,-0.03){1}{\line(1,0){0.16}}
\multiput(18.7,40.4)(0.15,-0.03){1}{\line(1,0){0.15}}
\multiput(18.56,40.43)(0.15,-0.03){1}{\line(1,0){0.15}}
\multiput(18.42,40.45)(0.14,-0.03){1}{\line(1,0){0.14}}
\multiput(18.28,40.48)(0.13,-0.03){1}{\line(1,0){0.13}}
\multiput(18.15,40.5)(0.13,-0.03){1}{\line(1,0){0.13}}
\multiput(18.03,40.53)(0.12,-0.03){1}{\line(1,0){0.12}}
\multiput(17.92,40.56)(0.11,-0.03){1}{\line(1,0){0.11}}
\multiput(17.81,40.58)(0.11,-0.03){1}{\line(1,0){0.11}}
\multiput(17.71,40.61)(0.1,-0.03){1}{\line(1,0){0.1}}
\multiput(17.62,40.64)(0.09,-0.03){1}{\line(1,0){0.09}}
\multiput(17.53,40.66)(0.09,-0.03){1}{\line(1,0){0.09}}
\multiput(17.45,40.69)(0.08,-0.03){1}{\line(1,0){0.08}}
\multiput(17.37,40.72)(0.07,-0.03){1}{\line(1,0){0.07}}
\multiput(17.31,40.74)(0.07,-0.03){1}{\line(1,0){0.07}}
\multiput(17.24,40.77)(0.06,-0.03){1}{\line(1,0){0.06}}
\multiput(17.19,40.8)(0.05,-0.03){1}{\line(1,0){0.05}}
\multiput(17.14,40.82)(0.05,-0.03){1}{\line(1,0){0.05}}
\multiput(17.1,40.85)(0.04,-0.03){1}{\line(1,0){0.04}}
\multiput(17.07,40.88)(0.03,-0.03){1}{\line(1,0){0.03}}
\multiput(17.04,40.91)(0.03,-0.03){1}{\line(1,0){0.03}}
\multiput(17.02,40.93)(0.02,-0.03){1}{\line(0,-1){0.03}}
\multiput(17.01,40.96)(0.01,-0.03){1}{\line(0,-1){0.03}}
\multiput(17,40.99)(0.01,-0.03){1}{\line(0,-1){0.03}}
\put(17,40.99){\line(0,1){0.03}}
\multiput(17,41.01)(0.01,0.03){1}{\line(0,1){0.03}}
\multiput(17.01,41.04)(0.01,0.03){1}{\line(0,1){0.03}}
\multiput(17.02,41.07)(0.02,0.03){1}{\line(0,1){0.03}}
\multiput(17.04,41.09)(0.03,0.03){1}{\line(1,0){0.03}}
\multiput(17.07,41.12)(0.03,0.03){1}{\line(1,0){0.03}}
\multiput(17.1,41.15)(0.04,0.03){1}{\line(1,0){0.04}}
\multiput(17.14,41.18)(0.05,0.03){1}{\line(1,0){0.05}}
\multiput(17.19,41.2)(0.05,0.03){1}{\line(1,0){0.05}}
\multiput(17.24,41.23)(0.06,0.03){1}{\line(1,0){0.06}}
\multiput(17.31,41.26)(0.07,0.03){1}{\line(1,0){0.07}}
\multiput(17.37,41.28)(0.07,0.03){1}{\line(1,0){0.07}}
\multiput(17.45,41.31)(0.08,0.03){1}{\line(1,0){0.08}}
\multiput(17.53,41.34)(0.09,0.03){1}{\line(1,0){0.09}}
\multiput(17.62,41.36)(0.09,0.03){1}{\line(1,0){0.09}}
\multiput(17.71,41.39)(0.1,0.03){1}{\line(1,0){0.1}}
\multiput(17.81,41.42)(0.11,0.03){1}{\line(1,0){0.11}}
\multiput(17.92,41.44)(0.11,0.03){1}{\line(1,0){0.11}}
\multiput(18.03,41.47)(0.12,0.03){1}{\line(1,0){0.12}}
\multiput(18.15,41.5)(0.13,0.03){1}{\line(1,0){0.13}}
\multiput(18.28,41.52)(0.13,0.03){1}{\line(1,0){0.13}}
\multiput(18.42,41.55)(0.14,0.03){1}{\line(1,0){0.14}}
\multiput(18.56,41.57)(0.15,0.03){1}{\line(1,0){0.15}}
\multiput(18.7,41.6)(0.15,0.03){1}{\line(1,0){0.15}}
\multiput(18.86,41.63)(0.16,0.03){1}{\line(1,0){0.16}}
\multiput(19.02,41.65)(0.17,0.03){1}{\line(1,0){0.17}}
\multiput(19.18,41.68)(0.17,0.03){1}{\line(1,0){0.17}}
\multiput(19.36,41.7)(0.18,0.03){1}{\line(1,0){0.18}}
\multiput(19.54,41.73)(0.19,0.03){1}{\line(1,0){0.19}}
\multiput(19.72,41.75)(0.19,0.03){1}{\line(1,0){0.19}}
\multiput(19.91,41.78)(0.2,0.02){1}{\line(1,0){0.2}}
\multiput(20.11,41.8)(0.2,0.02){1}{\line(1,0){0.2}}
\multiput(20.32,41.83)(0.21,0.02){1}{\line(1,0){0.21}}
\multiput(20.53,41.85)(0.22,0.02){1}{\line(1,0){0.22}}
\multiput(20.74,41.88)(0.22,0.02){1}{\line(1,0){0.22}}
\multiput(20.96,41.9)(0.23,0.02){1}{\line(1,0){0.23}}
\multiput(21.19,41.92)(0.23,0.02){1}{\line(1,0){0.23}}
\multiput(21.43,41.95)(0.24,0.02){1}{\line(1,0){0.24}}
\multiput(21.67,41.97)(0.25,0.02){1}{\line(1,0){0.25}}
\multiput(21.92,42)(0.25,0.02){1}{\line(1,0){0.25}}
\multiput(22.17,42.02)(0.26,0.02){1}{\line(1,0){0.26}}
\multiput(22.43,42.04)(0.26,0.02){1}{\line(1,0){0.26}}
\multiput(22.69,42.07)(0.27,0.02){1}{\line(1,0){0.27}}
\multiput(22.96,42.09)(0.28,0.02){1}{\line(1,0){0.28}}
\multiput(23.24,42.11)(0.28,0.02){1}{\line(1,0){0.28}}
\multiput(23.52,42.13)(0.29,0.02){1}{\line(1,0){0.29}}
\multiput(23.8,42.16)(0.29,0.02){1}{\line(1,0){0.29}}
\multiput(24.1,42.18)(0.3,0.02){1}{\line(1,0){0.3}}
\multiput(24.39,42.2)(0.3,0.02){1}{\line(1,0){0.3}}
\multiput(24.7,42.22)(0.31,0.02){1}{\line(1,0){0.31}}
\multiput(25.01,42.24)(0.31,0.02){1}{\line(1,0){0.31}}
\multiput(25.32,42.26)(0.32,0.02){1}{\line(1,0){0.32}}
\multiput(25.64,42.28)(0.32,0.02){1}{\line(1,0){0.32}}
\multiput(25.96,42.31)(0.33,0.02){1}{\line(1,0){0.33}}
\multiput(26.29,42.33)(0.33,0.02){1}{\line(1,0){0.33}}
\multiput(26.63,42.35)(0.34,0.02){1}{\line(1,0){0.34}}
\multiput(26.97,42.37)(0.34,0.02){1}{\line(1,0){0.34}}
\multiput(27.31,42.39)(0.35,0.02){1}{\line(1,0){0.35}}
\multiput(27.66,42.4)(0.35,0.02){1}{\line(1,0){0.35}}
\multiput(28.01,42.42)(0.36,0.02){1}{\line(1,0){0.36}}
\multiput(28.37,42.44)(0.36,0.02){1}{\line(1,0){0.36}}
\multiput(28.74,42.46)(0.37,0.02){1}{\line(1,0){0.37}}
\multiput(29.11,42.48)(0.37,0.02){1}{\line(1,0){0.37}}
\multiput(29.48,42.5)(0.38,0.02){1}{\line(1,0){0.38}}
\multiput(29.86,42.52)(0.38,0.02){1}{\line(1,0){0.38}}
\multiput(30.24,42.53)(0.39,0.02){1}{\line(1,0){0.39}}
\multiput(30.62,42.55)(0.39,0.02){1}{\line(1,0){0.39}}
\multiput(31.01,42.57)(0.39,0.02){1}{\line(1,0){0.39}}
\multiput(31.41,42.58)(0.4,0.02){1}{\line(1,0){0.4}}
\multiput(31.81,42.6)(0.4,0.02){1}{\line(1,0){0.4}}
\multiput(32.21,42.62)(0.41,0.02){1}{\line(1,0){0.41}}
\multiput(32.62,42.63)(0.41,0.02){1}{\line(1,0){0.41}}
\multiput(33.03,42.65)(0.41,0.02){1}{\line(1,0){0.41}}
\multiput(33.44,42.66)(0.42,0.01){1}{\line(1,0){0.42}}
\multiput(33.86,42.68)(0.42,0.01){1}{\line(1,0){0.42}}
\multiput(34.28,42.69)(0.43,0.01){1}{\line(1,0){0.43}}
\multiput(34.71,42.71)(0.43,0.01){1}{\line(1,0){0.43}}
\multiput(35.14,42.72)(0.43,0.01){1}{\line(1,0){0.43}}
\multiput(35.57,42.73)(0.44,0.01){1}{\line(1,0){0.44}}
\multiput(36.01,42.75)(0.44,0.01){1}{\line(1,0){0.44}}
\multiput(36.45,42.76)(0.44,0.01){1}{\line(1,0){0.44}}
\multiput(36.89,42.77)(0.45,0.01){1}{\line(1,0){0.45}}
\multiput(37.34,42.79)(0.45,0.01){1}{\line(1,0){0.45}}
\multiput(37.79,42.8)(0.45,0.01){1}{\line(1,0){0.45}}
\multiput(38.24,42.81)(0.45,0.01){1}{\line(1,0){0.45}}
\multiput(38.69,42.82)(0.46,0.01){1}{\line(1,0){0.46}}
\multiput(39.15,42.83)(0.46,0.01){1}{\line(1,0){0.46}}
\multiput(39.61,42.84)(0.46,0.01){1}{\line(1,0){0.46}}
\multiput(40.07,42.85)(0.47,0.01){1}{\line(1,0){0.47}}
\multiput(40.54,42.86)(0.47,0.01){1}{\line(1,0){0.47}}
\multiput(41.01,42.87)(0.47,0.01){1}{\line(1,0){0.47}}
\multiput(41.48,42.88)(0.47,0.01){1}{\line(1,0){0.47}}
\multiput(41.95,42.89)(0.47,0.01){1}{\line(1,0){0.47}}
\multiput(42.42,42.9)(0.48,0.01){1}{\line(1,0){0.48}}
\multiput(42.9,42.91)(0.48,0.01){1}{\line(1,0){0.48}}
\multiput(43.38,42.92)(0.48,0.01){1}{\line(1,0){0.48}}
\multiput(43.86,42.92)(0.48,0.01){1}{\line(1,0){0.48}}
\multiput(44.34,42.93)(0.48,0.01){1}{\line(1,0){0.48}}
\multiput(44.83,42.94)(0.49,0.01){1}{\line(1,0){0.49}}
\multiput(45.31,42.94)(0.49,0.01){1}{\line(1,0){0.49}}
\multiput(45.8,42.95)(0.49,0.01){1}{\line(1,0){0.49}}
\multiput(46.29,42.96)(0.49,0.01){1}{\line(1,0){0.49}}
\multiput(46.78,42.96)(0.49,0.01){1}{\line(1,0){0.49}}
\multiput(47.27,42.97)(0.49,0){1}{\line(1,0){0.49}}
\multiput(47.77,42.97)(0.49,0){1}{\line(1,0){0.49}}
\multiput(48.26,42.98)(0.5,0){1}{\line(1,0){0.5}}
\multiput(48.76,42.98)(0.5,0){1}{\line(1,0){0.5}}
\multiput(49.25,42.98)(0.5,0){1}{\line(1,0){0.5}}
\multiput(49.75,42.99)(0.5,0){1}{\line(1,0){0.5}}
\multiput(50.25,42.99)(0.5,0){1}{\line(1,0){0.5}}
\multiput(50.75,42.99)(0.5,0){1}{\line(1,0){0.5}}
\multiput(51.25,42.99)(0.5,0){1}{\line(1,0){0.5}}
\multiput(51.75,43)(0.5,0){1}{\line(1,0){0.5}}
\multiput(52.25,43)(0.5,0){1}{\line(1,0){0.5}}
\multiput(52.75,43)(0.5,0){1}{\line(1,0){0.5}}
\put(53.25,43){\line(1,0){0.5}}
\put(53.75,43){\line(1,0){0.5}}
\put(54.25,43){\line(1,0){0.5}}
\multiput(54.75,43)(0.5,-0){1}{\line(1,0){0.5}}
\multiput(55.25,43)(0.5,-0){1}{\line(1,0){0.5}}
\multiput(55.75,43)(0.5,-0){1}{\line(1,0){0.5}}
\multiput(56.25,43)(0.5,-0){1}{\line(1,0){0.5}}
\multiput(56.75,42.99)(0.5,-0){1}{\line(1,0){0.5}}
\multiput(57.25,42.99)(0.5,-0){1}{\line(1,0){0.5}}
\multiput(57.75,42.99)(0.5,-0){1}{\line(1,0){0.5}}
\multiput(58.25,42.99)(0.5,-0){1}{\line(1,0){0.5}}
\multiput(58.75,42.98)(0.5,-0){1}{\line(1,0){0.5}}
\multiput(59.24,42.98)(0.5,-0){1}{\line(1,0){0.5}}
\multiput(59.74,42.98)(0.49,-0){1}{\line(1,0){0.49}}
\multiput(60.23,42.97)(0.49,-0){1}{\line(1,0){0.49}}
\multiput(60.73,42.97)(0.49,-0.01){1}{\line(1,0){0.49}}
\multiput(61.22,42.96)(0.49,-0.01){1}{\line(1,0){0.49}}
\multiput(61.71,42.96)(0.49,-0.01){1}{\line(1,0){0.49}}
\multiput(62.2,42.95)(0.49,-0.01){1}{\line(1,0){0.49}}
\multiput(62.69,42.94)(0.49,-0.01){1}{\line(1,0){0.49}}
\multiput(63.17,42.94)(0.48,-0.01){1}{\line(1,0){0.48}}
\multiput(63.66,42.93)(0.48,-0.01){1}{\line(1,0){0.48}}
\multiput(64.14,42.92)(0.48,-0.01){1}{\line(1,0){0.48}}
\multiput(64.62,42.92)(0.48,-0.01){1}{\line(1,0){0.48}}
\multiput(65.1,42.91)(0.48,-0.01){1}{\line(1,0){0.48}}
\multiput(65.58,42.9)(0.47,-0.01){1}{\line(1,0){0.47}}
\multiput(66.05,42.89)(0.47,-0.01){1}{\line(1,0){0.47}}
\multiput(66.52,42.88)(0.47,-0.01){1}{\line(1,0){0.47}}
\multiput(66.99,42.87)(0.47,-0.01){1}{\line(1,0){0.47}}
\multiput(67.46,42.86)(0.47,-0.01){1}{\line(1,0){0.47}}
\multiput(67.93,42.85)(0.46,-0.01){1}{\line(1,0){0.46}}
\multiput(68.39,42.84)(0.46,-0.01){1}{\line(1,0){0.46}}
\multiput(68.85,42.83)(0.46,-0.01){1}{\line(1,0){0.46}}
\multiput(69.31,42.82)(0.45,-0.01){1}{\line(1,0){0.45}}
\multiput(69.76,42.81)(0.45,-0.01){1}{\line(1,0){0.45}}
\multiput(70.21,42.8)(0.45,-0.01){1}{\line(1,0){0.45}}
\multiput(70.66,42.79)(0.45,-0.01){1}{\line(1,0){0.45}}
\multiput(71.11,42.77)(0.44,-0.01){1}{\line(1,0){0.44}}
\multiput(71.55,42.76)(0.44,-0.01){1}{\line(1,0){0.44}}
\multiput(71.99,42.75)(0.44,-0.01){1}{\line(1,0){0.44}}
\multiput(72.43,42.73)(0.43,-0.01){1}{\line(1,0){0.43}}
\multiput(72.86,42.72)(0.43,-0.01){1}{\line(1,0){0.43}}
\multiput(73.29,42.71)(0.43,-0.01){1}{\line(1,0){0.43}}
\multiput(73.72,42.69)(0.42,-0.01){1}{\line(1,0){0.42}}
\multiput(74.14,42.68)(0.42,-0.01){1}{\line(1,0){0.42}}
\multiput(74.56,42.66)(0.41,-0.02){1}{\line(1,0){0.41}}
\multiput(74.97,42.65)(0.41,-0.02){1}{\line(1,0){0.41}}
\multiput(75.38,42.63)(0.41,-0.02){1}{\line(1,0){0.41}}
\multiput(75.79,42.62)(0.4,-0.02){1}{\line(1,0){0.4}}
\multiput(76.19,42.6)(0.4,-0.02){1}{\line(1,0){0.4}}
\multiput(76.59,42.58)(0.39,-0.02){1}{\line(1,0){0.39}}
\multiput(76.99,42.57)(0.39,-0.02){1}{\line(1,0){0.39}}
\multiput(77.38,42.55)(0.39,-0.02){1}{\line(1,0){0.39}}
\multiput(77.76,42.53)(0.38,-0.02){1}{\line(1,0){0.38}}
\multiput(78.14,42.52)(0.38,-0.02){1}{\line(1,0){0.38}}
\multiput(78.52,42.5)(0.37,-0.02){1}{\line(1,0){0.37}}
\multiput(78.89,42.48)(0.37,-0.02){1}{\line(1,0){0.37}}
\multiput(79.26,42.46)(0.36,-0.02){1}{\line(1,0){0.36}}
\multiput(79.63,42.44)(0.36,-0.02){1}{\line(1,0){0.36}}
\multiput(79.99,42.42)(0.35,-0.02){1}{\line(1,0){0.35}}
\multiput(80.34,42.4)(0.35,-0.02){1}{\line(1,0){0.35}}
\multiput(80.69,42.39)(0.34,-0.02){1}{\line(1,0){0.34}}
\multiput(81.03,42.37)(0.34,-0.02){1}{\line(1,0){0.34}}
\multiput(81.37,42.35)(0.33,-0.02){1}{\line(1,0){0.33}}
\multiput(81.71,42.33)(0.33,-0.02){1}{\line(1,0){0.33}}
\multiput(82.04,42.31)(0.32,-0.02){1}{\line(1,0){0.32}}
\multiput(82.36,42.28)(0.32,-0.02){1}{\line(1,0){0.32}}
\multiput(82.68,42.26)(0.31,-0.02){1}{\line(1,0){0.31}}
\multiput(82.99,42.24)(0.31,-0.02){1}{\line(1,0){0.31}}
\multiput(83.3,42.22)(0.3,-0.02){1}{\line(1,0){0.3}}
\multiput(83.61,42.2)(0.3,-0.02){1}{\line(1,0){0.3}}
\multiput(83.9,42.18)(0.29,-0.02){1}{\line(1,0){0.29}}
\multiput(84.2,42.16)(0.29,-0.02){1}{\line(1,0){0.29}}
\multiput(84.48,42.13)(0.28,-0.02){1}{\line(1,0){0.28}}
\multiput(84.76,42.11)(0.28,-0.02){1}{\line(1,0){0.28}}
\multiput(85.04,42.09)(0.27,-0.02){1}{\line(1,0){0.27}}
\multiput(85.31,42.07)(0.26,-0.02){1}{\line(1,0){0.26}}
\multiput(85.57,42.04)(0.26,-0.02){1}{\line(1,0){0.26}}
\multiput(85.83,42.02)(0.25,-0.02){1}{\line(1,0){0.25}}
\multiput(86.08,42)(0.25,-0.02){1}{\line(1,0){0.25}}
\multiput(86.33,41.97)(0.24,-0.02){1}{\line(1,0){0.24}}
\multiput(86.57,41.95)(0.23,-0.02){1}{\line(1,0){0.23}}
\multiput(86.81,41.92)(0.23,-0.02){1}{\line(1,0){0.23}}
\multiput(87.04,41.9)(0.22,-0.02){1}{\line(1,0){0.22}}
\multiput(87.26,41.88)(0.22,-0.02){1}{\line(1,0){0.22}}
\multiput(87.47,41.85)(0.21,-0.02){1}{\line(1,0){0.21}}
\multiput(87.68,41.83)(0.2,-0.02){1}{\line(1,0){0.2}}
\multiput(87.89,41.8)(0.2,-0.02){1}{\line(1,0){0.2}}
\multiput(88.09,41.78)(0.19,-0.03){1}{\line(1,0){0.19}}
\multiput(88.28,41.75)(0.19,-0.03){1}{\line(1,0){0.19}}
\multiput(88.46,41.73)(0.18,-0.03){1}{\line(1,0){0.18}}
\multiput(88.64,41.7)(0.17,-0.03){1}{\line(1,0){0.17}}
\multiput(88.82,41.68)(0.17,-0.03){1}{\line(1,0){0.17}}
\multiput(88.98,41.65)(0.16,-0.03){1}{\line(1,0){0.16}}
\multiput(89.14,41.63)(0.15,-0.03){1}{\line(1,0){0.15}}
\multiput(89.3,41.6)(0.15,-0.03){1}{\line(1,0){0.15}}
\multiput(89.44,41.57)(0.14,-0.03){1}{\line(1,0){0.14}}
\multiput(89.58,41.55)(0.13,-0.03){1}{\line(1,0){0.13}}
\multiput(89.72,41.52)(0.13,-0.03){1}{\line(1,0){0.13}}
\multiput(89.85,41.5)(0.12,-0.03){1}{\line(1,0){0.12}}
\multiput(89.97,41.47)(0.11,-0.03){1}{\line(1,0){0.11}}
\multiput(90.08,41.44)(0.11,-0.03){1}{\line(1,0){0.11}}
\multiput(90.19,41.42)(0.1,-0.03){1}{\line(1,0){0.1}}
\multiput(90.29,41.39)(0.09,-0.03){1}{\line(1,0){0.09}}
\multiput(90.38,41.36)(0.09,-0.03){1}{\line(1,0){0.09}}
\multiput(90.47,41.34)(0.08,-0.03){1}{\line(1,0){0.08}}
\multiput(90.55,41.31)(0.07,-0.03){1}{\line(1,0){0.07}}
\multiput(90.63,41.28)(0.07,-0.03){1}{\line(1,0){0.07}}
\multiput(90.69,41.26)(0.06,-0.03){1}{\line(1,0){0.06}}
\multiput(90.76,41.23)(0.05,-0.03){1}{\line(1,0){0.05}}
\multiput(90.81,41.2)(0.05,-0.03){1}{\line(1,0){0.05}}
\multiput(90.86,41.18)(0.04,-0.03){1}{\line(1,0){0.04}}
\multiput(90.9,41.15)(0.03,-0.03){1}{\line(1,0){0.03}}
\multiput(90.93,41.12)(0.03,-0.03){1}{\line(1,0){0.03}}
\multiput(90.96,41.09)(0.02,-0.03){1}{\line(0,-1){0.03}}
\multiput(90.98,41.07)(0.01,-0.03){1}{\line(0,-1){0.03}}
\multiput(90.99,41.04)(0.01,-0.03){1}{\line(0,-1){0.03}}

\put(54,37){\makebox(0,0)[cc]{$u_4=0$}}
\put(31,59){\circle*{1.5}}
\put(51,59){\circle*{1.5}}
\put(75,59){\circle*{1.5}}

\put(56,20){\makebox(0,0)[cc]{{\bf Figura 2.8} Esempio di circuito di
    soli resistori e generatori di corrente.}}
\end{picture}

Esso consta di $4$ nodi e $8$ lati. Abbiamo scelto di considerare il
nodo $0$ come riferimento dei potenziali ($u_0=0$).
In riferimento ai versi adottati per le intensit\`a di corrente dei
resistori, le $LKC$ ai nodi 1, 2, 3 sono

\[
\left\{
\begin{array}{l l l}
  -i_1+i_2=J_1+J_2\\
  -i_2-i_3+i_4=-J_2\\
  -i_4+i_5=-J_3
\end{array}\right.
\]

Ogni intesit\`a di corrente incognita del precedente sistema pu\`o
essere espressa in funzione delle tensione (e quindi in funzione dei
potenziali di nodo) mediante le equazioni caratteristiche dei
resistori. Se adottiamo, ad esempio, la convenzione dell'utilizzatore
abbiamo per le tensioni:

\[
\left\{
\begin{array}{l l l l l l l l}
  v_1=-u_1\\
  v_{j_1}=-u_1\\
  v_2=u_1-u_2\\
  v_{j_2}=-u_1+u_2\\
  v_3=-u_3\\
  v_4=u_2-u_3\\
  v_{j_3}=u_3\\
  v_5=u_3 
\end{array} \right.
\]

Utilizzando ora le equazioni caratteristiche dei singoli bipoli si
ottengono le relazioni che esprimono le intensit\`a di corrente in
termini di potenziali di nodo:

\[
i_1=\dfrac{-u_1}{R_1},\quad i_2=\dfrac{u_1-u_2}{R_2},\quad
i_3=\dfrac{-u_2}{R_3},\quad i_4=\dfrac{u_2-u_3}{R_4},\quad i_5=\dfrac{u_3}{R_5} 
\]

Sostiutendo queste realzioni nelle $LKC$ si ottengono le equazioni per
i potenziali di nodo:

\[
\dfrac{-u_1}{R_1}+\dfrac{u_2-u_1}{R_2}=J_1+J_2,\quad -\dfrac{u_2-u_1}{R_2}-
\dfrac{-u_2}{R_3}+\dfrac{u_2-u_3}{R_4}=-J_2,\quad
-\dfrac{u_2-u_3}{R_4}+\dfrac{u_3}{R_5}=-J_3
\]

Questo \`e un sistema di tre equazioni in tre incognite ($u_1, u_2,
u_3$). Una volta determinati i valori dei potenziali dei nodi, le
tensioni sui diversi bipoli possono essere espresse attraverso le
relazioni con questi ultimi.

\section{Correnti di Maglia}
Secondo il seguente grafo, avendo preso come versi di riferimento $+$ per
le correnti di maglia con versi di riferimento concordi con quello
dell'intensit\`a di corrente di lato e con $-$ tutte le correnti di
maglia che hanno versi di riferimento discordi:

\[
i_m=\sum_h(\pm) k_h
\]

\ifx\JPicScale\undefined\def\JPicScale{1}\fi
\unitlength \JPicScale mm
\begin{picture}(99.5,58)(10,15)
  \linethickness{0.3mm}
  \put(35,30){\line(1,0){30}}
  \multiput(35,30)(1.03,1.72){15}{\multiput(0,0)(0.13,0.22){4}{\line(0,1){0.22}}}
  \linethickness{0.3mm}
  \multiput(50,55)(0.12,-0.2){125}{\line(0,-1){0.2}}
  \linethickness{0.3mm}
  \put(65,30){\line(1,0){30}}
  \multiput(80,55)(1.03,-1.72){15}{\multiput(0,0)(0.13,-0.22){4}{\line(0,-1){0.22}}}
  \linethickness{0.3mm}
  \multiput(65,30)(0.12,0.2){125}{\line(0,1){0.2}}
  \multiput(50,55)(1.94,0){16}{\line(1,0){0.97}}
  \linethickness{0.3mm}
  \multiput(49,31)(0.12,-0.12){8}{\line(1,0){0.12}}
  \linethickness{0.3mm}
  \multiput(49,29)(0.12,0.12){8}{\line(1,0){0.12}}
  \linethickness{0.3mm}
  \put(42,43){\line(1,0){1}}
  \linethickness{0.3mm}
  \put(43,42){\line(0,1){1}}
  \linethickness{0.3mm}
  \put(57,42){\line(0,1){1}}
  \linethickness{0.3mm}
  \put(57,43){\line(1,0){1}}
  \linethickness{0.3mm}
  \multiput(64,56)(0.12,-0.12){8}{\line(1,0){0.12}}
  \linethickness{0.3mm}
  \multiput(64,54)(0.12,0.12){8}{\line(1,0){0.12}}
  \linethickness{0.3mm}
  \put(72,43){\line(1,0){1}}
  \linethickness{0.3mm}
  \put(73,42){\line(0,1){1}}
  \linethickness{0.3mm}
  \multiput(79,31)(0.12,-0.12){8}{\line(1,0){0.12}}
  \linethickness{0.3mm}
  \multiput(79,29)(0.12,0.12){8}{\line(1,0){0.12}}
  \linethickness{0.3mm}
  \put(87,42){\line(0,1){1}}
  \linethickness{0.3mm}
  \put(87,43){\line(1,0){1}}
  \linethickness{0.3mm}
  \put(32.5,27.5){\circle{3}}
  
  \linethickness{0.3mm}
  \put(98,27.5){\circle{3}}
  
  \linethickness{0.3mm}
  \put(47,56.5){\circle{3}}
  
  \linethickness{0.3mm}
  \put(83.5,56.5){\circle{3}}
  
  \put(50,55){\circle*{1.5}}
  \put(80,55){\circle*{1.5}}
  \put(35,30){\circle*{1.5}}
  \put(65,30){\circle*{1.5}}
  \put(95,30){\circle*{1.5}}
  
  \put(32.5,27.5){\makebox(0,0)[cc]{$5$}}
  
  \put(46.9,56.5){\makebox(0,0)[cc]{$1$}}
  
  \put(83.51,56.5){\makebox(0,0)[cc]{$2$}}

  \put(98,27.5){\makebox(0,0)[cc]{$3$}}
  
  \linethickness{0.3mm}
  \put(65,27.5){\circle{3}}
  
  \put(64.8,27.5){\makebox(0,0)[cc]{$4$}}
  
  \put(40,43){\makebox(0,0)[cc]{$1$}}
  
  \put(64.5,57.5){\makebox(0,0)[cc]{$2$}}
  
  \put(90,43){\makebox(0,0)[cc]{$3$}}
  
  \put(79,27){\makebox(0,0)[cc]{$4$}}
  
  \put(75,42){\makebox(0,0)[cc]{$5$}}
  
  \put(60,43){\makebox(0,0)[cc]{$6$}}
  
  \put(49,27){\makebox(0,0)[cc]{$7$}}
  
  \put(52,56.8){\makebox(0,0)[cc]{$u_1$}}
  \put(78,56.8){\makebox(0,0)[cc]{$u_2$}}
  \put(94,27.5){\makebox(0,0)[cc]{$u_3$}}
  \put(64.7,34){\makebox(0,0)[cc]{$u_4$}}
  \put(37,27.5){\makebox(0,0)[cc]{$u_5$}}

  \linethickness{0.3mm}
  \multiput(53.25,35.97)(0.13,0.19){2}{\line(0,1){0.19}}
  \multiput(53.52,36.34)(0.1,0.2){2}{\line(0,1){0.2}}
  \multiput(53.73,36.73)(0.15,0.42){1}{\line(0,1){0.42}}
  \multiput(53.88,37.15)(0.09,0.43){1}{\line(0,1){0.43}}
  \multiput(53.97,37.58)(0.03,0.44){1}{\line(0,1){0.44}}
  \multiput(53.97,38.46)(0.03,-0.44){1}{\line(0,-1){0.44}}
  \multiput(53.87,38.89)(0.1,-0.43){1}{\line(0,-1){0.43}}
  \multiput(53.71,39.3)(0.16,-0.42){1}{\line(0,-1){0.42}}
  \multiput(53.5,39.7)(0.11,-0.2){2}{\line(0,-1){0.2}}
  \multiput(53.23,40.07)(0.13,-0.18){2}{\line(0,-1){0.18}}
  \multiput(52.91,40.4)(0.11,-0.11){3}{\line(0,-1){0.11}}
  \multiput(52.54,40.7)(0.18,-0.15){2}{\line(1,0){0.18}}
  \multiput(52.14,40.96)(0.2,-0.13){2}{\line(1,0){0.2}}
  \multiput(51.7,41.17)(0.22,-0.11){2}{\line(1,0){0.22}}
  \multiput(51.23,41.33)(0.47,-0.16){1}{\line(1,0){0.47}}
  \multiput(50.75,41.44)(0.48,-0.11){1}{\line(1,0){0.48}}
  \multiput(50.25,41.49)(0.5,-0.05){1}{\line(1,0){0.5}}
  \put(49.75,41.49){\line(1,0){0.5}}
  \multiput(49.25,41.44)(0.5,0.05){1}{\line(1,0){0.5}}
  \multiput(48.77,41.33)(0.48,0.11){1}{\line(1,0){0.48}}
  \multiput(48.3,41.17)(0.47,0.16){1}{\line(1,0){0.47}}
  \multiput(47.86,40.96)(0.22,0.11){2}{\line(1,0){0.22}}
  \multiput(47.46,40.7)(0.2,0.13){2}{\line(1,0){0.2}}
  \multiput(47.09,40.4)(0.18,0.15){2}{\line(1,0){0.18}}
  \multiput(46.77,40.07)(0.11,0.11){3}{\line(0,1){0.11}}
  \multiput(46.5,39.7)(0.13,0.18){2}{\line(0,1){0.18}}
  \multiput(46.29,39.3)(0.11,0.2){2}{\line(0,1){0.2}}
  \multiput(46.13,38.89)(0.16,0.42){1}{\line(0,1){0.42}}
  \multiput(46.03,38.46)(0.1,0.43){1}{\line(0,1){0.43}}
  \multiput(46,38.02)(0.03,0.44){1}{\line(0,1){0.44}}
  \multiput(46,38.02)(0.03,-0.44){1}{\line(0,-1){0.44}}
  \multiput(46.03,37.58)(0.09,-0.43){1}{\line(0,-1){0.43}}
  \multiput(46.12,37.15)(0.15,-0.42){1}{\line(0,-1){0.42}}
  \multiput(46.27,36.73)(0.1,-0.2){2}{\line(0,-1){0.2}}
  \multiput(46.48,36.34)(0.13,-0.19){2}{\line(0,-1){0.19}}
  \put(53.25,35.97){\vector(-3,-4){0.12}}
  
  \linethickness{0.3mm}
  \multiput(83.25,35.97)(0.13,0.19){2}{\line(0,1){0.19}}
  \multiput(83.52,36.34)(0.1,0.2){2}{\line(0,1){0.2}}
  \multiput(83.73,36.73)(0.15,0.42){1}{\line(0,1){0.42}}
  \multiput(83.88,37.15)(0.09,0.43){1}{\line(0,1){0.43}}
  \multiput(83.97,37.58)(0.03,0.44){1}{\line(0,1){0.44}}
  \multiput(83.97,38.46)(0.03,-0.44){1}{\line(0,-1){0.44}}
  \multiput(83.87,38.89)(0.1,-0.43){1}{\line(0,-1){0.43}}
  \multiput(83.71,39.3)(0.16,-0.42){1}{\line(0,-1){0.42}}
  \multiput(83.5,39.7)(0.11,-0.2){2}{\line(0,-1){0.2}}
  \multiput(83.23,40.07)(0.13,-0.18){2}{\line(0,-1){0.18}}
  \multiput(82.91,40.4)(0.11,-0.11){3}{\line(0,-1){0.11}}
  \multiput(82.54,40.7)(0.18,-0.15){2}{\line(1,0){0.18}}
  \multiput(82.14,40.96)(0.2,-0.13){2}{\line(1,0){0.2}}
  \multiput(81.7,41.17)(0.22,-0.11){2}{\line(1,0){0.22}}
  \multiput(81.23,41.33)(0.47,-0.16){1}{\line(1,0){0.47}}
  \multiput(80.75,41.44)(0.48,-0.11){1}{\line(1,0){0.48}}
  \multiput(80.25,41.49)(0.5,-0.05){1}{\line(1,0){0.5}}
  \put(79.75,41.49){\line(1,0){0.5}}
  \multiput(79.25,41.44)(0.5,0.05){1}{\line(1,0){0.5}}
  \multiput(78.77,41.33)(0.48,0.11){1}{\line(1,0){0.48}}
  \multiput(78.3,41.17)(0.47,0.16){1}{\line(1,0){0.47}}
  \multiput(77.86,40.96)(0.22,0.11){2}{\line(1,0){0.22}}
  \multiput(77.46,40.7)(0.2,0.13){2}{\line(1,0){0.2}}
  \multiput(77.09,40.4)(0.18,0.15){2}{\line(1,0){0.18}}
  \multiput(76.77,40.07)(0.11,0.11){3}{\line(0,1){0.11}}
  \multiput(76.5,39.7)(0.13,0.18){2}{\line(0,1){0.18}}
  \multiput(76.29,39.3)(0.11,0.2){2}{\line(0,1){0.2}}
  \multiput(76.13,38.89)(0.16,0.42){1}{\line(0,1){0.42}}
  \multiput(76.03,38.46)(0.1,0.43){1}{\line(0,1){0.43}}
  \multiput(76,38.02)(0.03,0.44){1}{\line(0,1){0.44}}
  \multiput(76,38.02)(0.03,-0.44){1}{\line(0,-1){0.44}}
  \multiput(76.03,37.58)(0.09,-0.43){1}{\line(0,-1){0.43}}
  \multiput(76.12,37.15)(0.15,-0.42){1}{\line(0,-1){0.42}}
  \multiput(76.27,36.73)(0.1,-0.2){2}{\line(0,-1){0.2}}
  \multiput(76.48,36.34)(0.13,-0.19){2}{\line(0,-1){0.19}}
  \put(76.75,35.97){\vector(3,-4){0.12}}
  
  \linethickness{0.3mm}
  \multiput(68.51,45)(0.13,0.19){2}{\line(0,1){0.19}}
  \multiput(68.77,45.37)(0.1,0.2){2}{\line(0,1){0.2}}
  \multiput(68.98,45.77)(0.15,0.42){1}{\line(0,1){0.42}}
  \multiput(69.14,46.19)(0.09,0.43){1}{\line(0,1){0.43}}
  \multiput(69.23,46.62)(0.03,0.44){1}{\line(0,1){0.44}}
  \multiput(69.22,47.49)(0.03,-0.44){1}{\line(0,-1){0.44}}
  \multiput(69.12,47.92)(0.1,-0.43){1}{\line(0,-1){0.43}}
  \multiput(68.97,48.34)(0.16,-0.42){1}{\line(0,-1){0.42}}
  \multiput(68.75,48.73)(0.11,-0.2){2}{\line(0,-1){0.2}}
  \multiput(68.48,49.1)(0.13,-0.18){2}{\line(0,-1){0.18}}
  \multiput(68.16,49.44)(0.11,-0.11){3}{\line(0,-1){0.11}}
  \multiput(67.8,49.74)(0.18,-0.15){2}{\line(1,0){0.18}}
  \multiput(67.39,49.99)(0.2,-0.13){2}{\line(1,0){0.2}}
  \multiput(66.95,50.2)(0.22,-0.11){2}{\line(1,0){0.22}}
  \multiput(66.49,50.36)(0.47,-0.16){1}{\line(1,0){0.47}}
  \multiput(66,50.47)(0.48,-0.11){1}{\line(1,0){0.48}}
  \multiput(65.51,50.53)(0.5,-0.05){1}{\line(1,0){0.5}}
  \put(65,50.53){\line(1,0){0.5}}
  \multiput(64.51,50.47)(0.5,0.05){1}{\line(1,0){0.5}}
  \multiput(64.02,50.36)(0.48,0.11){1}{\line(1,0){0.48}}
  \multiput(63.56,50.2)(0.47,0.16){1}{\line(1,0){0.47}}
  \multiput(63.12,49.99)(0.22,0.11){2}{\line(1,0){0.22}}
  \multiput(62.71,49.74)(0.2,0.13){2}{\line(1,0){0.2}}
  \multiput(62.35,49.44)(0.18,0.15){2}{\line(1,0){0.18}}
  \multiput(62.03,49.1)(0.11,0.11){3}{\line(0,1){0.11}}
  \multiput(61.76,48.73)(0.13,0.18){2}{\line(0,1){0.18}}
  \multiput(61.54,48.34)(0.11,0.2){2}{\line(0,1){0.2}}
  \multiput(61.39,47.92)(0.16,0.42){1}{\line(0,1){0.42}}
  \multiput(61.29,47.49)(0.1,0.43){1}{\line(0,1){0.43}}
  \multiput(61.26,47.05)(0.03,0.44){1}{\line(0,1){0.44}}
  \multiput(61.26,47.05)(0.03,-0.44){1}{\line(0,-1){0.44}}
  \multiput(61.28,46.62)(0.09,-0.43){1}{\line(0,-1){0.43}}
  \multiput(61.37,46.19)(0.15,-0.42){1}{\line(0,-1){0.42}}
  \multiput(61.53,45.77)(0.1,-0.2){2}{\line(0,-1){0.2}}
  \multiput(61.74,45.37)(0.13,-0.19){2}{\line(0,-1){0.19}}
  \put(68.51,45){\vector(-3,-4){0.12}}
  
  \put(50,38){\makebox(0,0)[cc]{$k_1$}}
  
  \put(65,47){\makebox(0,0)[cc]{$k_2$}}
  
  \put(80,38){\makebox(0,0)[cc]{$k_3$}}
  
  \put(65,10){\makebox(0,0)[cc]{{\bf Figura 2.9} Un grafo, un albero
      (tr. continuo) e correnti di maglia.}}
  
\end{picture}

\vspace{2cm}

Consideriamo il grafo orientato come in {\bf Figura 2.9}. \`E stato
scelto un albero e sono indicate le correnti di maglia $k_1,k_2$ e
$k_3$ associate a maglie fondamentali corrispondenti all'albero
scelto. Le intensit\`a di corrente di lato sono espresse tramite le
correnti di maglia nel seguente modo:

\begin{equation}\label{eq:correntiDiMaglia}
\left\{
\begin{array}{l l l l l l l}
  i_1=k_1\\
  i_2=k_2\\
  i_3=k_3\\
  i_4=k_4\\
  i_5=-k_2-k_3\\
  i_6=-k_1+k_2\\
  i_7=k_1
\end{array} \right.
\end{equation}

\`E immediato verificare che le intensit\`a delle correnti espresse
attraverso la \ref{eq:correntiDiMaglia} verificano la $LKC$,
indipendentemente dai valori delle correnti di maglia. Consideriamo
tutte le maglie fondamentali a cui un dato nodo appartiene. Per
ciascuna di esse ci sono due e due soli lati che coincidono nel nodo
in esame. Di conseguenza la corrente di maglia di ciascuna di queste
maglie fondamentali compare due due sole volte nell'equazione di
Kirchhoff per le correnti applicata al nodo in esame, una volta con il
segno positivo ed una volta con il segno negativo. 
Ad esempio, consideriamo l'equazione di Kirchhoff al nodo
$\mathbf{1}$. Ad esso sono collegati i lati $1,2,6$. La corrente di
maglia $k_1$ \`e entrante nel nodo quando contribuisce all'intensit\`a
di corrente $i_1$ del lato $1$, mentre \`e uscente quando contribuisce
ad $i_6$; la corrente di maglia $k_2$ \`e entrante quando contribuisce
ad $i_6$ ed \`e uscente quando contribuisce ad $i_2$. Quindi
l'equazione di Kirchhoff per il nodo $\mathbf{1}$:

\[
-i_1+i_2-i_6=0
\]

diventa l'identit\`a

\[
k_1-k_2+(-k_1+k_2)=0
\]

%CAPITOLO 6 DEL LIBRO
\chapter{Doppi Bipoli}
\section{Generatori Controllati Lineari}
I \emph{generatori controllati lineari} sono dei particolari tipi di
doppi-bipoli a-dinamici lineari: una delle due grandezze - tensione o
intensit\`a di corrente - ad una delle due porte \`e direttamente
proporzionale al una delle due grandezze all'altra porta. Esistono
diversi tipi, qui di seguito sono descritti 4 possibili
configurazioni.

\subsection{Generatore di Tensione Controllato in Tensione}

\ifx\JPicScale\undefined\def\JPicScale{1}\fi
\unitlength \JPicScale mm
\begin{picture}(111,83)(15,15)
  \linethickness{0.3mm}
  \put(30,80){\line(1,0){30}}
  \put(30,80){\line(-1,0){0.12}}
  \put(30,80){\circle*{1.5}}
  \linethickness{0.3mm}
  \put(30,48){\line(1,0){30}}
  \put(30,48){\line(-1,0){0.12}}
  \put(30,48){\circle*{1.5}}
  \linethickness{0.3mm}
  \multiput(72,64)(0.12,0.12){50}{\line(1,0){0.12}}
  \linethickness{0.3mm}
  \multiput(72,64)(0.12,-0.12){50}{\line(1,0){0.12}}
  \linethickness{0.3mm}
  \multiput(78,58)(0.12,0.12){50}{\line(1,0){0.12}}
  \linethickness{0.3mm}
  \multiput(78,70)(0.12,-0.12){50}{\line(1,0){0.12}}
  \linethickness{0.3mm}
  \put(78,70){\line(0,1){10}}
  \linethickness{0.3mm}
  \put(78,80){\line(1,0){30}}
  \put(108,80){\line(1,0){0.12}}
  \put(108,80){\circle*{1.5}}
  \linethickness{0.3mm}
  \put(78,48){\line(1,0){30}}
  \put(108,48){\line(1,0){0.12}}
  \put(108,48){\circle*{1.5}}
  \linethickness{0.3mm}
  \put(78,48){\line(0,1){10}}
  \put(78,67){\makebox(0,0)[cc]{$+$}}

  \put(78,62){\makebox(0,0)[cc]{$-$}}

  \put(27,80){\makebox(0,0)[cc]{$+$}}

  \put(27,48){\makebox(0,0)[cc]{$-$}}

  \linethickness{0.3mm}
  \multiput(43,81)(0.12,-0.12){8}{\line(1,0){0.12}}
  \linethickness{0.3mm}
  \multiput(43,79)(0.12,0.12){8}{\line(1,0){0.12}}
  \linethickness{0.3mm}
  \multiput(91,80)(0.12,0.12){8}{\line(1,0){0.12}}
  \linethickness{0.3mm}
  \multiput(91,80)(0.12,-0.12){8}{\line(1,0){0.12}}
  \put(27,63){\makebox(0,0)[cc]{$v_1$}}

  \put(108,62){\makebox(0,0)[cc]{$v_2$}}

  \put(111,80){\makebox(0,0)[cc]{$+$}}

  \put(111,48){\makebox(0,0)[cc]{$-$}}

  \put(92,83){\makebox(0,0)[cc]{$i_2$}}

  \put(42,83){\makebox(0,0)[cc]{$i_1=0$}}

  \put(85,59){\makebox(0,0)[cc]{$\alpha v_1$}}

  \put(74,30){\makebox(0,0)[cc]{{\bf Figura 3.1} Rappresentazione
      di un Generatore di Tensione Controllato in Tensione}}

\end{picture}
\\
\\
\\
\\
\\

Le realzioni caratteristiche sono

\[
\begin{array}{l l}
  i_1=0 \\
  v_2=\alpha v_1\\
\end{array}
\]

Le relazioni caratteristiche possono essere epresse anche sotto forma
di notazione vettoriale. In questo caso abbiamo una notazione su base
ibrida.

\[
\left(
\begin{array}{c}
  i_1\\
  v_2
\end{array}
\right)=
\begin{pmatrix}
  0 & 0\\
  \alpha & 0
\end{pmatrix}
\left(
\begin{array}{c}
  v_1\\
  i_2
\end{array}
\right)
\]

\subsection{Generatore di Tensione Controllato in Corrente}
\ifx\JPicScale\undefined\def\JPicScale{1}\fi
\unitlength \JPicScale mm
\begin{picture}(111,83)(10,15)
  \linethickness{0.3mm}
  \put(30,80){\line(1,0){30}}
  \put(30,80){\line(-1,0){0.12}}
  \put(30,80){\circle*{1.5}}
  \linethickness{0.3mm}
  \put(30,48){\line(1,0){30}}
  \put(30,48){\line(-1,0){0.12}}
  \put(30,48){\circle*{1.5}}
  \linethickness{0.3mm}
  \multiput(72,64)(0.12,0.12){50}{\line(1,0){0.12}}
  \linethickness{0.3mm}
  \multiput(72,64)(0.12,-0.12){50}{\line(1,0){0.12}}
  \linethickness{0.3mm}
  \multiput(78,58)(0.12,0.12){50}{\line(1,0){0.12}}
  \linethickness{0.3mm}
  \multiput(78,70)(0.12,-0.12){50}{\line(1,0){0.12}}
  \linethickness{0.3mm}
  \put(78,70){\line(0,1){10}}
  \linethickness{0.3mm}
  \put(78,80){\line(1,0){30}}
  \put(108,80){\line(1,0){0.12}}
  \put(108,80){\circle*{1.5}}
  \linethickness{0.3mm}
  \put(78,48){\line(1,0){30}}
  \put(108,48){\line(1,0){0.12}}
  \put(108,48){\circle*{1.5}}
  \linethickness{0.3mm}
  \put(78,48){\line(0,1){10}}
  \put(78,67){\makebox(0,0)[cc]{$+$}}

  \put(78,62){\makebox(0,0)[cc]{$-$}}

  \put(27,80){\makebox(0,0)[cc]{$+$}}

  \put(27,48){\makebox(0,0)[cc]{$-$}}

  \linethickness{0.3mm}
  \multiput(43,81)(0.12,-0.12){8}{\line(1,0){0.12}}
  \linethickness{0.3mm}
  \multiput(43,79)(0.12,0.12){8}{\line(1,0){0.12}}
  \linethickness{0.3mm}
  \multiput(91,80)(0.12,0.12){8}{\line(1,0){0.12}}
  \linethickness{0.3mm}
  \multiput(91,80)(0.12,-0.12){8}{\line(1,0){0.12}}
  \put(27,63){\makebox(0,0)[cc]{$v_1=0$}}

  \put(108,62){\makebox(0,0)[cc]{$v_2$}}

  \put(111,80){\makebox(0,0)[cc]{$+$}}

  \put(111,48){\makebox(0,0)[cc]{$-$}}

  \put(92,83){\makebox(0,0)[cc]{$i_2$}}

  \put(42,83){\makebox(0,0)[cc]{$i_1$}}

  \put(83,59){\makebox(0,0)[cc]{$r i_1$}}

  \linethickness{0.3mm}
  \put(60,48){\line(0,1){32}}

  \put(74,30){\makebox(0,0)[cc]{{\bf Figura 3.2} Rappresentazione
      di un Generatore di Tensione Controllato in Corrente}}
\end{picture}

Le realzioni caratteristiche sono

\[
\begin{array}{l l}
  v_1=0\\
  v_2=\gamma v_1
\end{array}
\]

Le relazioni caratteristiche possono essere epresse anche sotto forma
di notazione vettoriale. In questo caso abbiamo una notazione su base
tensione.

\[
\left(
\begin{array}{c}
  v_1\\
  v_2
\end{array}
\right)=
\begin{pmatrix}
  0 & 0\\
  \gamma & 0
\end{pmatrix}
\left(
\begin{array}{c}
  i_1\\
  i_2
\end{array}
\right)
\]

\subsection{Generatore di Corrente Controllato in Tensione}
\ifx\JPicScale\undefined\def\JPicScale{1}\fi
\unitlength \JPicScale mm
\begin{picture}(112,83)(15,15)
  \linethickness{0.3mm}
  \put(30,80){\line(1,0){30}}
  \put(30,80){\line(-1,0){0.12}}
  \put(30,80){\circle*{1.5}}
  \linethickness{0.3mm}
  \put(30,48){\line(1,0){30}}
  \put(30,48){\line(-1,0){0.12}}
  \put(30,48){\circle*{1.5}}
  \linethickness{0.3mm}
  \multiput(72,64)(0.12,0.12){50}{\line(1,0){0.12}}
  \linethickness{0.3mm}
  \multiput(72,64)(0.12,-0.12){50}{\line(1,0){0.12}}
  \linethickness{0.3mm}
  \multiput(78,58)(0.12,0.12){50}{\line(1,0){0.12}}
  \linethickness{0.3mm}
  \multiput(78,70)(0.12,-0.12){50}{\line(1,0){0.12}}
  \linethickness{0.3mm}
  \put(78,70){\line(0,1){10}}
  \linethickness{0.3mm}
  \put(78,80){\line(1,0){30}}
  \put(108,80){\line(1,0){0.12}}
  \put(108,80){\circle*{1.5}}
  \linethickness{0.3mm}
  \put(78,48){\line(1,0){30}}
  \put(108,48){\line(1,0){0.12}}
  \put(108,48){\circle*{1.5}}
  \linethickness{0.3mm}
  \put(78,48){\line(0,1){10}}
  \put(27,80){\makebox(0,0)[cc]{$+$}}

  \put(27,48){\makebox(0,0)[cc]{$-$}}

  \linethickness{0.3mm}
  \multiput(43,81)(0.12,-0.12){8}{\line(1,0){0.12}}
  \linethickness{0.3mm}
  \multiput(43,79)(0.12,0.12){8}{\line(1,0){0.12}}
  \linethickness{0.3mm}
  \multiput(91,80)(0.12,0.12){8}{\line(1,0){0.12}}
  \linethickness{0.3mm}
  \multiput(91,80)(0.12,-0.12){8}{\line(1,0){0.12}}
  \put(27,63){\makebox(0,0)[cc]{$v_1$}}

  \put(108,62){\makebox(0,0)[cc]{$v_2$}}

  \put(111,80){\makebox(0,0)[cc]{$+$}}

  \put(112,48){\makebox(0,0)[cc]{$-$}}

  \put(92,83){\makebox(0,0)[cc]{$i_2$}}

  \put(42,83){\makebox(0,0)[cc]{$i_1=0$}}

  \put(84,59){\makebox(0,0)[cc]{$g v_1$}}

  \linethickness{0.3mm}
  \put(78,61){\line(0,1){6}}
  \put(78,61){\vector(0,-1){0.12}}

  \put(74,30){\makebox(0,0)[cc]{{\bf Figura 3.3} Rappresentazione
      di un Generatore di Corrente Controllato in Tensione}}
\end{picture}

Le relazioni caratteristiche sono
\[
\begin{array}{l l}
  i_1=0\\
  i_2=gv_1
\end{array}
\]

Le relazioni caratteristiche possono essere epresse anche sotto forma
di notazione vettoriale. In questo caso abbiamo una notazione su base
corrente.

\[
\left(
\begin{array}{c}
  i_1\\
  i_2
\end{array}
\right)=
\begin{pmatrix}
  0 & 0\\
  g & 0
\end{pmatrix}
\left(
\begin{array}{c}
  v_1\\
  v_2
\end{array}
\right)
\]

\subsection{Generatore di Corrente Controllato in Corrente}

\ifx\JPicScale\undefined\def\JPicScale{1}\fi
\unitlength \JPicScale mm
\begin{picture}(112,83)(15,15)
  \linethickness{0.3mm}
  \put(30,80){\line(1,0){30}}
  \put(30,80){\line(-1,0){0.12}}
  \put(30,80){\circle*{1.5}}
  \linethickness{0.3mm}
  \put(30,48){\line(1,0){30}}
  \put(30,48){\line(-1,0){0.12}}
  \put(30,48){\circle*{1.5}}
  \linethickness{0.3mm}
  \multiput(72,64)(0.12,0.12){50}{\line(1,0){0.12}}
  \linethickness{0.3mm}
  \multiput(72,64)(0.12,-0.12){50}{\line(1,0){0.12}}
  \linethickness{0.3mm}
  \multiput(78,58)(0.12,0.12){50}{\line(1,0){0.12}}
  \linethickness{0.3mm}
  \multiput(78,70)(0.12,-0.12){50}{\line(1,0){0.12}}
  \linethickness{0.3mm}
  \put(78,70){\line(0,1){10}}
  \linethickness{0.3mm}
  \put(78,80){\line(1,0){30}}
  \put(108,80){\line(1,0){0.12}}
  \put(108,80){\circle*{1.5}}
  \linethickness{0.3mm}
  \put(78,48){\line(1,0){30}}
  \put(108,48){\line(1,0){0.12}}
  \put(108,48){\circle*{1.5}}
  \linethickness{0.3mm}
  \put(78,48){\line(0,1){10}}
  \put(27,80){\makebox(0,0)[cc]{$+$}}
  
  \put(27,48){\makebox(0,0)[cc]{$-$}}
  
  \linethickness{0.3mm}
  \multiput(43,81)(0.12,-0.12){8}{\line(1,0){0.12}}
  \linethickness{0.3mm}
  \multiput(43,79)(0.12,0.12){8}{\line(1,0){0.12}}
  \linethickness{0.3mm}
  \multiput(91,80)(0.12,0.12){8}{\line(1,0){0.12}}
  \linethickness{0.3mm}
  \multiput(91,80)(0.12,-0.12){8}{\line(1,0){0.12}}
  \put(27,63){\makebox(0,0)[cc]{$v_1=0$}}
  
  \put(108,62){\makebox(0,0)[cc]{$v_2$}}
  
  \put(111,80){\makebox(0,0)[cc]{$+$}}
  
  \put(112,48){\makebox(0,0)[cc]{$-$}}
  
  \put(92,83){\makebox(0,0)[cc]{$i_2$}}
  
  \put(42,83){\makebox(0,0)[cc]{$i_1$}}
  
  \put(84,59){\makebox(0,0)[cc]{$\beta v_1$}}
  
  \linethickness{0.3mm}
  \put(78,61){\line(0,1){6}}
  \put(78,61){\vector(0,-1){0.12}}
  \linethickness{0.3mm}
  \put(60,48){\line(0,1){32}}
  
  \put(74,30){\makebox(0,0)[cc]{{\bf Figura 3.4} Rappresentazione
      di un Generatore di Corrente Controllato in Corrente}}
  
\end{picture}

Le realzioni caratteristiche sono

\[
\begin{array}{l l}
  v_1=0\\
  i_2=\beta i_1
\end{array}
\]

Le relazioni caratteristiche possono essere epresse anche sotto forma
di notazione vettoriale. In questo caso abbiamo una notazione su base
ibrida.

\[
\left(
\begin{array}{c}
  v_1\\
  i_2
\end{array}
\right)=
\begin{pmatrix}
  0 & 0\\
  \beta & 0
\end{pmatrix}
\left(
\begin{array}{c}
  i_1\\
  v_2
\end{array}
\right)
\]

\section{Trasformatore}
Il trasformatore pu\`o essere realizzato con due circuiti mutuamente
accoppiati. I due circuiti, \emph{primario} e \emph{secondario}, sono
realizzati avvolgendo del filo conduttore, smaltato con della vernice
isolante,su un supporto di materiale, fatto di ferrite o lamine
sottili di acciaio speciale.

\ifx\JPicScale\undefined\def\JPicScale{1}\fi
\unitlength \JPicScale mm
\begin{picture}(103,52)(35,15)
  \linethickness{0.3mm}
  \multiput(88,46)(0.48,-0.06){1}{\line(1,0){0.48}}
  \multiput(88.48,45.94)(0.45,-0.17){1}{\line(1,0){0.45}}
  \multiput(88.93,45.77)(0.2,-0.14){2}{\line(1,0){0.2}}
  \multiput(89.33,45.5)(0.11,-0.12){3}{\line(0,-1){0.12}}
  \multiput(89.65,45.14)(0.11,-0.21){2}{\line(0,-1){0.21}}
  \multiput(89.87,44.71)(0.12,-0.47){1}{\line(0,-1){0.47}}
  \put(89.99,43.76){\line(0,1){0.48}}
  \multiput(89.87,43.29)(0.12,0.47){1}{\line(0,1){0.47}}
  \multiput(89.65,42.86)(0.11,0.21){2}{\line(0,1){0.21}}
  \multiput(89.33,42.5)(0.11,0.12){3}{\line(0,1){0.12}}
  \multiput(88.93,42.23)(0.2,0.14){2}{\line(1,0){0.2}}
  \multiput(88.48,42.06)(0.45,0.17){1}{\line(1,0){0.45}}
  \multiput(88,42)(0.48,0.06){1}{\line(1,0){0.48}}

  \linethickness{0.3mm}
  \multiput(88,42)(0.48,-0.06){1}{\line(1,0){0.48}}
  \multiput(88.48,41.94)(0.45,-0.17){1}{\line(1,0){0.45}}
  \multiput(88.93,41.77)(0.2,-0.14){2}{\line(1,0){0.2}}
  \multiput(89.33,41.5)(0.11,-0.12){3}{\line(0,-1){0.12}}
  \multiput(89.65,41.14)(0.11,-0.21){2}{\line(0,-1){0.21}}
  \multiput(89.87,40.71)(0.12,-0.47){1}{\line(0,-1){0.47}}
  \put(89.99,39.76){\line(0,1){0.48}}
  \multiput(89.87,39.29)(0.12,0.47){1}{\line(0,1){0.47}}
  \multiput(89.65,38.86)(0.11,0.21){2}{\line(0,1){0.21}}
  \multiput(89.33,38.5)(0.11,0.12){3}{\line(0,1){0.12}}
  \multiput(88.93,38.23)(0.2,0.14){2}{\line(1,0){0.2}}
  \multiput(88.48,38.06)(0.45,0.17){1}{\line(1,0){0.45}}
  \multiput(88,38)(0.48,0.06){1}{\line(1,0){0.48}}

  \linethickness{0.3mm}
  \multiput(88,34)(0.48,-0.06){1}{\line(1,0){0.48}}
  \multiput(88.48,33.94)(0.45,-0.17){1}{\line(1,0){0.45}}
  \multiput(88.93,33.77)(0.2,-0.14){2}{\line(1,0){0.2}}
  \multiput(89.33,33.5)(0.11,-0.12){3}{\line(0,-1){0.12}}
  \multiput(89.65,33.14)(0.11,-0.21){2}{\line(0,-1){0.21}}
  \multiput(89.87,32.71)(0.12,-0.47){1}{\line(0,-1){0.47}}
  \put(89.99,31.76){\line(0,1){0.48}}
  \multiput(89.87,31.29)(0.12,0.47){1}{\line(0,1){0.47}}
  \multiput(89.65,30.86)(0.11,0.21){2}{\line(0,1){0.21}}
  \multiput(89.33,30.5)(0.11,0.12){3}{\line(0,1){0.12}}
  \multiput(88.93,30.23)(0.2,0.14){2}{\line(1,0){0.2}}
  \multiput(88.48,30.06)(0.45,0.17){1}{\line(1,0){0.45}}
  \multiput(88,30)(0.48,0.06){1}{\line(1,0){0.48}}

  \linethickness{0.3mm}
  \multiput(88,38)(0.48,-0.06){1}{\line(1,0){0.48}}
  \multiput(88.48,37.94)(0.45,-0.17){1}{\line(1,0){0.45}}
  \multiput(88.93,37.77)(0.2,-0.14){2}{\line(1,0){0.2}}
  \multiput(89.33,37.5)(0.11,-0.12){3}{\line(0,-1){0.12}}
  \multiput(89.65,37.14)(0.11,-0.21){2}{\line(0,-1){0.21}}
  \multiput(89.87,36.71)(0.12,-0.47){1}{\line(0,-1){0.47}}
  \put(89.99,35.76){\line(0,1){0.48}}
  \multiput(89.87,35.29)(0.12,0.47){1}{\line(0,1){0.47}}
  \multiput(89.65,34.86)(0.11,0.21){2}{\line(0,1){0.21}}
  \multiput(89.33,34.5)(0.11,0.12){3}{\line(0,1){0.12}}
  \multiput(88.93,34.23)(0.2,0.14){2}{\line(1,0){0.2}}
  \multiput(88.48,34.06)(0.45,0.17){1}{\line(1,0){0.45}}
  \multiput(88,34)(0.48,0.06){1}{\line(1,0){0.48}}

  \linethickness{0.3mm}
  \multiput(93.52,41.94)(0.48,0.06){1}{\line(1,0){0.48}}
  \multiput(93.07,41.77)(0.45,0.17){1}{\line(1,0){0.45}}
  \multiput(92.67,41.5)(0.2,0.14){2}{\line(1,0){0.2}}
  \multiput(92.35,41.14)(0.11,0.12){3}{\line(0,1){0.12}}
  \multiput(92.13,40.71)(0.11,0.21){2}{\line(0,1){0.21}}
  \multiput(92.01,40.24)(0.12,0.47){1}{\line(0,1){0.47}}
  \put(92.01,39.76){\line(0,1){0.48}}
  \multiput(92.01,39.76)(0.12,-0.47){1}{\line(0,-1){0.47}}
  \multiput(92.13,39.29)(0.11,-0.21){2}{\line(0,-1){0.21}}
  \multiput(92.35,38.86)(0.11,-0.12){3}{\line(0,-1){0.12}}
  \multiput(92.67,38.5)(0.2,-0.14){2}{\line(1,0){0.2}}
  \multiput(93.07,38.23)(0.45,-0.17){1}{\line(1,0){0.45}}
  \multiput(93.52,38.06)(0.48,-0.06){1}{\line(1,0){0.48}}

  \linethickness{0.3mm}
  \multiput(93.52,45.94)(0.48,0.06){1}{\line(1,0){0.48}}
  \multiput(93.07,45.77)(0.45,0.17){1}{\line(1,0){0.45}}
  \multiput(92.67,45.5)(0.2,0.14){2}{\line(1,0){0.2}}
  \multiput(92.35,45.14)(0.11,0.12){3}{\line(0,1){0.12}}
  \multiput(92.13,44.71)(0.11,0.21){2}{\line(0,1){0.21}}
  \multiput(92.01,44.24)(0.12,0.47){1}{\line(0,1){0.47}}
  \put(92.01,43.76){\line(0,1){0.48}}
  \multiput(92.01,43.76)(0.12,-0.47){1}{\line(0,-1){0.47}}
  \multiput(92.13,43.29)(0.11,-0.21){2}{\line(0,-1){0.21}}
  \multiput(92.35,42.86)(0.11,-0.12){3}{\line(0,-1){0.12}}
  \multiput(92.67,42.5)(0.2,-0.14){2}{\line(1,0){0.2}}
  \multiput(93.07,42.23)(0.45,-0.17){1}{\line(1,0){0.45}}
  \multiput(93.52,42.06)(0.48,-0.06){1}{\line(1,0){0.48}}

  \linethickness{0.3mm}
  \multiput(93.52,37.94)(0.48,0.06){1}{\line(1,0){0.48}}
  \multiput(93.07,37.77)(0.45,0.17){1}{\line(1,0){0.45}}
  \multiput(92.67,37.5)(0.2,0.14){2}{\line(1,0){0.2}}
  \multiput(92.35,37.14)(0.11,0.12){3}{\line(0,1){0.12}}
  \multiput(92.13,36.71)(0.11,0.21){2}{\line(0,1){0.21}}
  \multiput(92.01,36.24)(0.12,0.47){1}{\line(0,1){0.47}}
  \put(92.01,35.76){\line(0,1){0.48}}
  \multiput(92.01,35.76)(0.12,-0.47){1}{\line(0,-1){0.47}}
  \multiput(92.13,35.29)(0.11,-0.21){2}{\line(0,-1){0.21}}
  \multiput(92.35,34.86)(0.11,-0.12){3}{\line(0,-1){0.12}}
  \multiput(92.67,34.5)(0.2,-0.14){2}{\line(1,0){0.2}}
  \multiput(93.07,34.23)(0.45,-0.17){1}{\line(1,0){0.45}}
  \multiput(93.52,34.06)(0.48,-0.06){1}{\line(1,0){0.48}}

  \linethickness{0.3mm}
  \multiput(93.52,33.94)(0.48,0.06){1}{\line(1,0){0.48}}
  \multiput(93.07,33.77)(0.45,0.17){1}{\line(1,0){0.45}}
  \multiput(92.67,33.5)(0.2,0.14){2}{\line(1,0){0.2}}
  \multiput(92.35,33.14)(0.11,0.12){3}{\line(0,1){0.12}}
  \multiput(92.13,32.71)(0.11,0.21){2}{\line(0,1){0.21}}
  \multiput(92.01,32.24)(0.12,0.47){1}{\line(0,1){0.47}}
  \put(92.01,31.76){\line(0,1){0.48}}
  \multiput(92.01,31.76)(0.12,-0.47){1}{\line(0,-1){0.47}}
  \multiput(92.13,31.29)(0.11,-0.21){2}{\line(0,-1){0.21}}
  \multiput(92.35,30.86)(0.11,-0.12){3}{\line(0,-1){0.12}}
  \multiput(92.67,30.5)(0.2,-0.14){2}{\line(1,0){0.2}}
  \multiput(93.07,30.23)(0.45,-0.17){1}{\line(1,0){0.45}}
  \multiput(93.52,30.06)(0.48,-0.06){1}{\line(1,0){0.48}}

  \linethickness{0.3mm}
  \put(94,46){\line(1,0){20}}
  \put(114,46){\line(1,0){0.12}}
  \put(114,46){\circle*{1.5}}
  \linethickness{0.3mm}
  \put(68,30){\line(1,0){20}}
  \put(88,30){\line(1,0){0.12}}
  \put(68,30){\circle*{1.5}}
  \linethickness{0.3mm}
  \put(68,46){\line(1,0){20}}
  \put(88,46){\line(1,0){0.12}}
  \put(68,46){\circle*{1.5}}
  \linethickness{0.3mm}
  \put(94,30){\line(1,0){20}}
  \put(114,30){\line(1,0){0.12}}
  \put(114,30){\circle*{1.5}}
  \linethickness{0.3mm}
  \put(78,24){\line(0,1){28}}
  \linethickness{0.3mm}
  \put(78,24){\line(1,0){26}}
  \linethickness{0.3mm}
  \put(104,24){\line(0,1){28}}
  \linethickness{0.3mm}
  \put(78,52){\line(1,0){26}}
  \put(86,49){\makebox(0,0)[cc]{$L_1$$L_2$$M$}}

  \put(90.23,15.65){\makebox(0,0)[cc]{{\bf Figura 3.5} Rappresentazione simbolica di un trasformatore}}
\end{picture}

\vspace{1cm}

\subsection{Mutuo Accoppiamento}

Le equazioni caratteristiche del \emph{mutuo accoppiamento} sono

\[
\left\{
\begin{array}{l l}
  v_1=L_1\dfrac{d\;i_1}{dt}+M\dfrac{d\;i_2}{dt} \\
  v_2=L_2\dfrac{d\;i_2}{dt}+M\dfrac{d\;i_1}{dt} \\
\end{array} \right.
\]

\vspace{0.5cm}

\fbox{{\bf \emph {N.B.}}\hspace{10mm}
  \begin{minipage}[c]{94mm}
    \emph{In caso di mutuo accoppiamento perfetto abbiamo}
    \[
    k=\dfrac{M}{\sqrt{L_1L_2}}=1 
    \]
    \[
    M^2=L_1L_2\implies \dfrac{M}{L_1}=\dfrac{L_2}{M}
    \]
\end{minipage}}

\vspace{0.5cm}
\[
\left\{
\begin{array}{l l}
  v_1=L_1\left(\dfrac{d\;i_1}{dt}+\dfrac{M}{L_1}\cdot \dfrac{d\;i_2}{dt}\right) \\
  v_2=M\left(\dfrac{d\;i_1}{dt}+\dfrac{L_2}{M}\cdot \dfrac{d\;i_2}{dt}\right) \\
\end{array} \right.
\]

\vspace{0.5cm}
Facendo il rapporto membro a membro abbiamo
\[
\dfrac{v_1}{v_2}=\dfrac{L_1}{M}\implies n \quad \text{$n$: Rapporto di Trasformazione}
\]

Simbolicamente il trasformatore ideale si pu\`o rappresentare in
questo modo


\ifx\JPicScale\undefined\def\JPicScale{1}\fi
\unitlength \JPicScale mm
\begin{picture}(103,75)(15,15)
  \linethickness{0.3mm}
  \put(70,62){\line(0,1){10}}
  \linethickness{0.3mm}
  \put(70,48){\line(0,1){10}}
  \linethickness{0.3mm}
  \put(72,62){\line(0,1){10}}
  \linethickness{0.3mm}
  \put(72,48){\line(0,1){10}}
  \linethickness{0.3mm}
  \multiput(70,62)(0.48,-0.06){1}{\line(1,0){0.48}}
  \multiput(70.48,61.94)(0.45,-0.17){1}{\line(1,0){0.45}}
  \multiput(70.93,61.77)(0.2,-0.14){2}{\line(1,0){0.2}}
  \multiput(71.33,61.5)(0.11,-0.12){3}{\line(0,-1){0.12}}
  \multiput(71.65,61.14)(0.11,-0.21){2}{\line(0,-1){0.21}}
  \multiput(71.87,60.71)(0.12,-0.47){1}{\line(0,-1){0.47}}
  \put(71.99,59.76){\line(0,1){0.48}}
  \multiput(71.87,59.29)(0.12,0.47){1}{\line(0,1){0.47}}
  \multiput(71.65,58.86)(0.11,0.21){2}{\line(0,1){0.21}}
  \multiput(71.33,58.5)(0.11,0.12){3}{\line(0,1){0.12}}
  \multiput(70.93,58.23)(0.2,0.14){2}{\line(1,0){0.2}}
  \multiput(70.48,58.06)(0.45,0.17){1}{\line(1,0){0.45}}
  \multiput(70,58)(0.48,0.06){1}{\line(1,0){0.48}}

  \linethickness{0.3mm}
  \multiput(71.52,61.94)(0.48,0.06){1}{\line(1,0){0.48}}
  \multiput(71.07,61.77)(0.45,0.17){1}{\line(1,0){0.45}}
  \multiput(70.67,61.5)(0.2,0.14){2}{\line(1,0){0.2}}
  \multiput(70.35,61.14)(0.11,0.12){3}{\line(0,1){0.12}}
  \multiput(70.13,60.71)(0.11,0.21){2}{\line(0,1){0.21}}
  \multiput(70.01,60.24)(0.12,0.47){1}{\line(0,1){0.47}}
  \put(70.01,59.76){\line(0,1){0.48}}
  \multiput(70.01,59.76)(0.12,-0.47){1}{\line(0,-1){0.47}}
  \multiput(70.13,59.29)(0.11,-0.21){2}{\line(0,-1){0.21}}
  \multiput(70.35,58.86)(0.11,-0.12){3}{\line(0,-1){0.12}}
  \multiput(70.67,58.5)(0.2,-0.14){2}{\line(1,0){0.2}}
  \multiput(71.07,58.23)(0.45,-0.17){1}{\line(1,0){0.45}}
  \multiput(71.52,58.06)(0.48,-0.06){1}{\line(1,0){0.48}}

  \linethickness{0.3mm}
  \put(72,72){\line(1,0){29}}
  \put(101,72){\line(1,0){0.12}}
  \put(101,72){\circle*{1.5}}
  \linethickness{0.3mm}
  \put(72,48){\line(1,0){29}}
  \put(101,48){\line(1,0){0.12}}
  \put(101,48){\circle*{1.5}}
  \linethickness{0.3mm}
  \put(42,72){\line(1,0){28}}
  \put(42,72){\line(-1,0){0.12}}
  \put(42,72){\circle*{1.5}}
  \linethickness{0.3mm}
  \put(42,48){\line(1,0){28}}
  \put(42,48){\line(-1,0){0.12}}
  \put(42,48){\circle*{1.5}}
  \put(71,75){\makebox(0,0)[cc]{n:1}}

  \put(73,75){\makebox(0,0)[cc]{}}

  \linethickness{0.3mm}
  \multiput(55,73)(0.12,-0.12){8}{\line(1,0){0.12}}
  \linethickness{0.3mm}
  \multiput(55,71)(0.12,0.12){8}{\line(1,0){0.12}}
  \linethickness{0.3mm}
  \multiput(86,72)(0.12,0.12){8}{\line(1,0){0.12}}
  \linethickness{0.3mm}
  \multiput(86,72)(0.12,-0.12){8}{\line(1,0){0.12}}
  \put(39,72){\makebox(0,0)[cc]{+}}

  \put(104,72){\makebox(0,0)[cc]{$+$}}

  \put(104,48){\makebox(0,0)[cc]{$-$}}

  \put(39,48){\makebox(0,0)[cc]{$-$}}

  \put(40,59){\makebox(0,0)[cc]{$v_1$}}

  \put(103,59){\makebox(0,0)[cc]{$v_2$}}

  \put(56,75){\makebox(0,0)[cc]{$i_1$}}

  \put(87,75){\makebox(0,0)[cc]{$i_2$}}

  \put(70.23,30.65){\makebox(0,0)[cc]{{\bf Figura 3.6} Trasformatore
      ideale con doppio bipolo generico sul primario}}
\end{picture}

Continuando ad agire con delle semplici manipolazioni algebriche
abbiamo le seguenti relazioni.
\[
v_1 = L_1\left(\dfrac{d\;i_1}{dt}+\dfrac{M}{L_1}\cdot
\dfrac{d\;i_2}{dt}\right) = L_1 \dfrac{d}{dt}
\underbrace{\left(i_1+\dfrac{1}{n}\;i_2\right)}_\text{$i^*$} = L_1\dfrac{d\;i^*}{dt}
\]

Aggiungiamo al circuito un bipolo generico

\ifx\JPicScale\undefined\def\JPicScale{1}\fi
\unitlength \JPicScale mm
\begin{picture}(103,75)(15,15)
  \linethickness{0.3mm}
  \put(70,62){\line(0,1){10}}
  \linethickness{0.3mm}
  \put(70,48){\line(0,1){10}}
  \linethickness{0.3mm}
  \put(72,62){\line(0,1){10}}
  \linethickness{0.3mm}
  \put(72,48){\line(0,1){10}}
  \linethickness{0.3mm}
  \multiput(70,62)(0.48,-0.06){1}{\line(1,0){0.48}}
  \multiput(70.48,61.94)(0.45,-0.17){1}{\line(1,0){0.45}}
  \multiput(70.93,61.77)(0.2,-0.14){2}{\line(1,0){0.2}}
  \multiput(71.33,61.5)(0.11,-0.12){3}{\line(0,-1){0.12}}
  \multiput(71.65,61.14)(0.11,-0.21){2}{\line(0,-1){0.21}}
  \multiput(71.87,60.71)(0.12,-0.47){1}{\line(0,-1){0.47}}
  \put(71.99,59.76){\line(0,1){0.48}}
  \multiput(71.87,59.29)(0.12,0.47){1}{\line(0,1){0.47}}
  \multiput(71.65,58.86)(0.11,0.21){2}{\line(0,1){0.21}}
  \multiput(71.33,58.5)(0.11,0.12){3}{\line(0,1){0.12}}
  \multiput(70.93,58.23)(0.2,0.14){2}{\line(1,0){0.2}}
  \multiput(70.48,58.06)(0.45,0.17){1}{\line(1,0){0.45}}
  \multiput(70,58)(0.48,0.06){1}{\line(1,0){0.48}}

  \linethickness{0.3mm}
  \multiput(71.52,61.94)(0.48,0.06){1}{\line(1,0){0.48}}
  \multiput(71.07,61.77)(0.45,0.17){1}{\line(1,0){0.45}}
  \multiput(70.67,61.5)(0.2,0.14){2}{\line(1,0){0.2}}
  \multiput(70.35,61.14)(0.11,0.12){3}{\line(0,1){0.12}}
  \multiput(70.13,60.71)(0.11,0.21){2}{\line(0,1){0.21}}
  \multiput(70.01,60.24)(0.12,0.47){1}{\line(0,1){0.47}}
  \put(70.01,59.76){\line(0,1){0.48}}
  \multiput(70.01,59.76)(0.12,-0.47){1}{\line(0,-1){0.47}}
  \multiput(70.13,59.29)(0.11,-0.21){2}{\line(0,-1){0.21}}
  \multiput(70.35,58.86)(0.11,-0.12){3}{\line(0,-1){0.12}}
  \multiput(70.67,58.5)(0.2,-0.14){2}{\line(1,0){0.2}}
  \multiput(71.07,58.23)(0.45,-0.17){1}{\line(1,0){0.45}}
  \multiput(71.52,58.06)(0.48,-0.06){1}{\line(1,0){0.48}}

  \linethickness{0.3mm}
  \put(72,72){\line(1,0){29}}
  \put(101,72){\line(1,0){0.12}}
  \put(101,72){\circle*{1.5}}
  \linethickness{0.3mm}
  \put(72,48){\line(1,0){29}}
  \put(101,48){\line(1,0){0.12}}
  \put(101,48){\circle*{1.5}}
  \linethickness{0.3mm}
  \put(42,72){\line(1,0){28}}
  \put(42,72){\line(-1,0){0.12}}
  \put(42,72){\circle*{1.5}}
  \put(51,72){\circle*{1.5}}
  \linethickness{0.3mm}
  \put(42,48){\line(1,0){28}}
  \put(42,48){\line(-1,0){0.12}}
  \put(42,48){\circle*{1.5}}
  \put(55.98,48){\circle*{1.5}}
  \put(71,75){\makebox(0,0)[cc]{n:1}}

  \put(73,75){\makebox(0,0)[cc]{}}

  \linethickness{0.3mm}
  \multiput(55,73)(0.12,-0.12){8}{\line(1,0){0.12}}
  \linethickness{0.3mm}
  \multiput(55,71)(0.12,0.12){8}{\line(1,0){0.12}}
  \linethickness{0.3mm}
  \multiput(86,72)(0.12,0.12){8}{\line(1,0){0.12}}
  \linethickness{0.3mm}
  \multiput(86,72)(0.12,-0.12){8}{\line(1,0){0.12}}
  \put(39,72){\makebox(0,0)[cc]{+}}

  \put(104,72){\makebox(0,0)[cc]{$+$}}

  \put(104,48){\makebox(0,0)[cc]{$-$}}

  \put(39,48){\makebox(0,0)[cc]{$-$}}

  \put(40,59){\makebox(0,0)[cc]{$v_1$}}

  \put(103,59){\makebox(0,0)[cc]{$v_2$}}

  \linethickness{0.3mm}
  \put(51,65){\line(0,1){7}}
  \linethickness{0.3mm}
  \put(48,65){\line(1,0){6}}
  \linethickness{0.3mm}
  \put(54,55){\line(0,1){10}}
  \linethickness{0.3mm}
  \put(48,55){\line(1,0){6}}
  \linethickness{0.3mm}
  \linethickness{0.3mm}
  \put(48,55){\line(0,1){10}}
  \linethickness{0.3mm}
  \put(51,52){\line(0,1){3}}
  \linethickness{0.3mm}
  \put(51,52){\line(1,0){5}}
  \linethickness{0.3mm}
  \put(56,48){\line(0,1){4}}
  \linethickness{0.3mm}
  \multiput(47,73)(0.12,-0.12){8}{\line(1,0){0.12}}
  \linethickness{0.3mm}
  \multiput(47,71)(0.12,0.12){8}{\line(1,0){0.12}}
  \linethickness{0.3mm}
  \multiput(50,68)(0.12,-0.12){8}{\line(1,0){0.12}}
  \linethickness{0.3mm}
  \multiput(51,67)(0.12,0.12){8}{\line(1,0){0.12}}
  \put(55,75){\makebox(0,0)[cc]{$i'$}}

  \put(46,75){\makebox(0,0)[cc]{$i_1$}}

  \put(54,67){\makebox(0,0)[cc]{$i^*$}}

  \put(88,75){\makebox(0,0)[cc]{$i_2$}}

  \put(70.23,30.65){\makebox(0,0)[cc]{{\bf Figura 3.7} Trasformatore
      ideale con doppio bipolo generico sul primario}}
\end{picture}

\[
i'=-\dfrac{1}{n}\; i_2 \quad \text{in quanto} \quad i_1=i^*+i'
\]

\[
i^*=i_1-i'=i_1+\dfrac{1}{n}\; i_2
\]

\subsection{Trasformatore Reale}

\ifx\JPicScale\undefined\def\JPicScale{1}\fi
\unitlength \JPicScale mm
\begin{picture}(103,76)(0,15)
  \linethickness{0.3mm}
  \put(70,62){\line(0,1){10}}
  \linethickness{0.3mm}
  \put(70,48){\line(0,1){10}}
  \linethickness{0.3mm}
  \put(72,62){\line(0,1){10}}
  \linethickness{0.3mm}
  \put(72,48){\line(0,1){10}}
  \linethickness{0.3mm}
  \multiput(70,62)(0.48,-0.06){1}{\line(1,0){0.48}}
  \multiput(70.48,61.94)(0.45,-0.17){1}{\line(1,0){0.45}}
  \multiput(70.93,61.77)(0.2,-0.14){2}{\line(1,0){0.2}}
  \multiput(71.33,61.5)(0.11,-0.12){3}{\line(0,-1){0.12}}
  \multiput(71.65,61.14)(0.11,-0.21){2}{\line(0,-1){0.21}}
  \multiput(71.87,60.71)(0.12,-0.47){1}{\line(0,-1){0.47}}
  \put(71.99,59.76){\line(0,1){0.48}}
  \multiput(71.87,59.29)(0.12,0.47){1}{\line(0,1){0.47}}
  \multiput(71.65,58.86)(0.11,0.21){2}{\line(0,1){0.21}}
  \multiput(71.33,58.5)(0.11,0.12){3}{\line(0,1){0.12}}
  \multiput(70.93,58.23)(0.2,0.14){2}{\line(1,0){0.2}}
  \multiput(70.48,58.06)(0.45,0.17){1}{\line(1,0){0.45}}
  \multiput(70,58)(0.48,0.06){1}{\line(1,0){0.48}}

  \linethickness{0.3mm}
  \multiput(71.52,61.94)(0.48,0.06){1}{\line(1,0){0.48}}
  \multiput(71.07,61.77)(0.45,0.17){1}{\line(1,0){0.45}}
  \multiput(70.67,61.5)(0.2,0.14){2}{\line(1,0){0.2}}
  \multiput(70.35,61.14)(0.11,0.12){3}{\line(0,1){0.12}}
  \multiput(70.13,60.71)(0.11,0.21){2}{\line(0,1){0.21}}
  \multiput(70.01,60.24)(0.12,0.47){1}{\line(0,1){0.47}}
  \put(70.01,59.76){\line(0,1){0.48}}
  \multiput(70.01,59.76)(0.12,-0.47){1}{\line(0,-1){0.47}}
  \multiput(70.13,59.29)(0.11,-0.21){2}{\line(0,-1){0.21}}
  \multiput(70.35,58.86)(0.11,-0.12){3}{\line(0,-1){0.12}}
  \multiput(70.67,58.5)(0.2,-0.14){2}{\line(1,0){0.2}}
  \multiput(71.07,58.23)(0.45,-0.17){1}{\line(1,0){0.45}}
  \multiput(71.52,58.06)(0.48,-0.06){1}{\line(1,0){0.48}}

  \linethickness{0.3mm}
  \put(72,72){\line(1,0){29}}
  \put(101,72){\line(1,0){0.12}}
  \put(101,72){\circle*{1.5}}
  \put(58,72){\circle*{1.5}}
  \linethickness{0.3mm}
  \put(72,48){\line(1,0){29}}
  \put(101,48){\line(1,0){0.12}}
  \put(101,48){\circle*{1.5}}
  \put(58,48){\circle*{1.5}}
  \put(71,75){\makebox(0,0)[cc]{n:1}}

  \put(26,72){\makebox(0,0)[cc]{$+$}}

  \put(105,72){\makebox(0,0)[cc]{$+$}}

  \put(105,48){\makebox(0,0)[cc]{$-$}}

  \put(26,48){\makebox(0,0)[cc]{$-$}}

  \linethickness{0.3mm}
  \put(30,72){\line(1,0){8}}
  \put(30,72){\line(-1,0){0.12}}
  \put(30,72){\circle*{1.5}}
  \linethickness{0.3mm}
  \put(30,48){\line(1,0){40}}
  \put(30,48){\line(-1,0){0.12}}
  \put(30,48){\circle*{1.5}}
  \linethickness{0.3mm}
  \put(54,72){\line(1,0){16}}
  \linethickness{0.3mm}
  \multiput(38,72)(0.06,0.48){1}{\line(0,1){0.48}}
  \multiput(38.06,72.48)(0.17,0.45){1}{\line(0,1){0.45}}
  \multiput(38.23,72.93)(0.14,0.2){2}{\line(0,1){0.2}}
  \multiput(38.5,73.33)(0.12,0.11){3}{\line(1,0){0.12}}
  \multiput(38.86,73.65)(0.21,0.11){2}{\line(1,0){0.21}}
  \multiput(39.29,73.87)(0.47,0.12){1}{\line(1,0){0.47}}
  \put(39.76,73.99){\line(1,0){0.48}}
  \multiput(40.24,73.99)(0.47,-0.12){1}{\line(1,0){0.47}}
  \multiput(40.71,73.87)(0.21,-0.11){2}{\line(1,0){0.21}}
  \multiput(41.14,73.65)(0.12,-0.11){3}{\line(1,0){0.12}}
  \multiput(41.5,73.33)(0.14,-0.2){2}{\line(0,-1){0.2}}
  \multiput(41.77,72.93)(0.17,-0.45){1}{\line(0,-1){0.45}}
  \multiput(41.94,72.48)(0.06,-0.48){1}{\line(0,-1){0.48}}

  \linethickness{0.3mm}
  \multiput(46,72)(0.06,0.48){1}{\line(0,1){0.48}}
  \multiput(46.06,72.48)(0.17,0.45){1}{\line(0,1){0.45}}
  \multiput(46.23,72.93)(0.14,0.2){2}{\line(0,1){0.2}}
  \multiput(46.5,73.33)(0.12,0.11){3}{\line(1,0){0.12}}
  \multiput(46.86,73.65)(0.21,0.11){2}{\line(1,0){0.21}}
  \multiput(47.29,73.87)(0.47,0.12){1}{\line(1,0){0.47}}
  \put(47.76,73.99){\line(1,0){0.48}}
  \multiput(48.24,73.99)(0.47,-0.12){1}{\line(1,0){0.47}}
  \multiput(48.71,73.87)(0.21,-0.11){2}{\line(1,0){0.21}}
  \multiput(49.14,73.65)(0.12,-0.11){3}{\line(1,0){0.12}}
  \multiput(49.5,73.33)(0.14,-0.2){2}{\line(0,-1){0.2}}
  \multiput(49.77,72.93)(0.17,-0.45){1}{\line(0,-1){0.45}}
  \multiput(49.94,72.48)(0.06,-0.48){1}{\line(0,-1){0.48}}

  \linethickness{0.3mm}
  \multiput(50,72)(0.06,0.48){1}{\line(0,1){0.48}}
  \multiput(50.06,72.48)(0.17,0.45){1}{\line(0,1){0.45}}
  \multiput(50.23,72.93)(0.14,0.2){2}{\line(0,1){0.2}}
  \multiput(50.5,73.33)(0.12,0.11){3}{\line(1,0){0.12}}
  \multiput(50.86,73.65)(0.21,0.11){2}{\line(1,0){0.21}}
  \multiput(51.29,73.87)(0.47,0.12){1}{\line(1,0){0.47}}
  \put(51.76,73.99){\line(1,0){0.48}}
  \multiput(52.24,73.99)(0.47,-0.12){1}{\line(1,0){0.47}}
  \multiput(52.71,73.87)(0.21,-0.11){2}{\line(1,0){0.21}}
  \multiput(53.14,73.65)(0.12,-0.11){3}{\line(1,0){0.12}}
  \multiput(53.5,73.33)(0.14,-0.2){2}{\line(0,-1){0.2}}
  \multiput(53.77,72.93)(0.17,-0.45){1}{\line(0,-1){0.45}}
  \multiput(53.94,72.48)(0.06,-0.48){1}{\line(0,-1){0.48}}

  \linethickness{0.3mm}
  \multiput(42,72)(0.06,0.48){1}{\line(0,1){0.48}}
  \multiput(42.06,72.48)(0.17,0.45){1}{\line(0,1){0.45}}
  \multiput(42.23,72.93)(0.14,0.2){2}{\line(0,1){0.2}}
  \multiput(42.5,73.33)(0.12,0.11){3}{\line(1,0){0.12}}
  \multiput(42.86,73.65)(0.21,0.11){2}{\line(1,0){0.21}}
  \multiput(43.29,73.87)(0.47,0.12){1}{\line(1,0){0.47}}
  \put(43.76,73.99){\line(1,0){0.48}}
  \multiput(44.24,73.99)(0.47,-0.12){1}{\line(1,0){0.47}}
  \multiput(44.71,73.87)(0.21,-0.11){2}{\line(1,0){0.21}}
  \multiput(45.14,73.65)(0.12,-0.11){3}{\line(1,0){0.12}}
  \multiput(45.5,73.33)(0.14,-0.2){2}{\line(0,-1){0.2}}
  \multiput(45.77,72.93)(0.17,-0.45){1}{\line(0,-1){0.45}}
  \multiput(45.94,72.48)(0.06,-0.48){1}{\line(0,-1){0.48}}

  \linethickness{0.3mm}
  \multiput(58,64)(0.48,0.06){1}{\line(1,0){0.48}}
  \multiput(58.48,64.06)(0.45,0.17){1}{\line(1,0){0.45}}
  \multiput(58.93,64.23)(0.2,0.14){2}{\line(1,0){0.2}}
  \multiput(59.33,64.5)(0.11,0.12){3}{\line(0,1){0.12}}
  \multiput(59.65,64.86)(0.11,0.21){2}{\line(0,1){0.21}}
  \multiput(59.87,65.29)(0.12,0.47){1}{\line(0,1){0.47}}
  \put(59.99,65.76){\line(0,1){0.48}}
  \multiput(59.87,66.71)(0.12,-0.47){1}{\line(0,-1){0.47}}
  \multiput(59.65,67.14)(0.11,-0.21){2}{\line(0,-1){0.21}}
  \multiput(59.33,67.5)(0.11,-0.12){3}{\line(0,-1){0.12}}
  \multiput(58.93,67.77)(0.2,-0.14){2}{\line(1,0){0.2}}
  \multiput(58.48,67.94)(0.45,-0.17){1}{\line(1,0){0.45}}
  \multiput(58,68)(0.48,-0.06){1}{\line(1,0){0.48}}

  \linethickness{0.3mm}
  \multiput(58,60)(0.48,0.06){1}{\line(1,0){0.48}}
  \multiput(58.48,60.06)(0.45,0.17){1}{\line(1,0){0.45}}
  \multiput(58.93,60.23)(0.2,0.14){2}{\line(1,0){0.2}}
  \multiput(59.33,60.5)(0.11,0.12){3}{\line(0,1){0.12}}
  \multiput(59.65,60.86)(0.11,0.21){2}{\line(0,1){0.21}}
  \multiput(59.87,61.29)(0.12,0.47){1}{\line(0,1){0.47}}
  \put(59.99,61.76){\line(0,1){0.48}}
  \multiput(59.87,62.71)(0.12,-0.47){1}{\line(0,-1){0.47}}
  \multiput(59.65,63.14)(0.11,-0.21){2}{\line(0,-1){0.21}}
  \multiput(59.33,63.5)(0.11,-0.12){3}{\line(0,-1){0.12}}
  \multiput(58.93,63.77)(0.2,-0.14){2}{\line(1,0){0.2}}
  \multiput(58.48,63.94)(0.45,-0.17){1}{\line(1,0){0.45}}
  \multiput(58,64)(0.48,-0.06){1}{\line(1,0){0.48}}

  \linethickness{0.3mm}
  \multiput(58,56)(0.48,0.06){1}{\line(1,0){0.48}}
  \multiput(58.48,56.06)(0.45,0.17){1}{\line(1,0){0.45}}
  \multiput(58.93,56.23)(0.2,0.14){2}{\line(1,0){0.2}}
  \multiput(59.33,56.5)(0.11,0.12){3}{\line(0,1){0.12}}
  \multiput(59.65,56.86)(0.11,0.21){2}{\line(0,1){0.21}}
  \multiput(59.87,57.29)(0.12,0.47){1}{\line(0,1){0.47}}
  \put(59.99,57.76){\line(0,1){0.48}}
  \multiput(59.87,58.71)(0.12,-0.47){1}{\line(0,-1){0.47}}
  \multiput(59.65,59.14)(0.11,-0.21){2}{\line(0,-1){0.21}}
  \multiput(59.33,59.5)(0.11,-0.12){3}{\line(0,-1){0.12}}
  \multiput(58.93,59.77)(0.2,-0.14){2}{\line(1,0){0.2}}
  \multiput(58.48,59.94)(0.45,-0.17){1}{\line(1,0){0.45}}
  \multiput(58,60)(0.48,-0.06){1}{\line(1,0){0.48}}

  \linethickness{0.3mm}
  \multiput(58,52)(0.48,0.06){1}{\line(1,0){0.48}}
  \multiput(58.48,52.06)(0.45,0.17){1}{\line(1,0){0.45}}
  \multiput(58.93,52.23)(0.2,0.14){2}{\line(1,0){0.2}}
  \multiput(59.33,52.5)(0.11,0.12){3}{\line(0,1){0.12}}
  \multiput(59.65,52.86)(0.11,0.21){2}{\line(0,1){0.21}}
  \multiput(59.87,53.29)(0.12,0.47){1}{\line(0,1){0.47}}
  \put(59.99,53.76){\line(0,1){0.48}}
  \multiput(59.87,54.71)(0.12,-0.47){1}{\line(0,-1){0.47}}
  \multiput(59.65,55.14)(0.11,-0.21){2}{\line(0,-1){0.21}}
  \multiput(59.33,55.5)(0.11,-0.12){3}{\line(0,-1){0.12}}
  \multiput(58.93,55.77)(0.2,-0.14){2}{\line(1,0){0.2}}
  \multiput(58.48,55.94)(0.45,-0.17){1}{\line(1,0){0.45}}
  \multiput(58,56)(0.48,-0.06){1}{\line(1,0){0.48}}

  \linethickness{0.3mm}
  \put(58,68){\line(0,1){4}}
  \linethickness{0.3mm}
  \put(58,48){\line(0,1){4}}
  \put(45,78){\makebox(0,0)[cc]{$\Delta L$}}

  \put(64,62){\makebox(0,0)[cc]{$\dfrac{M^2}{L_2}$}}

  \put(26,60){\makebox(0,0)[cc]{$v_1$}}

  \put(102,60){\makebox(0,0)[cc]{$v_2$}}

  \put(65.23,30.65){\makebox(0,0)[cc]{{\bf Figura 3.8} Schema simbolico
      di trasformatore reale}}
\end{picture}

Per un trasformatore reale abbiamo che il \emph {coefficiente di mutua
  induzione} {\bf $M$} risulta essere molto minore del prodotto $L_1L_2$
\[
M^2<<L_1L_2
\]

Prendendo in esame $L_1$ abbiamo

\[
L_1=L'+\Delta L \implies \Delta L=L_1-\dfrac{M^2}{L_2}
\]

\[
L_1 \implies L'_1L_2=M^2
\]

\[
\left\{
\begin{array}{l l}
  v_1=L'_1\dfrac{d\;i_1}{dt}+M\dfrac{d\;i_2}{dt}+\Delta L\; \dfrac{d\;i_1}{dt} \\
  v_2=L_2\dfrac{d\;i_2}{dt}+M\dfrac{d\;i_1}{dt} \\
\end{array} \right.
\]
\vspace{1cm}
\section{Doppi Bipoli di Resistori Lineari}
\subsection{Rappresentazione dei Doppi Bipoli}

Un doppio bipolo a-dinamico in generale \`e descritto da due realzioni
algebriche che legano l'intensit\`a di corrente e le tensioni alle due
porte. Quando le realzioni sono esplicite danno origine a diverse
rappresentazioni:

\begin{itemize}
  \item Rappresentazione su {\textsc{\bf Base Corrente}}: le
    intensit\`a di corrente $i_1$ e $i_2$ sono le variabili {\bf
      indipendenti} e le tensioni $v_1$ e $v_2$ sono le variabili {\bf
    dipendenti}
  \item Rappresentazione su {\textsc{\bf Base Tensione}}: le tensioni
    $v_1$ e $v_2$ sono le variabili {\bf indipendenti} e le correnti
    $i_1$ e $i_2$ sono le variabili {\bf dipendenti}.
  \item Rappresentazione su {\textsc{\bf Base Ibrida}}: la tensione
    $v_1$ e $v_2$ sono le variabili {\bf indipendenti} e la corrente
    $i_1$ e $v_1$ sono le variabili {\bf dipendenti} (o viceversa).
  \item Rappresentazione di {\textsc{\bf Trasmissione}}: la tensione
    $v_1$ e la corrente $i_1$ sono le variabili {\bf indipendenti} e
    la tensione $v_2$ e la corrente $i_2$ sono le variabili {\bf
      dipendenti} (o viceversa)
\end{itemize}

\subsection{Propriet\`a di Reciprocit\`a}

Vi sono tre forme per la propriet\`a di reciprocit\`a e sono le
seguenti.

\subsubsection{Prima Forma}

Cosideriamo i circuiti come in {\bf Figura 3.9}.

\ifx\JPicScale\undefined\def\JPicScale{1}\fi
\unitlength \JPicScale mm
\begin{picture}(162,80)(15,15)
\linethickness{0.3mm}
\put(20,74){\line(1,0){40}}
\put(20,38){\line(0,1){36}}
\put(60,38){\line(0,1){36}}
\put(20,38){\line(1,0){40}}
\linethickness{0.3mm}
\put(6,68){\line(1,0){14}}
\linethickness{0.3mm}
\put(6,44){\line(1,0){14}}
\linethickness{0.3mm}
\put(60,44){\line(1,0){14}}
\linethickness{0.3mm}
\put(60,68){\line(1,0){14}}
\linethickness{0.3mm}
\put(101,74){\line(1,0){40}}
\put(101,38){\line(0,1){36}}
\put(141,38){\line(0,1){36}}
\put(101,38){\line(1,0){40}}
\linethickness{0.3mm}
\put(87,68){\line(1,0){14}}
\linethickness{0.3mm}
\put(87,44){\line(1,0){14}}
\linethickness{0.3mm}
\put(141,44){\line(1,0){14}}
\linethickness{0.3mm}
\put(141,68){\line(1,0){14}}
\linethickness{0.3mm}
\put(6,56){\circle{8}}

\linethickness{0.3mm}
\put(6,60){\line(0,1){8}}
\linethickness{0.3mm}
\put(6,44){\line(0,1){8}}
\linethickness{0.3mm}
\put(155,56){\circle{8}}

\linethickness{0.3mm}
\put(155,60){\line(0,1){8}}
\linethickness{0.3mm}
\put(155,44){\line(0,1){8}}
\linethickness{0.3mm}
\put(74,44){\line(0,1){24}}
\linethickness{0.3mm}
\put(87,44){\line(0,1){24}}
\linethickness{0.3mm}
\multiput(12,69)(0.12,-0.12){8}{\line(1,0){0.12}}
\linethickness{0.3mm}
\multiput(12,67)(0.12,0.12){8}{\line(1,0){0.12}}
\linethickness{0.3mm}
\multiput(65,68)(0.12,0.12){8}{\line(1,0){0.12}}
\linethickness{0.3mm}
\multiput(65,68)(0.12,-0.12){8}{\line(1,0){0.12}}
\linethickness{0.3mm}
\multiput(95,69)(0.12,-0.12){8}{\line(1,0){0.12}}
\linethickness{0.3mm}
\multiput(95,67)(0.12,0.12){8}{\line(1,0){0.12}}
\linethickness{0.3mm}
\multiput(147,68)(0.12,0.12){8}{\line(1,0){0.12}}
\linethickness{0.3mm}
\multiput(147,68)(0.12,-0.12){8}{\line(1,0){0.12}}
\put(6,58){\makebox(0,0)[cc]{$+$}}

\put(6,54){\makebox(0,0)[cc]{$-$}}

\put(155,58){\makebox(0,0)[cc]{$+$}}

\put(155,54){\makebox(0,0)[cc]{$-$}}

\linethickness{0.3mm}
\multiput(31.88,57)(0.12,0.17){24}{\line(0,1){0.17}}
\linethickness{0.3mm}
\multiput(34.75,61)(0.12,-0.17){24}{\line(0,-1){0.17}}
\linethickness{0.3mm}
\multiput(37.62,57)(0.12,0.17){24}{\line(0,1){0.17}}
\linethickness{0.3mm}
\multiput(40.5,61)(0.12,-0.17){24}{\line(0,-1){0.17}}
\linethickness{0.3mm}
\multiput(43.38,57)(0.12,0.17){24}{\line(0,1){0.17}}
\linethickness{0.3mm}
\multiput(46.25,61)(0.12,-0.17){24}{\line(0,-1){0.17}}
\linethickness{0.3mm}
\put(49.12,57){\line(1,0){2.88}}
\put(37.62,65){\makebox(0,0)[bl]{$R$}}

\linethickness{0.15mm}
\put(29,57){\line(1,0){2.88}}
\linethickness{0.3mm}
\multiput(111.88,57)(0.12,0.17){24}{\line(0,1){0.17}}
\linethickness{0.3mm}
\multiput(114.75,61)(0.12,-0.17){24}{\line(0,-1){0.17}}
\linethickness{0.3mm}
\multiput(117.62,57)(0.12,0.17){24}{\line(0,1){0.17}}
\linethickness{0.3mm}
\multiput(120.5,61)(0.12,-0.17){24}{\line(0,-1){0.17}}
\linethickness{0.3mm}
\multiput(123.38,57)(0.12,0.17){24}{\line(0,1){0.17}}
\linethickness{0.3mm}
\multiput(126.25,61)(0.12,-0.17){24}{\line(0,-1){0.17}}
\linethickness{0.3mm}
\put(129.12,57){\line(1,0){2.88}}
\put(117.62,65){\makebox(0,0)[bl]{$R$}}

\linethickness{0.15mm}
\put(109,57){\line(1,0){2.88}}
\put(12,71){\makebox(0,0)[cc]{$i'_1$}}

\put(-1,56){\makebox(0,0)[cc]{$v_1$}}

\put(40,80){\makebox(0,0)[cc]{$C'$}}

\put(121,79){\makebox(0,0)[cc]{$C''$}}

\put(94,71){\makebox(0,0)[cc]{$i''_1$}}

\put(66,55){\makebox(0,0)[cc]{$v'_2=0$}}

\put(66,71){\makebox(0,0)[cc]{$i'_2$}}

\put(93,55){\makebox(0,0)[cc]{$v''_1=0$}}

\put(147,71){\makebox(0,0)[cc]{$i''_2$}}

\put(162,56){\makebox(0,0)[cc]{$v_2$}}

\put(29,57){\circle*{1.5}}
\put(52,57){\circle*{1.5}}
\put(69,68){\circle*{1.5}}
\put(69,44){\circle*{1.5}}
\put(91,68){\circle*{1.5}}
\put(91,44){\circle*{1.5}}
\put(109,57){\circle*{1.5}}
\put(132,57){\circle*{1.5}}

\put(75,20){\makebox(0,0)[cc]{{\bf Figura 3.9} Prima forma di
    Reciprocit\`a}}
\end{picture}

Per il torema di {\bf Tellegen} possiamo scrivere che:

\begin{equation}\label{equ:reciprocitaT1}
v_1i''_1+v'_2i''_2+\sum_{k=1}^{N_R} v'_ki''_k=0
\end{equation}

\begin{equation}\label{equ:reciprocitaT2}
v''_1i'_1+v_2i'_2+\sum_{k=1}^{N_R} v''_ki'_k=0
\end{equation}

Dove {\bf $N_R$} \`e il numero di resisori presenti all'interno del
doppio bipolo.

\`E immediato verificare che i termini $v'_2i''_2$ e $v''_1i'_1$ sono
nulli.

Considerand che nella relazione caratteristica del $k-esimo$ resistore
si ha che:

\[
\mathbf{v'_ki''_k}=R_ki'_ki''_k=R_ki''_ki'_k=\mathbf{v''_ki'_k}
\]

Sottraendo membro a membro le due equazioni \ref{equ:reciprocitaT1} e
\ref{equ:reciprocitaT2} restano i termini:

\[
v_1i''_2-v_2i'_2=0 \Longrightarrow v_1i''_1=v_2i'_2
\]


\subsubsection{Seconda Forma}

\ifx\JPicScale\undefined\def\JPicScale{1}\fi
\unitlength \JPicScale mm
\begin{picture}(162,80)(15,15)
\linethickness{0.3mm}
\put(20,74){\line(1,0){40}}
\put(20,38){\line(0,1){36}}
\put(60,38){\line(0,1){36}}
\put(20,38){\line(1,0){40}}
\linethickness{0.3mm}
\put(6,68){\line(1,0){14}}
\linethickness{0.3mm}
\put(6,44){\line(1,0){14}}
\linethickness{0.3mm}
\put(60,44){\line(1,0){14}}
\linethickness{0.3mm}
\put(60,68){\line(1,0){14}}
\linethickness{0.3mm}
\put(101,74){\line(1,0){40}}
\put(101,38){\line(0,1){36}}
\put(141,38){\line(0,1){36}}
\put(101,38){\line(1,0){40}}
\linethickness{0.3mm}
\put(87,68){\line(1,0){14}}
\linethickness{0.3mm}
\put(87,44){\line(1,0){14}}
\linethickness{0.3mm}
\put(141,44){\line(1,0){14}}
\linethickness{0.3mm}
\put(141,68){\line(1,0){14}}
\linethickness{0.3mm}
\put(6,56){\circle{8}}

\linethickness{0.3mm}
\put(6,60){\line(0,1){8}}
\linethickness{0.3mm}
\put(6,44){\line(0,1){8}}
\linethickness{0.3mm}
\put(155,56){\circle{8}}

\linethickness{0.3mm}
\put(155,60){\line(0,1){8}}
\linethickness{0.3mm}
\put(155,44){\line(0,1){8}}
\linethickness{0.3mm}
\multiput(12,69)(0.12,-0.12){8}{\line(1,0){0.12}}
\linethickness{0.3mm}
\multiput(12,67)(0.12,0.12){8}{\line(1,0){0.12}}
\linethickness{0.3mm}
\multiput(65,68)(0.12,0.12){8}{\line(1,0){0.12}}
\linethickness{0.3mm}
\multiput(65,68)(0.12,-0.12){8}{\line(1,0){0.12}}
\linethickness{0.3mm}
\multiput(95,69)(0.12,-0.12){8}{\line(1,0){0.12}}
\linethickness{0.3mm}
\multiput(95,67)(0.12,0.12){8}{\line(1,0){0.12}}
\linethickness{0.3mm}
\multiput(147,68)(0.12,0.12){8}{\line(1,0){0.12}}
\linethickness{0.3mm}
\multiput(147,68)(0.12,-0.12){8}{\line(1,0){0.12}}
\linethickness{0.3mm}
\multiput(31.88,57)(0.12,0.17){24}{\line(0,1){0.17}}
\linethickness{0.3mm}
\multiput(34.75,61)(0.12,-0.17){24}{\line(0,-1){0.17}}
\linethickness{0.3mm}
\multiput(37.62,57)(0.12,0.17){24}{\line(0,1){0.17}}
\linethickness{0.3mm}
\multiput(40.5,61)(0.12,-0.17){24}{\line(0,-1){0.17}}
\linethickness{0.3mm}
\multiput(43.38,57)(0.12,0.17){24}{\line(0,1){0.17}}
\linethickness{0.3mm}
\multiput(46.25,61)(0.12,-0.17){24}{\line(0,-1){0.17}}
\linethickness{0.3mm}
\put(49.12,57){\line(1,0){2.88}}
\put(37.62,65){\makebox(0,0)[bl]{$R$}}

\linethickness{0.15mm}
\put(29,57){\line(1,0){2.88}}
\linethickness{0.3mm}
\multiput(111.88,57)(0.12,0.17){24}{\line(0,1){0.17}}
\linethickness{0.3mm}
\multiput(114.75,61)(0.12,-0.17){24}{\line(0,-1){0.17}}
\linethickness{0.3mm}
\multiput(117.62,57)(0.12,0.17){24}{\line(0,1){0.17}}
\linethickness{0.3mm}
\multiput(120.5,61)(0.12,-0.17){24}{\line(0,-1){0.17}}
\linethickness{0.3mm}
\multiput(123.38,57)(0.12,0.17){24}{\line(0,1){0.17}}
\linethickness{0.3mm}
\multiput(126.25,61)(0.12,-0.17){24}{\line(0,-1){0.17}}
\linethickness{0.3mm}
\put(129.12,57){\line(1,0){2.88}}
\put(117.62,65){\makebox(0,0)[bl]{$R$}}

\linethickness{0.15mm}
\put(109,57){\line(1,0){2.88}}
\put(12,71){\makebox(0,0)[cc]{$i'_1$}}

\put(-1,56){\makebox(0,0)[cc]{$i_1$}}

\put(40,80){\makebox(0,0)[cc]{$C'$}}

\put(121,79){\makebox(0,0)[cc]{$C''$}}

\put(162,56){\makebox(0,0)[cc]{$i_2$}}

\linethickness{0.3mm}
\put(6,53){\line(0,1){6}}
\put(6,59){\vector(0,1){0.12}}
\linethickness{0.3mm}
\put(155,53){\line(0,1){6}}
\put(155,59){\vector(0,1){0.12}}
\put(75,55){\makebox(0,0)[cc]{$v'_2$}}

\put(88,55){\makebox(0,0)[cc]{$v''_1$}}

\put(77,68){\makebox(0,0)[cc]{$+$}}

\put(77,44){\makebox(0,0)[cc]{$-$}}

\put(84,68){\makebox(0,0)[cc]{$+$}}

\put(84,44){\makebox(0,0)[cc]{$-$}}

\put(66,71){\makebox(0,0)[cc]{$i'_2=0$}}

\put(94,71){\makebox(0,0)[cc]{$i''_1=0$}}

\put(15,56){\makebox(0,0)[cc]{$v'_1$}}

\put(16,65){\makebox(0,0)[cc]{$+$}}

\put(16,46){\makebox(0,0)[cc]{$-$}}

\put(144,65){\makebox(0,0)[cc]{$+$}}

\put(144,46){\makebox(0,0)[cc]{$-$}}

\put(145,56){\makebox(0,0)[cc]{$v''_2$}}

\put(16,68){\circle*{1.5}}
\put(16,44){\circle*{1.5}}
\put(74,68){\circle*{1.5}}
\put(74,44){\circle*{1.5}}
\put(87,68){\circle*{1.5}}
\put(87,44){\circle*{1.5}}
\put(29,57){\circle*{1.5}}
\put(52,57){\circle*{1.5}}
\put(109,57){\circle*{1.5}}
\put(132,57){\circle*{1.5}}
\put(144,68){\circle*{1.5}}
\put(144,44){\circle*{1.5}}

\put(75,20){\makebox(0,0)[cc]{{\bf Figura 3.10} Seconda forma di
    Reciprocit\`a}}
\end{picture}

\subsubsection{Terza Forma}

\ifx\JPicScale\undefined\def\JPicScale{1}\fi
\unitlength \JPicScale mm
\begin{picture}(162,80)(15,15)
\linethickness{0.3mm}
\put(20,74){\line(1,0){40}}
\put(20,38){\line(0,1){36}}
\put(60,38){\line(0,1){36}}
\put(20,38){\line(1,0){40}}
\linethickness{0.3mm}
\put(6,68){\line(1,0){14}}
\linethickness{0.3mm}
\put(6,44){\line(1,0){14}}
\linethickness{0.3mm}
\put(60,44){\line(1,0){14}}
\linethickness{0.3mm}
\put(60,68){\line(1,0){14}}
\linethickness{0.3mm}
\put(101,74){\line(1,0){40}}
\put(101,38){\line(0,1){36}}
\put(141,38){\line(0,1){36}}
\put(101,38){\line(1,0){40}}
\linethickness{0.3mm}
\put(87,68){\line(1,0){14}}
\linethickness{0.3mm}
\put(87,44){\line(1,0){14}}
\linethickness{0.3mm}
\put(141,44){\line(1,0){14}}
\linethickness{0.3mm}
\put(141,68){\line(1,0){14}}
\linethickness{0.3mm}
\put(6,56){\circle{8}}

\linethickness{0.3mm}
\put(6,60){\line(0,1){8}}
\linethickness{0.3mm}
\put(6,44){\line(0,1){8}}
\linethickness{0.3mm}
\put(155,56){\circle{8}}

\linethickness{0.3mm}
\put(155,60){\line(0,1){8}}
\linethickness{0.3mm}
\put(155,44){\line(0,1){8}}
\linethickness{0.3mm}
\multiput(12,69)(0.12,-0.12){8}{\line(1,0){0.12}}
\linethickness{0.3mm}
\multiput(12,67)(0.12,0.12){8}{\line(1,0){0.12}}
\linethickness{0.3mm}
\multiput(65,68)(0.12,0.12){8}{\line(1,0){0.12}}
\linethickness{0.3mm}
\multiput(65,68)(0.12,-0.12){8}{\line(1,0){0.12}}
\linethickness{0.3mm}
\multiput(95,69)(0.12,-0.12){8}{\line(1,0){0.12}}
\linethickness{0.3mm}
\multiput(95,67)(0.12,0.12){8}{\line(1,0){0.12}}
\linethickness{0.3mm}
\multiput(147,68)(0.12,0.12){8}{\line(1,0){0.12}}
\linethickness{0.3mm}
\multiput(147,68)(0.12,-0.12){8}{\line(1,0){0.12}}
\linethickness{0.3mm}
\multiput(31.88,57)(0.12,0.17){24}{\line(0,1){0.17}}
\linethickness{0.3mm}
\multiput(34.75,61)(0.12,-0.17){24}{\line(0,-1){0.17}}
\linethickness{0.3mm}
\multiput(37.62,57)(0.12,0.17){24}{\line(0,1){0.17}}
\linethickness{0.3mm}
\multiput(40.5,61)(0.12,-0.17){24}{\line(0,-1){0.17}}
\linethickness{0.3mm}
\multiput(43.38,57)(0.12,0.17){24}{\line(0,1){0.17}}
\linethickness{0.3mm}
\multiput(46.25,61)(0.12,-0.17){24}{\line(0,-1){0.17}}
\linethickness{0.3mm}
\put(49.12,57){\line(1,0){2.88}}
\put(37.62,65){\makebox(0,0)[bl]{$R$}}

\linethickness{0.15mm}
\put(29,57){\line(1,0){2.88}}
\linethickness{0.3mm}
\multiput(111.88,57)(0.12,0.17){24}{\line(0,1){0.17}}
\linethickness{0.3mm}
\multiput(114.75,61)(0.12,-0.17){24}{\line(0,-1){0.17}}
\linethickness{0.3mm}
\multiput(117.62,57)(0.12,0.17){24}{\line(0,1){0.17}}
\linethickness{0.3mm}
\multiput(120.5,61)(0.12,-0.17){24}{\line(0,-1){0.17}}
\linethickness{0.3mm}
\multiput(123.38,57)(0.12,0.17){24}{\line(0,1){0.17}}
\linethickness{0.3mm}
\multiput(126.25,61)(0.12,-0.17){24}{\line(0,-1){0.17}}
\linethickness{0.3mm}
\put(129.12,57){\line(1,0){2.88}}
\put(117.62,65){\makebox(0,0)[bl]{$R$}}

\linethickness{0.15mm}
\put(109,57){\line(1,0){2.88}}
\put(12,71){\makebox(0,0)[cc]{$i'_1$}}

\put(-1,56){\makebox(0,0)[cc]{$i_1$}}

\put(40,80){\makebox(0,0)[cc]{$C'$}}

\put(121,79){\makebox(0,0)[cc]{$C''$}}

\put(147,71){\makebox(0,0)[cc]{$i''_2$}}

\put(162,56){\makebox(0,0)[cc]{$v_2$}}

\linethickness{0.3mm}
\put(6,53){\line(0,1){6}}
\put(6,59){\vector(0,1){0.12}}
\put(66,55){\makebox(0,0)[cc]{$v'_2=0$}}

\put(85,55){\makebox(0,0)[cc]{$v''_1$}}

\put(84,68){\makebox(0,0)[cc]{$+$}}

\put(84,44){\makebox(0,0)[cc]{$-$}}

\put(66,71){\makebox(0,0)[cc]{$i'_2$}}

\put(94,71){\makebox(0,0)[cc]{$i''_1=0$}}

\put(15,56){\makebox(0,0)[cc]{$v'_1$}}

\put(16,65){\makebox(0,0)[cc]{$+$}}

\put(16,46){\makebox(0,0)[cc]{$-$}}

\linethickness{0.3mm}
\put(74,44){\line(0,1){24}}
\put(155,58){\makebox(0,0)[cc]{$+$}}

\put(155,54){\makebox(0,0)[cc]{$-$}}

\put(16,68){\circle*{1.5}}
\put(16,44){\circle*{1.5}}
\put(87,68){\circle*{1.5}}
\put(87,44){\circle*{1.5}}
\put(29,57){\circle*{1.5}}
\put(52,57){\circle*{1.5}}
\put(109,57){\circle*{1.5}}
\put(132,57){\circle*{1.5}}

\put(75,20){\makebox(0,0)[cc]{{\bf Figura 3.11} Terza forma di
    Reciprocit\`a}}
\end{picture}



\chapter{Circuiti a-Dinamici Lineari}
\section{Generatore Equivalente di Th\`evenin-Norton}
Consideriamo un generico circuito costituito da resistori lineari e
generatori reali come in figura.

\ifx\JPicScale\undefined\def\JPicScale{1}\fi
\unitlength \JPicScale mm
\begin{picture}(94,70)(15,15)
\linethickness{0.3mm}
\put(60,70){\line(1,0){34}}
\put(60,36){\line(0,1){34}}
\put(94,36){\line(0,1){34}}
\put(60,36){\line(1,0){34}}
\linethickness{0.3mm}
\put(48,64){\line(1,0){12}}
\put(48,64){\line(-1,0){0.12}}
\put(48,64){\circle*{1.5}}
\linethickness{0.3mm}
\put(48,42){\line(1,0){12}}
\put(48,42){\line(-1,0){0.12}}
\put(48,42){\circle*{1.5}}
\linethickness{0.3mm}
\put(77,62){\circle{8}}

\linethickness{0.3mm}
\put(77,52){\circle{8}}

\linethickness{0.3mm}
\put(66,62){\line(1,0){7}}
\put(66,62){\line(-1,0){0.12}}
\put(66,62){\circle*{1.5}}
\linethickness{0.3mm}
\put(81,62){\line(1,0){7}}
\put(88,62){\line(1,0){0.12}}
\put(88,62){\circle*{1.5}}
\linethickness{0.3mm}
\put(66,52){\line(1,0){7}}
\put(66,52){\line(-1,0){0.12}}
\put(66,52){\circle*{1.5}}
\linethickness{0.3mm}
\put(81,52){\line(1,0){7}}
\put(88,52){\line(1,0){0.12}}
\put(88,52){\circle*{1.5}}
\linethickness{0.3mm}
\multiput(72.5,42)(0.12,0.15){13}{\line(0,1){0.15}}
\linethickness{0.3mm}
\multiput(74,44)(0.12,-0.15){13}{\line(0,-1){0.15}}
\linethickness{0.3mm}
\multiput(75.5,42)(0.12,0.15){13}{\line(0,1){0.15}}
\linethickness{0.3mm}
\multiput(77,44)(0.12,-0.15){13}{\line(0,-1){0.15}}
\linethickness{0.3mm}
\multiput(78.5,42)(0.12,0.15){13}{\line(0,1){0.15}}
\linethickness{0.3mm}
\multiput(80,44)(0.12,-0.15){13}{\line(0,-1){0.15}}
\linethickness{0.3mm}
\put(81.5,42){\line(1,0){1.5}}
\linethickness{0.15mm}
\put(71,42){\line(1,0){1.5}}
\linethickness{0.3mm}
\put(82,42){\line(1,0){6}}
\put(88,42){\line(1,0){0.12}}
\put(88,42){\circle*{1.5}}
\linethickness{0.3mm}
\put(66,42){\line(1,0){6}}
\put(66,42){\line(-1,0){0.12}}
\put(66,42){\circle*{1.5}}
\put(79,62){\makebox(0,0)[cc]{$+$}}

\linethickness{0.3mm}
\put(74.38,61.38){\line(0,1){1}}
\put(44,64){\makebox(0,0)[cc]{$+$}}

\put(44,42){\makebox(0,0)[cc]{$-$}}

\put(46,53){\makebox(0,0)[cc]{$v$}}

\linethickness{0.3mm}
\multiput(53,65)(0.12,-0.12){8}{\line(1,0){0.12}}
\linethickness{0.3mm}
\multiput(53,63)(0.12,0.12){8}{\line(1,0){0.12}}
\put(52,67){\makebox(0,0)[cc]{$i$}}

\linethickness{0.3mm}
\put(74,52){\line(1,0){6}}
\put(80,52){\vector(1,0){0.12}}

\put(72.23,25.65){\makebox(0,0)[cc]{{\bf Figura 4.1} Circuito generico
costutito da resistori e generatori lineari}}
\end{picture}

Dobbiamo determinare la relazione caratteristica tra l'intensit\`a di
corrente $i$ e la tensione $v$ per tutti i volori ammessi.
Pu\`o essere effettuato su base tensione (\emph{Norton}) o su base
corrente (\emph{Th\`evenin}).

Prendiamo in esame la caratterizzazione base corrente.

\subsection{Generatore Equivalente di Th\`evenin}
Se prensiamo in esame la caratterizzazione base corrente applichiamo
al circuito generico un generatore di corrente.

\ifx\JPicScale\undefined\def\JPicScale{1}\fi
\unitlength \JPicScale mm
\begin{picture}(94,70)(15,15)
\linethickness{0.3mm}
\put(60,70){\line(1,0){34}}
\put(60,36){\line(0,1){34}}
\put(94,36){\line(0,1){34}}
\put(60,36){\line(1,0){34}}
\linethickness{0.3mm}
\put(48,64){\line(1,0){12}}
\put(48,64){\line(-1,0){0.12}}
\put(48,64){\circle*{1.5}}
\linethickness{0.3mm}
\put(48,42){\line(1,0){12}}
\put(48,42){\line(-1,0){0.12}}
\put(48,42){\circle*{1.5}}
\linethickness{0.3mm}
\put(77,62){\circle{8}}

\linethickness{0.3mm}
\put(77,52){\circle{8}}

\linethickness{0.3mm}
\put(66,62){\line(1,0){7}}
\put(66,62){\line(-1,0){0.12}}
\put(66,62){\circle*{1.5}}
\linethickness{0.3mm}
\put(81,62){\line(1,0){7}}
\put(88,62){\line(1,0){0.12}}
\put(88,62){\circle*{1.5}}
\linethickness{0.3mm}
\put(66,52){\line(1,0){7}}
\put(66,52){\line(-1,0){0.12}}
\put(66,52){\circle*{1.5}}
\linethickness{0.3mm}
\put(81,52){\line(1,0){7}}
\put(88,52){\line(1,0){0.12}}
\put(88,52){\circle*{1.5}}
\linethickness{0.3mm}
\multiput(72.5,42)(0.12,0.15){13}{\line(0,1){0.15}}
\linethickness{0.3mm}
\multiput(74,44)(0.12,-0.15){13}{\line(0,-1){0.15}}
\linethickness{0.3mm}
\multiput(75.5,42)(0.12,0.15){13}{\line(0,1){0.15}}
\linethickness{0.3mm}
\multiput(77,44)(0.12,-0.15){13}{\line(0,-1){0.15}}
\linethickness{0.3mm}
\multiput(78.5,42)(0.12,0.15){13}{\line(0,1){0.15}}
\linethickness{0.3mm}
\multiput(80,44)(0.12,-0.15){13}{\line(0,-1){0.15}}
\linethickness{0.3mm}
\put(81.5,42){\line(1,0){1.5}}

\linethickness{0.15mm}
\put(71,42){\line(1,0){1.5}}
\linethickness{0.3mm}
\put(82,42){\line(1,0){6}}
\put(88,42){\line(1,0){0.12}}
\put(88,42){\circle*{1.5}}
\linethickness{0.3mm}
\put(66,42){\line(1,0){6}}
\put(66,42){\line(-1,0){0.12}}
\put(66,42){\circle*{1.5}}
\put(79,62){\makebox(0,0)[cc]{$+$}}

\linethickness{0.3mm}
\put(74.38,61.38){\line(0,1){1}}
\put(44,64){\makebox(0,0)[cc]{$+$}}

\put(44,42){\makebox(0,0)[cc]{$-$}}

\put(53,58){\makebox(0,0)[cc]{$v$}}

\linethickness{0.3mm}
\multiput(53,65)(0.12,-0.12){8}{\line(1,0){0.12}}
\linethickness{0.3mm}
\multiput(53,63)(0.12,0.12){8}{\line(1,0){0.12}}
\linethickness{0.3mm}
\put(74,52){\line(1,0){6}}
\put(80,52){\vector(1,0){0.12}}
\linethickness{0.3mm}
\put(48,53){\circle{8}}

\linethickness{0.3mm}
\put(48,57){\line(0,1){7}}
\linethickness{0.3mm}
\put(48,42){\line(0,1){7}}
\linethickness{0.3mm}
\put(48,50){\line(0,1){6}}
\put(48,56){\vector(0,1){0.12}}
\put(41,53){\makebox(0,0)[cc]{$i$}}

\put(72.23,25.65){\makebox(0,0)[cc]{{\bf Figura 4.2} Circuito generico
con generatore di corrente}}
\end{picture}

Ora possiamo spegnere, all'interno del bipolo generico, i generatori
indipendenti come in {\bf Figura 4.3}. Poi spegnamo il generatore di corrente per la
caratterizzazione come in {\bf Figura 4.4}.

\ifx\JPicScale\undefined\def\JPicScale{1}\fi
\unitlength \JPicScale mm
\begin{picture}(90,70)(50,10)
\linethickness{0.3mm}
\put(60,70){\line(1,0){34}}
\put(60,36){\line(0,1){34}}
\put(94,36){\line(0,1){34}}
\put(60,36){\line(1,0){34}}
\linethickness{0.3mm}
\put(48,64){\line(1,0){12}}
\put(48,64){\line(-1,0){0.12}}
\put(48,64){\circle*{1.5}}
\linethickness{0.3mm}
\put(48,42){\line(1,0){12}}
\put(48,42){\line(-1,0){0.12}}
\put(48,42){\circle*{1.5}}
\linethickness{0.3mm}
\put(66,62){\line(1,0){16}}
\put(66,62){\line(-1,0){0.12}}
\put(66,62){\circle*{1.5}}
\linethickness{0.3mm}
\put(81,62){\line(1,0){7}}
\put(88,62){\line(1,0){0.12}}
\put(88,62){\circle*{1.5}}
\linethickness{0.3mm}
\put(66,52){\line(1,0){7}}
\put(73,52){\line(1,0){0.12}}
\put(73,52){\circle*{1.5}}
\put(66,52){\line(-1,0){0.12}}
\put(66,52){\circle*{1.5}}
\linethickness{0.3mm}
\put(81,52){\line(1,0){7}}
\put(88,52){\line(1,0){0.12}}
\put(88,52){\circle*{1.5}}
\put(81,52){\line(-1,0){0.12}}
\put(81,52){\circle*{1.5}}
\linethickness{0.3mm}
\multiput(72.5,42)(0.12,0.15){13}{\line(0,1){0.15}}
\linethickness{0.3mm}
\multiput(74,44)(0.12,-0.15){13}{\line(0,-1){0.15}}
\linethickness{0.3mm}
\multiput(75.5,42)(0.12,0.15){13}{\line(0,1){0.15}}
\linethickness{0.3mm}
\multiput(77,44)(0.12,-0.15){13}{\line(0,-1){0.15}}
\linethickness{0.3mm}
\multiput(78.5,42)(0.12,0.15){13}{\line(0,1){0.15}}
\linethickness{0.3mm}
\multiput(80,44)(0.12,-0.15){13}{\line(0,-1){0.15}}
\linethickness{0.3mm}
\put(81.5,42){\line(1,0){1.5}}

\linethickness{0.15mm}
\put(71,42){\line(1,0){1.5}}
\linethickness{0.3mm}
\put(82,42){\line(1,0){6}}
\put(88,42){\line(1,0){0.12}}
\put(88,42){\circle*{1.5}}
\linethickness{0.3mm}
\put(66,42){\line(1,0){6}}
\put(66,42){\line(-1,0){0.12}}
\put(66,42){\circle*{1.5}}
\put(44,64){\makebox(0,0)[cc]{$+$}}

\put(44,42){\makebox(0,0)[cc]{$-$}}

\put(53,58){\makebox(0,0)[cc]{$v'$}}

\linethickness{0.3mm}
\put(48,53){\circle{8}}

\linethickness{0.3mm}
\put(48,57){\line(0,1){7}}
\linethickness{0.3mm}
\put(48,42){\line(0,1){7}}
\linethickness{0.3mm}
\put(48,50){\line(0,1){6}}
\put(48,56){\vector(0,1){0.12}}
\put(41,53){\makebox(0,0)[cc]{$i$}}

\put(75,25){\makebox (0,0)[cc]{{\bf Figura 4.3}}}
\end{picture}

\ifx\JPicScale\undefined\def\JPicScale{1}\fi
\unitlength \JPicScale mm
\begin{picture}(0,10)(-14.5,0)
\linethickness{0.3mm}
\put(60,70){\line(1,0){34}}
\put(60,36){\line(0,1){34}}
\put(94,36){\line(0,1){34}}
\put(60,36){\line(1,0){34}}
\linethickness{0.3mm}
\put(48,64){\line(1,0){12}}
\put(48,64){\line(-1,0){0.12}}
\put(48,64){\circle*{1.5}}
\linethickness{0.3mm}
\put(48,42){\line(1,0){12}}
\put(48,42){\line(-1,0){0.12}}
\put(48,42){\circle*{1.5}}
\linethickness{0.3mm}
\put(77,62){\circle{8}}

\linethickness{0.3mm}
\put(77,52){\circle{8}}

\linethickness{0.3mm}
\put(66,62){\line(1,0){7}}
\put(66,62){\line(-1,0){0.12}}
\put(66,62){\circle*{1.5}}
\linethickness{0.3mm}
\put(81,62){\line(1,0){7}}
\put(88,62){\line(1,0){0.12}}
\put(88,62){\circle*{1.5}}
\linethickness{0.3mm}
\put(66,52){\line(1,0){7}}
\put(66,52){\line(-1,0){0.12}}
\put(66,52){\circle*{1.5}}
\linethickness{0.3mm}
\put(81,52){\line(1,0){7}}
\put(88,52){\line(1,0){0.12}}
\put(88,52){\circle*{1.5}}
\linethickness{0.3mm}
\multiput(72.5,42)(0.12,0.15){13}{\line(0,1){0.15}}
\linethickness{0.3mm}
\multiput(74,44)(0.12,-0.15){13}{\line(0,-1){0.15}}
\linethickness{0.3mm}
\multiput(75.5,42)(0.12,0.15){13}{\line(0,1){0.15}}
\linethickness{0.3mm}
\multiput(77,44)(0.12,-0.15){13}{\line(0,-1){0.15}}
\linethickness{0.3mm}
\multiput(78.5,42)(0.12,0.15){13}{\line(0,1){0.15}}
\linethickness{0.3mm}
\multiput(80,44)(0.12,-0.15){13}{\line(0,-1){0.15}}
\linethickness{0.3mm}
\put(81.5,42){\line(1,0){1.5}}
\linethickness{0.15mm}
\put(71,42){\line(1,0){1.5}}
\linethickness{0.3mm}
\put(82,42){\line(1,0){6}}
\put(88,42){\line(1,0){0.12}}
\put(88,42){\circle*{1.5}}
\linethickness{0.3mm}
\put(66,42){\line(1,0){6}}
\put(66,42){\line(-1,0){0.12}}
\put(66,42){\circle*{1.5}}
\put(79,62){\makebox(0,0)[cc]{$+$}}

\linethickness{0.3mm}
\put(74.38,61.38){\line(0,1){1}}
\put(44,64){\makebox(0,0)[cc]{$+$}}

\put(44,42){\makebox(0,0)[cc]{$-$}}

\put(46,53){\makebox(0,0)[cc]{$v''$}}

\linethickness{0.3mm}
\multiput(53,65)(0.12,-0.12){8}{\line(1,0){0.12}}
\linethickness{0.3mm}
\multiput(53,63)(0.12,0.12){8}{\line(1,0){0.12}}
\put(52,67){\makebox(0,0)[cc]{$i''=0$}}

\linethickness{0.3mm}
\put(74,52){\line(1,0){6}}
\put(80,52){\vector(1,0){0.12}}

\put(74.23,25.65){\makebox(0,0)[cc]{{\bf Figura 4.4}}}
\end{picture}

Il circuito in {\bf Figura 4.3} \`e costituito da soli resistori
lineari, circuiti aperti e corto-circuiti. Quindi pu\`o essere
rappresentato tramite un resistore di equivalente $R_{TH}$. Quindi

\[
v'=R_{Th}i
\]

Nel circuito in {\bf Figura 4.4}, invece, le sorgenti sono quelle dei
bipoli interne al bipolo in questione.
Indichiamo con $E_0=v''$ la tensione a vuoto.
Applichiamo la sovrapposizione degli effetti e abbiamo

\[
v=v'+v''
\]

Sostituendo abbiamo

\[
v=R_{Th}i+E_0
\]

\begin{definizione}
Il comportamento ai terminali di un qualsiasi bipolo costituito da
resistori lineari e generatori ideali \`e descrivibile attraverso un
generatore reale di tensione con valori della resistenza equivalente e
tensione a vuoto opportuni.
\end{definizione}

\ifx\JPicScale\undefined\def\JPicScale{1}\fi
\unitlength \JPicScale mm
\begin{picture}(90.62,62.5)(15,15)
\linethickness{0.3mm}
\multiput(69.25,60)(0.12,0.12){10}{\line(1,0){0.12}}
\linethickness{0.3mm}
\multiput(70.5,61.25)(0.12,-0.12){10}{\line(1,0){0.12}}
\linethickness{0.3mm}
\multiput(71.75,60)(0.12,0.12){10}{\line(1,0){0.12}}
\linethickness{0.3mm}
\multiput(73,61.25)(0.12,-0.12){10}{\line(1,0){0.12}}
\linethickness{0.3mm}
\multiput(74.25,60)(0.12,0.12){10}{\line(1,0){0.12}}
\linethickness{0.3mm}
\multiput(75.5,61.25)(0.12,-0.12){10}{\line(1,0){0.12}}
\linethickness{0.3mm}
\put(76.75,60){\line(1,0){1.25}}
\put(71.75,62.5){\makebox(0,0)[bl]{$R_{Th}$}}

\linethickness{0.15mm}
\put(68,60){\line(1,0){1.25}}
\linethickness{0.3mm}
\put(61,60){\line(1,0){8}}
\linethickness{0.3mm}
\put(77,60){\line(1,0){11}}
\put(88,60){\line(1,0){0.12}}
\put(88,60){\circle*{1.5}}
\linethickness{0.3mm}
\put(61,51){\line(0,1){9}}
\linethickness{0.3mm}
\put(61,47){\circle{8}}

\put(61,49){\makebox(0,0)[cc]{$+$}}

\put(60.95,44.38){\makebox(0,0)[cc]{$-$}}

\put(52,47.3){\makebox(0,0)[cc]{$E_0$}}
\linethickness{0.3mm}
\put(61,34){\line(0,1){9}}
\linethickness{0.3mm}
\put(61,34){\line(1,0){27}}
\put(88,34){\line(1,0){0.12}}
\put(88,34){\circle*{1.5}}
\put(87.5,46.88){\makebox(0,0)[cc]{$v$}}

\put(91.62,60){\makebox(0,0)[cc]{$+$}}

\put(91.62,34.1){\makebox(0,0)[cc]{$-$}}

\put(74.62,20.1){\makebox(0,0)[cc]{{\bf Figura 4.5} Circuito
    equivalente di Th\`evenin}}
\end{picture}

\end{document}
